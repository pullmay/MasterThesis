%%% Example for master's thesis (in English)
% \documentclass[english]{ampmt} % pdflatex
\documentclass[dvipdfmx,english]{ampmt} % dvipdfmx

%----for warning
% \DeclareFontShape{JT2}{mc}{m}{it}{<->ssub*mc/m/n}{}
% \DeclareFontShape{JT2}{mc}{m}{sl}{<->ssub*mc/m/n}{}
% \DeclareFontShape{JT2}{mc}{m}{sc}{<->ssub*mc/m/n}{}
% \DeclareFontShape{JT2}{gt}{m}{it}{<->ssub*gt/m/n}{}
% \DeclareFontShape{JT2}{gt}{m}{sl}{<->ssub*gt/m/n}{}
% \DeclareFontShape{JT2}{mc}{bx}{it}{<->ssub*gt/m/n}{}
% \DeclareFontShape{JT2}{mc}{bx}{sl}{<->ssub*gt/m/n}{}
% %
% \DeclareFontShape{JY2}{mc}{m}{it}{<->ssub*mc/m/n}{}
% \DeclareFontShape{JY2}{mc}{m}{sl}{<->ssub*mc/m/n}{}
% \DeclareFontShape{JY2}{mc}{m}{sc}{<->ssub*mc/m/n}{}
% \DeclareFontShape{JY2}{gt}{m}{it}{<->ssub*gt/m/n}{}
% \DeclareFontShape{JY2}{gt}{m}{sl}{<->ssub*gt/m/n}{}
% \DeclareFontShape{JY2}{mc}{bx}{it}{<->ssub*gt/m/n}{}
% \DeclareFontShape{JY2}{mc}{bx}{sl}{<->ssub*gt/m/n}{}

\newtheorem{theorem}{Theorem}
\newtheorem{definition}[theorem]{Definition}
\newtheorem{lemma}[theorem]{Lemma}
%% from here-------------------------
\makeatletter
\DeclareRobustCommand{\qed}{%
  \ifmmode % if math mode, assume display: omit penalty etc.
  \else \leavevmode\unskip\penalty9999 \hbox{}\nobreak\hfill
  \fi
  \quad\hbox{\qedsymbol}}
\newcommand{\openbox}{\leavevmode
  \hbox to.77778em{%
  \hfil\vrule
  \vbox to.675em{\hrule width.6em\vfil\hrule}%
  \vrule\hfil}}
\newcommand{\qedsymbol}{\openbox}
\newenvironment{proof}[1][\proofname]{\par
  \normalfont
  \topsep6\p@\@plus6\p@ \trivlist
  \item[\hskip\labelsep\itshape
    #1.]\ignorespaces
}{%
  \qed\endtrivlist
}
\newcommand{\proofname}{Proof}
\makeatother
%% upto here----------------------------
%--- Title ----------------------------------------------------------------------
\title[Study on a further improvement of \\ Maurer's universal statistical test]
      {Study on a further improvement of \\  Maurer's universal statistical test}
      % [title for spine (option)]{title}
%--- Supervisors ----------------------------------------------------------------
\supervisors{Ken UMENO}{Professor}             % First supervisor  {name}{title}
            {Atsushi IWASAKI}{Assistant Professor} % Second supervisor {name}{title}
            {}{}                               % Third supervisor  {name}{title}
%--- Author ---------------------------------------------------------------------
\author{Yasunari HIKIMA}
%-- Submission date -------------------------------------------------------------
\submissiondate{2020}{February}   % {year}{month}
%-- Width of a spine ------------------------------------------------------------
\setlength{\wdspine}{15mm}
%-- Number of output spines -----------------------------------------------------
\def\numberofspines{1}
%-- Abstract --------------------------------------------------------------------
\abstract{%
Maurer's universal statistical test is a hypothesis test for evaluating the randomness of a binary sequence and it is included in NIST SP 800-22 which is one of the most famous test suites. 
%
The test statistic of Maurer's test relates to the entropy of a tested sequence and hence the test can detect various defects of the sequence about randomness.
It has been reported that flipping a part of bits in a sequence makes Maurer's test more sensitive.
%
The test with flipping is called highly sensitive universal statistical test. To perform the highly sensitive test, the variance for the reference distribution is necessary, however, the theoretical value has not been derived.
%
In this thesis, we theoretically derive the variance for the reference distribution of the highly sensitive test and investigate the validity for testing randomness.
}
%-- Packages and definitions of your own macros ---------------------------------
\usepackage{amsmath,amssymb}
% \usepackage{amsthm}
\usepackage{newtxtext,newtxmath} % Times font
\newcommand{\rme}{\mathrm{e}}
\usepackage{here}
\usepackage{subcaption}
\usepackage{algorithm}
\usepackage{algpseudocode}
\usepackage{tikz}
\usepackage{enumerate}
\usetikzlibrary{patterns}
%-- Control of output -----------------------------------------------------------
\begin{document}
\ifoutputbody
%-- Inside cover, abstract and table of contents ---------------------------------
\makeinsidecover                % Inside cover
\makeabstract                   % Abstract
\maketoc                        % Table of contents
\setcounter{page}{1}
%-- Body -------------------------------------------------------------------------
\section{Introduction}\label{sec:introduction}
%-----------------------------------------------------------------------------------------------%
%-----------------------------------------------------------------------------------------------%
%-----------------------------------------------------------------------------------------------%
\subsection{Random sequence}%乱数列 %definition/application/generation
%乱数列は直感的には何のパターンや規則も持たない数字の列であると考えることができるが,その定義を明確に述べることは極めて難しい.なぜなら,ある系列が他のある系列よりもよりランダムであると述べる
Throughout this thesis, we only consider a binary sequence.
%
A random sequence is intuitively considered as a number sequence without any recognizable patterns or regularities, and it is not easy to define such properties mathematically.
The reader may consider a sequence to be random if its each bit is independent and symmetrically distributed, but this description is inconsistent with an intuitive definition. When we intuitively say ``random sequence'', this term is used to represent a specific sequence, not to represent a sequence of random variables.
\par
Several approaches to define a random sequence have been proposed.
In Algorithmic information theory, several definitions have been proposed based on Kolmogorov 
complexity \cite{kolmogorov1968three,chaitin1966length,chaitin1969length,chaitin1975theory}. 
Kolmogorov complexity of a finite sequence is defined as the minimal length of a program which generates the sequence with a given universal machine. Let $s$ be a finite sequence and $u$ be a universal machine. Then, Kolmogorov complexity of $s$ for $u$, $K_u(s)$ is written as 
\begin{align}
	K_u(s):= \min_{p:u(p)=s} r(p),	
\end{align}
where $p$ is a program that generates $s$ by $u$ and $r (p)$ is the length of the program $p$. 
%
Then, a finite binary sequence is regarded as random if its Kolmogorov complexity is almost equal to its length. In other words, a finite binary sequence which cannot be compressed is random. 
%
We can define randomness for given finite sequences by its notion, however, it is shown to be impossible to compute Kolmogorov complexity. Furthermore, this complexity depends on a choice of universal machine.
Other definitions for randomness which is associated with the definition by Kolmogorov complexity have been proposed by Demuth \cite{demuth1988remarks}, Martin-L\"{o}f \cite{martin1966definition,martin1971complexity} and Schnorr \cite{schnorr1971unified,schnorr1973process,schnorr1977general}.
\par
In cryptography, an approach based on a notion of indistinguishability has been proposed. 
Let $\{X_n\}_{n\geq 1}$ and $\{Y_n\}_{n\geq 1}$ be sequences of random variables. 
%
We say $\{X_n\}_{n\geq 1}$ and $\{Y_n\}_{n\geq 1}$ are computationally indistinguishable if for any probabilistic polynomial time algorithm $\mathcal{A}$ and positive polynomial $p$, there exists $k_0$ such that 
\begin{align}\label{eq:indistinguishable}
	\bigl| \mathrm{Pr}[\mathcal{A}(X_k) = 1] - \mathrm{Pr}[\mathcal{A}(Y_k) = 1] \bigr| < \frac{1}{p(k)},
\end{align}
for any $k>k_0$.
With this computational indistinguishability, we can define a cryptographically secure pseudo random number generator (CSPRNG). 
Let $g_n$ be a map from $\{0,1\}^n$ to $\{0,1\}^{h(n)}$, where $h(n)$ is a polynomial of $n$.
We say that a sequence of maps $g =\{g_n\}_{n\geq 1}$ is CSPRNG if this sequence satisfies the following three properties:
\begin{itemize}
	\item The relation $n < h(n)$ holds for any $n\geq 1$.
	\item For any $n\geq 1$ and input $x\in \{0,1\}^n$, there exists a deterministic algorithm for computing $g_n(x)$ in polynomial time depending on $n$.
	\item $\{g(U_n)\}_{n\geq 1}$ and $\{U_{h(n)}\}_{n\geq 1}$ are computationally indistinguishable, where $U_n$ is a random variable uniformly distributed on $\{0,1\}^n$.
\end{itemize}
Pseudo random numbers in cryptography are referred to as sequences generated by a CSPRNG. 
Under some assumptions, several algorithms for CSPRNG such as Blum--Blum--Shub algorithm \cite{blum1986simple} and Blum--Micali algorithm \cite{blum1984generate} have been proposed. 
However, the definition based on computationally indistinguishability also has obstacles such as:
\begin{itemize}
	\item For an arbitrary finite sequence, there exist CSPRNG and $x\in\{0,1\}^\ast$ such that the CSPRNG outputs the sequence when $x$ is input. Thus, it is meaningless to define ``randomness'' for a specific sequence by the above definition.
	\item It is impossible to verify whether a sequence satisfies Eq. (\ref{eq:indistinguishable}) for any probabilistic polynomial time algorithm.
	\item There exists no sequence satisfying the definition if $\mathrm{P}=\mathrm{NP}$.
\end{itemize}
To summarize, it is not easy to define randomness to a specific finite sequence.
%
\par
In spite of the difficulty of defining randomness, ``random sequences'' are widely used in many fields such as numerical simulations (e.g., Monte Carlo method), randomized algorithm (e.g., Simulated Annealing), and cryptography (e.g., key generation). 
In engineering applications, a sequence generated by a hardware (or physical) random number generator (HRNG) or a pseudo random number generator (PRNG) are extensively used.
%
Note that PRNG does not mean CSPRNG. 
An HRNG is a device for generating numbers from a physical process such as thermal noise in a transistor, whereas a PRNG is a deterministic algorithm for generating numbers.
%
In general, it is not easy to predict the output generated by an HRNG. On the other hand, a PRNG is important in practice because it generates a sequence much faster than an HRNG and a sequence can be reproductive by using the same initial value called seed. These advantages are useful in many situations such as numerical experiments.
%
However, both generators are nothing but generate a number sequence regarded as a random sequence approximately which is required for applications.
Hence, such sequences are necessary to be examined whether it satisfy the properties as ``random sequences'' or not.
%-----------------------------------------------------------------------------------------------%
%-----------------------------------------------------------------------------------------------%
%-----------------------------------------------------------------------------------------------%
%-----------------------------------------------------------------------------------------------%
\subsection{Tests for randomness}\label{subsec:1-2}%乱数検定
%乱数生成器から出力された系列が乱数が持つべき性質を満たしているのかを確認・テストすることは重要な問題である,
%工学的な応用の場面で用いられる乱数においては,統計的な性質を満たしていればよい.一方,暗号で用いられる乱数においては,激しい攻撃にも耐えられる必要がある
%暗号で用いられる系列は,予測不可能性という性質が必要である.すなわち,系列の部分列から次のビットが何であるかを予測することができない系列であることが求められる.しかしながら,与えられた系列がそのような性質を満たすのかどうかを調べることは困難である.
As seen in the previous subsection, it is a hard task to evaluate whether a finite binary sequence satisfies ``randomness'' or not. In practice, a statistical hypothesis test has been extensively considered to evaluate the randomness of a binary sequence. A number of statistical tests has been proposed. In most of the cases, the randomness is evaluated by multiple statistical tests since one statistical test is designed to detect the specific defect of a binary sequence and cannot detect other types of defects.
%
%前節で述べた定義を満たしているかどうかを確かめればよいと思われるかもしれないが,それらの定義は無限列に対して定義されているので現実的ではない.
%コルモゴロフ複雑性を用いてランダム性を判定すればよいと思われるかもしれないが,計算することができない.
%
%そこで,暗号用途に適した特性を持つような擬似乱数の定義として「暗号学的擬似乱数」が提唱された.それは次のようなものである.
%この定義と同値な条件として,Yaoによる次ビット予測テストによる特徴づけがある.実際に,このテストを行うことは現実的ではない.なぜなら....
%そこで通常は,有限個の統計的検定を複数個組み合わせたtest suiteによってrandomnessを判定することが考えられる. 
%
\par
There are some test suites such as TestU01 test suite \cite{l2007testu01}, the BSI (Bundesamt fr Sicherheit in der Informationstechnik) test suite \cite{schindler1999functionality,killmann2001proposal}, Marsaglia's DIEHARD test suite, Crypt-X statistical test suite \cite{caelli1992crypt} and FIPS 140-2 test suite \cite{fips2001140}.
\par
%
NIST Special Publication 800-22 (NIST SP 800-22) \cite{rukhin2001statistical,bassham2010sp} proposed by National Institute of Standards and Technology (NIST) is one of the standard statistical test suites that was originally used for selecting Advanced Encryption Standard (AES) \cite{rijmen2001advanced}. 
%
%帰無仮説は与えられた系列は乱数である.もっと詳しく言うと,与えられた系列は{0,1}^n上の一様分布から生成された,である.
NIST SP 800-22 consists of fifteen kinds of statistical tests, and the null hypothesis $\mathcal{H}_0$ is that a given binary sequence is truly random. We can regard a $n$-bit given binary sequence as a sample from a uniform distribution on $\{0,1\}^n$.
 % a given sequence of length $n$ is considered to be generated by a uniform distribution on $\{0,1\}^{n}$. 
Associated with the null hypothesis, the alternative hypothesis $\mathcal{H}_1$ is that the sequence is not random.
%NISTにおける検定方法の概略を述べる.まず,与えられた系列から検定統計量と呼ばれる実数を計算する.その値を理想的な乱数の場合における値と比較することによって,ランダム性を評価する.
% According to the latest version of NIST SP 800-22 Revision 1a \cite{bassham2010sp}, the process of every test is basically the same and described as follows. 
NIST SP 800-22 specifies the evaluating process as follows.
For a tested binary sequence of length $n$, a p-value is computed. If p-value is equal or larger than $\alpha$, the null hypothesis $\mathcal{H}_0$ is accepted, where $\alpha$ is the significance level of the test. Repeat the same procedure for $m$ sample sequences and obtain $m$ p-values. 
%
% NIST SP 800-22 recommends to perform the following two additional statistical tests for the $m$ p-values.
NIST SP 800-22 recommends to perform additional statistical tests for the $m$ p-values.
The following two tests are specified under the null hypothesis that $m$ p-values independently follow a uniform distribution in $[0,1]$.
% To examine more in detail, the following two additional two statistical tests, ``Proportion test'' and ``Uniformity test'' , are recommended in \cite{bassham2010sp} to execute:
\begin{enumerate}
  %各系列に対してp>=alphaなる確率は,1-alphaで与えらえる.この試行をm回行うと,平均でm(1-alpha),分散malpha(1-alpha)となる.
  \item (Proportion test) Let $m_p$ be the number of sequences whose p-value satisfies p-value $\geq \alpha$ for the given $m$ sequences. The null hypothesis is rejected if $m_p$ lies outside the significant interval $[m(1-\alpha)-\xi\sigma, m(1-\alpha)+\xi\sigma]$, where $m(1-\alpha)$ and $\sigma = \sqrt{m\alpha(1-\alpha)}$ is the expected value and  standard deviation of $m_p$, respectively.
  %
  % \item Let $m_p$ be the number of sequences whose p-value satisfies p-value $\geq \alpha$ for given $m$ sequences. If $m$ is large enough, $m_p/m$ can be approximated by a normal random variable, with expected value $1-\alpha$ and standard deviation $\sigma=\sqrt{\alpha(1-\alpha)/m}$. The null hypothesis $\mathcal{H}_0$ is rejected if $m_p/m$ lies outside the significant interval $[1-\alpha-3\sigma,1-\alpha+3\sigma]$.
  %
  \item (Uniformity test) The distribution of the $m$ p-values against the uniform distribution on $[0,1]$ is tested with a Chi-Square goodness of fit test in $k$ bins. This is again a statistical test, which yields a level-two p-value $p_T$. Given a significance level $\alpha_T$, the null hypothesis is rejected if $p_T \leq \alpha_T$.
\end{enumerate}
%
NIST SP 800-22 recommends to choose parameters as $m=1000,\,\alpha=0.01,\,\xi=3, \,k=10$ and $\alpha_T=0.0001$.
% Parameters are recommended in \cite{bassham2010sp} to choose as $m=1000,\,\alpha=0.01,\,k=10$ and $\alpha_T=0.0001$. 
% Repeat the same procedure for $m$ tested sequences and compute $m$ p-value. The recommendation is $\alpha=0.01$. Count the number of tested sequences such that p-value $\geq \alpha$ and define by $m_p$. Then, the assumption under of randomness, $m_p$ follows $\mathrm{Bin}(m,1-\alpha)$, which is approximated by $\mathcal{N}(m(1-\alpha),m\alpha(1-\alpha))$, and $m_p/m$approximately follows $\mathcal{N}(1-\alpha,\frac{\alpha(1-\alpha)}{m})$. Hence, the range of acceptable rate is determined by
% \begin{align}
%    1-\alpha-3\sqrt{\frac{\alpha(1-\alpha)}{m}} < \frac{m_p}{m} < 1-\alpha+3\sqrt{\frac{\alpha(1-\alpha)}{m}},
% \end{align} 
% and it is concluded that tested sequences are non-random if the above proportion does not hold. In the next place, the distribution of p-values are considered. The p-value of truly random sequences distributes uniformly over $(0,1)$. 
%
% \par
% There are TestU01 test suite \cite{l2007testu01}, the BSI (Bundesamt fr Sicherheit in der Informationstechnik) test suite \cite{schindler1999functionality,killmann2001proposal}, Marsaglia's DIEHARD test suite, Crypt-X statistical test suite \cite{caelli1992crypt} and FIPS 140-2 test suite \cite{fips2001140} as an example of other statistical test suits for evaluating a random number generator.
% Other statistical test suites for evaluating a random number generator are TestU01 test suite \cite{l2007testu01}, the BSI (Bundesamt fr Sicherheit in der Informationstechnik) test suite \cite{schindler1999functionality,killmann2001proposal}, Marsaglia's DIEHARD test suite, Crypt-X statistical test suite \cite{caelli1992crypt}, FIPS 140-2 test suite \cite{fips2001140}, etc. 
%
% Note that the binary sequence being tested cannot be regarded as random even if it passes all the 15 statistical tests. 
%
% Since it is hard to define the ``randomness'', hypothesis tests are carried out by setting several evaluation criteria.
% Consequently, the randomness of sequences cannot be guaranteed by the statistical tests, though non-randomness sequences can be rejected.
% 
%-----------------------------------------------------------------------------------------------%
%-----------------------------------------------------------------------------------------------%
%-----------------------------------------------------------------------------------------------%
% \subsection{Notation}
% The notation $\mathbb{N},\,\mathbb{R}$ are the set of natural numbers and the set of real numbers, respectively.
% A symbol $B$ denotes the set $\{0,1\}$ and $B^n$ denotes the set of binary strings of length $n$. For a binary sequence $x^n=x_1,x_2,\dots,x_n$, a symbol $b_k$ denotes the $k$-th substring $b_k=x_{L(k-1)+1},x_{L(k-1)+2},,\dots,x_{Lk}$ 
%-----------------------------------------------------------------------------------------------%
%-----------------------------------------------------------------------------------------------%

%-----------------------------------------------------------------------------------------------%
\subsection{Outline}%概要
This thesis is organized as follows. 
%
\par
%
In Section \ref{sec:universal}, we introduce the statistical tests proposed by Maurer in 1992 \cite{maurer1992universal} and by Coron in 1999 \cite{coron1999security}. These statistical tests are referred to as ``Maurer's universal test'' and ``Coron's universal test'', respectively. 
Coron's universal test is an improvement of Maurer's universal test.
%
We also introduce the statistical test proposed by Yamamoto and Liu in 2016 \cite{yamamoto2016highly}. The test is referred to as ``highly sensitive universal statistical test'' and the test is constructed on the basis of Maurer's universal test and Coron's universal test.
%
In the highly sensitive test, a tested sequence is converted to another binary sequence as each bit of a tested sequence is stochastically flipped under a certain distribution.
%
It is suggested in \cite{yamamoto2016highly} that the converting make the test more sensitive.
%
\par
In Section \ref{sec:distribution}, we derive one and two dimensional distributions. We need these distribution to derive the variance for the reference distribution of the highly sensitive test. In existing literature, the theoretical results for a truly random sequence without flipping have been obtained. However, we cannot apply the results directly to the highly sensitive test since a tested sequence is biased.
%
\par
In Section \ref{sec:3}, we derive the theoretical variance for the reference distribution of the highly sensitive test. 
We also show some results of experiments in this section. 
Firstly, we show the variance can be computed accurately by the derived equation. 
Secondly, we show the fitted curve for computing the variance for effectively. 
Thirdly, we show the difference between the highly sensitive test with proposed parameter and the test with the existing value. 
%
\par
In Section \ref{sec:conclusion}, we conclude this thesis.
\newpage
\section{Universal statistical test}\label{sec:universal}
%The purpose of the test is to detect whether or not the sequence can be significantly compressed without loss of information. A compressible sequence is considered to be non­ random.
% In this section, we first state the universal statistical test proposed by Maurer \cite{maurer1992universal} and by Coron \cite{coron1999security}. In the next place, we show the Highly Sensitive Universal Statistical Test proposed by Yamamoto and Liu \cite{yamamoto2016highly}.
In this section, we introduce ``Maurer's universal statistical test'' \cite{maurer1992universal}, ``Coron's universal statistical test' \cite{coron1999security} and ``highly sensitive universal statistical test'' \cite{yamamoto2016highly}.
%
\subsection{Maurer's universal statistical test}
Maurer's universal test is one of the tests included in NIST SP 800-22 \cite{rukhin2001statistical,bassham2010sp}.
% A statistical test for the randomness of a binary sequence has been proposed by Maurer \cite{maurer1992universal}, and it is included in NIST SP 800-22 \cite{rukhin2001statistical,bassham2010sp}. 
Maurer's universal statistical test aims at detecting non-randomness based on the test statistic value which is relating to the source's entropy, and the non-randomness is evaluated whether the computed entropy attains the maximum or not. 
%
% Unlike the other types of statistical tests which is designed to detect specific defects, Maurer's universal test is able to detect the wide range of statistical defects consists of that those can be modeled by an ergodic stationary source with finite memory. 
Unlike other types of statistical tests which is designed to detect specific defects, Maurer's universal test is able to detect a wide range of statistical defects.
%
%ergodic statistical sourceから生成された系列というのは,各ビットがある確率に従って生成される.その確率は,前に出力されたビット列から決まる.もし,理想的な乱数の場合,前の系列が何であれ,それぞれ1/2の確率で0と1が生成される.
\par
Let us consider an information source $S$ which generates a sequence $U_1,U_2,\dots$. For each $i$, we regard $U_i$ as a sample from a random variable.
 % of random variables. 
A source $S$ is called a finite memory source if there exists a positive integer $M$ such that the conditional probability of $U_n$, given $U_1,\dots,U_{n-1}$, depends only on the most previous $M$ bits, i.e.,
\begin{align}\label{eq:memory}
	P_{U_n\mid U_{n-1}\dots U_{1}}(u_n \mid u_{n-1}\dots u_{1}) = P_{U_n\mid U_{n-1}\dots U_{n-M}}(u_n \mid u_{n-1}\dots u_{n-M}),
\end{align}
for $n>M$ and for every binary sequence $u_1,\dots,u_n\in\{0,1\}^n$. The smallest $M$ satisfying Eq. (\ref{eq:memory}) is called the memory of the source. The probability distribution of $U_n$ is thus determined by the source's state $\Lambda_n = U_{n-M},\dots,U_{n-1}$ at time $n$. Let $\Lambda_1 = U_0 \dots U_{-M+1}$ be the initial state where $U_{-M+1}\dots U_0$ are dummy random variables. Then, the source is called \textit{stationary}, if the information source $S$ satisfies the following relation in addition to Eq. (\ref{eq:memory}),
\begin{align}
	P_{U_n\mid\Lambda_n}(u\mid \lambda) = P_{U_1\mid\Lambda_1}(u\mid \lambda),
\end{align}
for all $n>M$, $u\in\{0,1\}$ and $\lambda \in \{0,1\}^M$. 
%
% Therefore, the probability of taking each bit to be ``$1$'' or ``$0$'' is given by a probability and it depends only on the most previous finite sequence when a sequence is generated by a stationary source. 
%
Therefore, it depends only on the most previous $M$-bit sequence when a sequence is generated by a stationary source.
%
On the other hand, each bit is independent of the previous sequence, and the probability of taking ``$1$'' or ``$0$'' is exactly the same as $\frac{1}{2}$ if we say a sequence is truly random in intuitive sense.
%
% For a binary sequence generated by an ergodic stationary source, each bit takes ``$1$'' or ``$0$'' in some probability. The probability of taking $n$-th bit to be ``$1$'' or ``$0$'' is depend only on the most previous bits, that is, the following relation holds.
% \begin{align}
% 	P_{U_n\mid U_{n-1}\dots U_{1}}(u_n \mid u_{n-1}\dots u_{1}) = P_{U_n\mid U_{n-1}\dots U_{n-M}}(u_n \mid u_{n-1}\dots u_{n-M}),
% \end{align}
% where $M$ is called memory of the source. For a truly random sequence, the probability of taking each bit to be $1$ or $0$ is independent of previous bits, and the probability of taking each bit to be $1$ or $0$ is exactly the same as $\frac{1}{2}$. 
% any one of the general class of statistical defects which can be modeled by ergodic stationary source with finite memory. 
%
\par
%
The formulation of Maurer's universal test is motivated by the universal source coding algorithms that has been proposed in \cite{elias1987interval,willems1989universal} and the procedure is described as follows. In the following, let $B$ be the set $\{0,1\}$, and $x^n = x_1,x_2,\dots,x_n \in B^n$ be a binary sequence of length $n$, where $x_i\in B$. The test takes as input three positive integers $L,\,Q$ and $K$, and a binary sequence $s^n \in B^n$ generated by a tested source. The sequence is divided into adjacent non-overlapping blocks of length $L$. Then, the first $Q$ blocks ($L\times Q$-bits) are used for initialization, and the remaining $K$ blocks ($L\times K$-bits) are used for the test. Without loss of generality, we can assume that $n=L\times(Q+K)$ holds\footnote{%
If the relation $n=L\times(Q+K)$ does not hold, then let $K$ be $\left\lfloor \frac{n}{L} \right\rfloor - Q$.
}. 
%
Let $b_k(x^n)$ be the $k$-th block of $x^n$, i.e., $b_k(x^n) = x_{L(k-1)+1}, x_{L(k-1)+2}, \dots, x_{Lk}$. Considering the situation that the $(n-m)$-th block takes the same value with the $n$-th block and the blocks from $(n-m+1)$-th to $(n-1)$-th blocks do not take the value, we define an integer-valued variable $A_n(x^n)$ as $m$. 
% The mapping from a binary sequence $x^n$ to a integer-valued variable $A_n(x^n)$ is given by
% \begin{align}
% 	A_{n}(x^n) = \left\{ \begin{array}{ll}
% 	n, \:\: &\text{if} \:\: b_{n-m} \neq b_n \:\: \text{for} \:\: 1 \leq m \leq n-1, \\
% 	\min \{ m\in\mathbb{N} \mid m \geq 1, b_{n-m} = b_n \}, \:\: &\text{otherwise}.
% 	\end{array} \right.
% \end{align}
% for $n=Q+1,Q+2,\dots,Q+K$.
%
Figure \ref{fig:A_n_example} illustrates an example for $L=3$. The test function $f_M$, which maps a binary sequence to a real number, is defined by 
% The test statistic value $f_M:B^{n} \to \mathbb{R}$ is defined as the average of the binary logarithm of $A_{Q+1},A_{Q+2},\dots,A_{Q+K}$, i.e., 
\begin{align}\label{eq:fM}
	f_M(x^n) = \frac{1}{K} \sum_{n=Q+1}^{Q+K} \log_2 A_n(x^n),
\end{align}
where $A_n(x^n)$ is defined by
\begin{align}\label{eq:An}
	A_{n}(x^n) = \left\{ \begin{array}{ll}
	n, \quad \text{if} \:\: b_{n-m}(x^n) \neq b_n(x^n) \:\: \text{for} \:\: 1 \leq m \leq n-1, \\
	\min \{ m\in\mathbb{N} \mid m \geq 1,\, b_{n-m}(x^n) = b_n(x^n) \}, \quad \text{otherwise},
	\end{array} \right.
\end{align}
for $n=Q+1,Q+2,\dots,Q+K$.
% The mapping from $f_M(x^n)$ to the p-value $p_M$ is given by
% \begin{align}
% 	% p\mathchar`- \mathrm{value}_M 
% 	p_M
% 	= \mathrm{erfc} \left( \left| \frac{f_M(x^n) - \mathbb{E}[f_M(R^n)]}{\sqrt{2} \sigma_M} \right| \right),
% \end{align}
% where $R^n$ denotes a binary sequence of length $n$ generated by a BSS.
% where $\mathbb{E}[f_M(R^n)]$ denotes the expectation value, $\sigma_M^2$ denotes the variance of $f_M(R^n)$ under null hypothesis, and the $\mathrm{erfc}$ is the complementary error function defined by
% \begin{align}
%    \text{erfc}(z) = \frac{2}{\sqrt{\pi}} \int_{z}^{\infty} \mathrm{e}^{-t^2}\, \mathrm{d}t.
% \end{align}
%-------------------------------------------------------------------
\begin{figure}
\centering
\begin{tikzpicture}
\draw (-1.5,0)--(-1.0,0);
\draw (-1.5,1)--(-1.0,1);
\draw (-1.0,0) rectangle (0,1) node at (-0.5, 0.25) [above] {$100$};
\draw (0,0) rectangle (1,1) node at (0.5, 0.25) [above] {$001$};
\filldraw [draw=black, fill=pink] (1,0) rectangle (2,1) node at (1.5, 0.25) [above] {$000$} node (A) at (1.5, 1.0) [above] {$b_{n-6}$};
\filldraw [draw=black, fill=white] (2,0) rectangle (3,1) node at (2.5, 0.25) [above] {$101$} ;
\filldraw [draw=black, fill=white] (3,0) rectangle (4,1) node at (3.5, 0.25) [above] {$110$};
\filldraw [draw=black, fill=white] (4,0) rectangle (5,1) node at (4.5, 0.25) [above] {$111$};
\filldraw [draw=black, fill=white] (5,0) rectangle (6,1) node at (5.5, 0.25) [above] {$100$};
\filldraw [draw=black, fill=white] (6,0) rectangle (7,1) node at (6.5, 0.25) [above] {$001$};
\filldraw [draw=black, fill=pink] (7,0) rectangle (8,1) node at (7.5, 0.25) [above] {$000$} node (AA) at (7.5, 1.0) [above] {$b_n$};
\draw (8,0) rectangle (9,1) node at (8.5, 0.25) [above] {$101$};
\draw (9,0) rectangle (10,1) node at (9.5, 0.25) [above] {$011$};
\draw (10,0)--(10.5,0);
\draw (10,1)--(10.5,1);
%
\draw[<->] (A) to[bend left=15] node [above] {\small Match} (AA);
\end{tikzpicture}
\caption{An example of the situation of $A_n=6$ for $L=3$.}
\label{fig:A_n_example}
\end{figure}
%-------------------------------------------------------------------
%
\par
%
In order to evaluate the non-randomness of a given binary sequence, it is necessary to derive the expected value and the variance of the reference distribution for a truly random sequence. Note that a truly random sequence is a binary sequence generated by a uniform distribution on $\{0,1\}^n$. 
% It has been shown in \cite{maurer1992universal} that the following relations hold under admissible assumption $Q\to\infty$.
Under the assumption $Q\to\infty$, the expectation and the variance are given by
% a binary sequence of length $n$ generated by a binary symmetric source (BSS) which means a truly random sequence. Let $R^n$ be a binary random sequence of length $n$ generated by a BSS. It has been show in \cite{maurer1992universal} that $\mathbb{E}[f_M(R^n)]$ and $\sigma_M^2$ are obtained as
\begin{align}
	\mathbb{E}[f_M(R^n)]  &= 2^{-L}\sum_{i=1}^{\infty}(1-2^{-L})^{i-1} \log_2 i \label{eq:mean_maurer},\\
	\sigma_M^2 &= c_M(L,K)^2 \times \frac{\mathrm{Var}[\log_2 A_n(R^n)]}{K} \label{eq:sigma_maurer},
\end{align}
%
where $R^n$ denotes a truly random sequence of length $n$. 
% represents the correction factor by which $\sigma_M^2$ is reduced compared with what it would be if the terms $A_n(R^n)$ were independent. It has been proposed in \cite{maurer1992universal} that $c_M(L,K)$ for $K\geq 2^L$ can be approximated by
%
In \cite{maurer1992universal}, the following approximation is empirically proposed:
% It is necessary to derive the expected value $\mathbb{E}[f_M(R^n)]$ and the variance $\sigma_M^2$ for calculating p-value from Eq. (). It is shown in \cite{maurer1992universal} that $\mathbb{E}[f_M(R^n)]$ and $\sigma_M^2$ are obtained as
% \begin{align}
% 	\mathbb{E}[f_M(R^n)] &= \mathbb{E}[\log_2 A_n] = 2^{-L}\sum_{i=1}^{\infty}(1-2^{-L})^{i-1} \log_2 i,\\
% 	\sigma_M^2 &= \mathrm{Var}[f_M(R^n)] = c_M(L,K)^2 \times \frac{\mathrm{Var}[\log_2 A_n]}{K},
% \end{align}
% where $c_M(L,K)$ is the correction factor by which $\sigma_M^2$ is reduced compared with what it would be if the terms $A_n$ were independent. It is suggested in \cite{maurer1992universal} that $c_M(L,K)$ can be approximated as
\begin{align}\label{eq:cM_maurer}
	c_M(L,K) \simeq 0.7 - \frac{0.8}{L} + \left( 1.6 + \frac{12.8}{L} \right) K^{-4/L}.
\end{align}
The approximation above has been obtained by numerical simulations. 
%
In \cite{coron1998accurate}, the accurate expression of $c_M(L,K)$ has been obtained theoretically. 
% However, it has a high computational cost that an approximated value $c_M(L,K)$ for $K\geq33\times2^L$ is given as follows:
However, it requires much cost to compute the $c_M(L,K)$ for given $L$ and $K$, and hence an approximation of the theoretical form is given as follows:

% by deriving the joint distribution of $(A_n(R^n),\,A_{n+k}(R^n))$
 % and $c_M(L,K)$ for $K\geq33\times2^L$ is shown to be approximated by
% It also has been suggested in \cite{coron1998accurate} that the precise value of $c_M(L,K)$ for $K\geq33\times2^L$ is expressed as
\begin{align}\label{eq:cM_coron}
	c_M(L,K)^2 \simeq d_M(L) + \frac{e_M(L)\times2^L}{K}, 
\end{align} 
where $d_M(L)$ and $e_M(L)$ are listed in \cite{coron1998accurate} for $L=3,4,\dots,16$. 
The approximation in Eq. (\ref{eq:cM_coron}) is accurate for $K\geq 33\times 2^L$ in practice.
%
Note that $\mathrm{Var}[\log_2 A_n]$ in Eq. (\ref{eq:sigma_maurer}) can be computed by the definition of variance by
\begin{align}
 	\mathrm{Var}[\log_2 A_n(R^n)] = 2^{-L}\sum_{i=1}^{\infty}(1-2^{-L})^{i-1} (\log_2 i)^2 - (\mathbb{E}[f_M(R^n)])^2.
\end{align}
% where $\mathbb{E}[f_M(R^n)]$ is given by Eq. (\ref{eq:mean_maurer}).
% the values are listed in \cite{maurer1992universal} for $L=1,2,\dots,16$.
%
\par
%
To implement the test, it is necessary to set the parameters. In \cite{maurer1992universal}, the study recommends to set parameters $L\,Q$ and $K$ as $6\leq L,\, \leq 16$, $Q \geq 10 \times 2^L$ and $K \geq 1000\times 2^L$, respectively. 
%
The study also insists that rejection rate $\rho$ should be chosen as $\rho \in [0.001, 0.01]$. 
%
Then, it is concluded that the null hypothesis of Maurer's test\footnote{A null hypothesis of Maurer's test is that a binary sequence of length $n$ follows a uniform distribution on $\{0,1\}^n$} 
is rejected if either $f_M(x^n)<t_1$ or $f_M(x^n)>t_2$ holds, where the thresholds $t_1$ and $t_2$ are written as
\begin{align}\label{eq:ysigma}
\begin{split}
	t_1 &= \mathbb{E}[f_M(R^n)] - y\sigma_M, \\
	t_2 &= \mathbb{E}[f_M(R^n)] + y\sigma_M.
\end{split}
\end{align}
Using the complementary error function $\mathrm{erfc}$ defined by
\begin{align}
   \text{erfc}(z) = \frac{2}{\sqrt{\pi}} \int_{z}^{\infty} \mathrm{e}^{-u^2} \, \mathrm{d}u,
\end{align}
the value $y$ in Eq. (\ref{eq:ysigma}) is given as $\mathrm{erfc}(y)=\frac{\rho}{2}$. 
% where $\mathrm{erfc}(y)=\frac{\rho}{2}$. Note that $\mathrm{erfc}$ is the complementary error function defined by
% \begin{align}
%    \text{erfc}(z) = \frac{2}{\sqrt{\pi}} \int_{z}^{\infty} \mathrm{e}^{-u^2} \, \mathrm{d}u.
% \end{align}
% In the above relation, the standard deviation $\sigma_M$ is given by Eqs. (\ref{eq:sigma_maurer}) or (\ref{eq:cM_coron}) and $y$ is the number of standard deviations that the computed test statistic value $f_M(x^n)$ is allowed to be away from the expected value $\mathbb{E}[f_M(R^n)]$. The parameter $y$ should be chosen such that $\mathcal{N}(-y)=\frac{\rho}{2}$ where $\mathcal{N}:\mathbb{R}\to\mathbb{R}$ is the integral of the normal density function defined by
% Note that $y$ in the above equation satisfies
% \begin{align}
% 	\frac{1}{\sqrt{2\pi}}\int_{-\infty}^{-y} \mathrm{e}^{-\frac{u^2}{2}} \, \mathrm{d}{u} = \frac{\rho}{2}.
% \end{align}
Notice that it is implicitly assumed that $f_M(R^n)$ follows a normal distribution.
%
\par
In NIST SP 800-22, the non-randomness is evaluated by the p-value shown as
\begin{align}
	p\mathchar`- \mathrm{value}_M 
	% p_M
	= \mathrm{erfc} \left( \left| \frac{f_M(x^n) - \mathbb{E}[f_M(R^n)]}{\sqrt{2} \sigma_M} \right| \right),
\end{align}
where $\mathrm{erfc}$ is the complementary error function defined by
% \begin{align}
%    \text{erfc}(z) = \frac{2}{\sqrt{\pi}} \int_{z}^{\infty} \mathrm{e}^{-u^2} \, \mathrm{d}u.
% \end{align}
Then, the null hypothesis of Maurer's test is rejected if $p\mathchar`- \mathrm{value}_M < \alpha$, where $\alpha$ is a significance level.
%
% In NIST SP 800-22 \cite{rukhin2001statistical,bassham2010sp}, the non-randomness of a given binary sequence $x^n\inB^n$ is evaluated by the p-value defined by
% % the mapping from $f_M(x^n)$ to the p-value is given by
% \begin{align}
% 	p\mathchar`- \mathrm{value}_M 
% 	% p_M
% 	= \mathrm{erfc} \left( \left| \frac{f_M(x^n) - \mathbb{E}[f_M(R^n)]}{\sqrt{2} \sigma_M} \right| \right),
% \end{align}
% where $\mathrm{erfc}$ is the complementary error function defined by
% \begin{align}
%    \text{erfc}(z) = \frac{2}{\sqrt{\pi}} \int_{z}^{\infty} \mathrm{e}^{-u^2} \, \mathrm{d}u.
% \end{align}
% %
% In the case that the computed p-value is less than the significance level $\alpha$, it is concluded that a binary sequence being tested is not random. 
%-------------------------------------------------------------------
%-------------------------------------------------------------------
%-------------------------------------------------------------------
\par
There is much concern in the asymptotic relation between the Maurer's test statistic and the source's per-bit entropy. In \cite{maurer1992universal}, the expected value of the test statistic for a truly random sequence $\mathbb{E}[f_M(R^n)]$ is closely related to the entropy of blocks. It has been shown that the following relation holds:
\begin{align}\label{eq:maurer_asymptotic_R}
	\lim_{L\to\infty} \left[ \mathbb{E}[f_M(R^n)] -L \right] = C,
\end{align}
where $C$ is a constant whose value is equal to $-\frac{\ln 2}{\gamma} \simeq -0.8327$ and $\gamma$ is Euler's constant \cite{hardy1979introduction}. We provide the proof of $C=-\frac{\ln 2}{\gamma}$ is given in Appendix \ref{appendix:A}. 
%
Let us consider a binary sequence $U_{\mathrm{BMS}_p}^n$ generated by a binary memoryless source $\mathrm{BMS}_p$. Note that a sequence $U_{\mathrm{BMS}_p}^n$ follows a distribution on $\{0,1\}^n$ taking each bit to be ``$1$'' with probability $p\in (0,1)$ independently. 
% We can show that the relation $K_L=L\times H(p)$ holds for a binary memoryless source.
%
For a binary sequence $U_{\mathrm{BMS}_p}^n \in B^n$, the following relation
\begin{align}
	\lim_{L\to\infty} \left[ \mathbb{E}[f_M(U_{\mathrm{BMS}_p}^n)] -L\times H(p) \right] = C
\end{align}
holds for any $p \in (0,1)$, where $H$ is the binary entropy function corresponding to $\mathrm{Pr}[X=1]=p$. Here, $X$ is a random variable. 
%
In \cite{maurer1992universal}, a similar result has been studied to show for a binary sequence $U_s^n$ generated by every ergodic stationary source $S$. The study in \cite{coron1998accurate} develops the idea and proves that the following relation holds:
% It has been conjectured in \cite{maurer1992universal} that the similar result would hold for every binary ergodic stationary source $S$ with output sequence $U_s^n$,
\begin{align}
	\lim_{L\to\infty} \left[ \mathbb{E}[f_M(U_s^n)] - K_L \right] = C,
\end{align}
where $K_L$ is equivalent to the entropy of $L$ bit blocks defined by
\begin{align}\label{eq:K_L}
	K_L = -\sum_{b \in B^n} \mathrm{Pr}[b] \log_2 \mathrm{Pr}[b].
\end{align}
Other asymptotic relations between Maurer's test statistic and a source's entropy can be found in \cite{wegenkittl2001entropy,choe2000average,abadi2004version,kim2014estimation,kim2018low}.
% It has been proven in \cite{coron1999security} that test statistic given in Eq. (\ref{eq:fM}) is closely related to the source's entropy.
% % The test statistic is closely related to the source's per-bit entropy.
% It has been proven in \cite{maurer1992universal} that the following relation holds for a binary binary symmetric source generated by STP.
% \begin{align}
% 	\lim_{L\to\infty} \left[ \mathbb{E}[f_M(R^n)] -L \right] = C = -\frac{\ln 2}{\gamma}.
% \end{align}
% where $C$ is a constant whose value is proven to be equal to $-\frac{\ln 2}{\gamma}$. The proof is given in Appendix.
% It has been conjectured that the same relation holds for any binary ergodic stationary source $U_s^n$.
% \begin{align}
% 	\lim_{L\to\infty} \left[ \mathbb{E}[f_M(U_s^n)] -L\times H(q) \right] = C.
% \end{align}
% However, it has been in \cite{coron1998accurate} proven that the above relation does not hold, and refuse that.
% \begin{align}
% 	\lim_{L\to\infty} \left[ \mathbb{E}[f_M(U_s^n)] - H_s \right] = C.
% \end{align}
% where $H_s$ is the entropy.
%-------------------------------------------------------------------
%-------------------------------------------------------------------
% \par
% It has proven in \cite{maurer1992universal} that $\mathbb{E}[f_M(R^n)]-L$ converges to the constant $C=-0.832746$ as $L\to\infty$, and conjectured that the following relation holds for any $q\in (0,1)$. 
% \begin{align}
% 	\lim_{L\to\infty} \left[ \mathbb{E}[f_M(U_S^n)] - L\times H(q) \right] = C.
% \end{align}
% However, it has proven in \cite{coron1999security} that the conjecture does not correct in the case of $q \neq 0.5$.
%-------------------------------------------------------------------%
%-------------------------------------------------------------------%
% \begin{algorithm}[t]
% \caption{The procedure of Maurer's universal statistical test based on NIST SP 800-22}
% % \label{alg:1}
% \begin{algorithmic}[1]
% \State Set the positive integers $L,\, Q$ and $K$ as a parameter, a significance level $\alpha$, and the number of sample sequences $m$.
% \State Divide a binary sequence $x^n$ into adjacent non-overlapping blocks of length $L$, and compute $A_n$ for $n=Q+1,Q+2,\dots,Q+K$ from Eq.(\ref{eq:An}). \label{state:divide}
% % The recommendation is $L$ as between $6$ and $16$, $Q \geq 10\times2^L$, $K \geq 1000\times2^L$. 
% % \State Divide a binary sequence $x^n$ into adjacent non-overlapping blocks of length $L$ and calculate $A_n$ from Eq. (\ref{eq:An}) for $n=Q+1,Q+2,\dots,Q+K$.
% \State Compute a test statistic from Eq.(\ref{eq:fM}).
% \State Compute a $p \, \mathchar`- \mathrm{value}$ from Eq. (\ref{eq:pM}). \label{state:pvalue}
% % \State If the computed p-value is less than $\alpha$, then conclude the sequence is non-random. Otherwise, conclude the sequence is random.
% \State Do \ref{state:divide} to \ref{state:pvalue} for $m$ sample sequences, and obtain $m$ p-values.
% \State (Second-level test Ⅰ: Proportion of sequences passing a test)
% \State (Second-level test Ⅰ: Proportion of sequences passing a test)
% \end{algorithmic}
% \end{algorithm}
%-------------------------------------------------------------------%
%-------------------------------------------------------------------%
%-------------------------------------------------------------------%
%-------------------------------------------------------------------%
%-------------------------------------------------------------------%
%-------------------------------------------------------------------%
%-------------------------------------------------------------------%
%-------------------------------------------------------------------%
%-------------------------------------------------------------------%
%-------------------------------------------------------------------%
%-------------------------------------------------------------------%
%-------------------------------------------------------------------%
%-------------------------------------------------------------------%
%-------------------------------------------------------------------%
%-------------------------------------------------------------------%
\subsection{Coron's universal statistical test}
As seen in the previous subsection, Maurer's universal test can detect a wide range of statistic defects modeled by an ergodic statistic source with finite memory, however, the test only provides an asymptotic measure of the source's entropy. 
%
To address the problem, a modified version of Maurer's universal test called ``Coron's universal test'' has been proposed in \cite{coron1999security}. In this test, the expectation of test statistic value is exactly equal to the source's entropy. The main procedure is the same as Maurer's test except for the test statistic. The test function $f_C$, which maps a binary sequence to a real number, is defined by 
% The test function $f_C:B^n\to\mathbb{R}$ is defined as
\begin{align}\label{eq:fC}
	f_C(x^n) = \frac{1}{K} \sum_{n=Q+1}^{Q+K} g(A_n(x^n)),
\end{align}
where $A_n(x^n)$ is defined by Eq. (\ref{eq:An}) and $g:\mathbb{N}\to\mathbb{R}$ is given by
% the function $g:\mathbb{N}\to\mathbb{R}$ is defined as
\begin{align}\label{eq:function_g}
	g(m) = (\log_2 \mathrm{e}) \sum_{k=1}^{m-1}\frac{1}{k},
\end{align}
for $m\geq 2$. Note that we set $g(1)=0$.
%
For a binary sequence $U_s^n$ generated by an ergodic stationary source $S$, the expected value is exactly equal to the entropy of $L$ blocks of $S$, i.e.,
\begin{align}\label{eq:coron_expected_value_for_U}
	\mathbb{E}[f_C(U_s^n)] = K_L,
\end{align}
where $K_L$ corresponds to the entropy of blocks given by Eq. (\ref{eq:K_L}).
%
\par
It is necessary to derive the expected value and the variance of the reference distribution just like Maurer's test. From Eq. (\ref{eq:coron_expected_value_for_U}), the expected value for $R^n$ is obtained by
\begin{align}
\label{eq:coron_expected_value_truly_random}
	\mathbb{E}[f_C(R^n)] &= L, 
\end{align}
since $K_L=L$ when the source is the binary symmetric source. 
% It has been obtained \cite{coron1999security} that the variance can be given by
The variance is also obtained as
\begin{align}
	\sigma_C^2 &= c_C(L,K)^2 \times	\frac{\mathrm{Var}[g(A_n)]}{K}.
\end{align}
In \cite{coron1999security}, the following approximation is empirically proposed as
\begin{align}
	c_C(L,K) \simeq d_C(L) + \frac{e_C(L)\times 2^L}{K},
\end{align}
where $d_C(L)$ and $e_C(L)$ are listed in \cite{coron1999security} for $L=3,4,\dots,16$.
% It has been obtained in \cite{coron1999security} that the following relations hold for a truly random sequence $R^n$.
% \begin{align}
% 	\mathbb{E}[f_C(R^n)] &= L, \label{eq:coron_expected_value_truly_random}\\
% 	\sigma_C^2 &= c_C(L,K)^2 \times	\frac{\mathrm{Var}[g(A_n)]}{K},
% \end{align}
% where $c_C(L,K)$ is similar to the one of $c_M(L,K)$ given in Eq. (\ref{eq:cM_coron}) and approximated for $K\geq 33\times 2^L$ by
% \begin{align}
% 	c_C(L,K)^2 \simeq d_C(L) + \frac{e_C(L)\times 2^L}{K},
% \end{align}
% where $d_C(L)$ and $e_C(L)$ are listed in \cite{coron1999security} for $L=3,4,\dots,16$. 
Note that $\mathrm{Var}[g(A_n)]$ can be calculated by the definition of variance as
\begin{align}
\begin{split}
	\mathrm{Var}[g(A_n)] 
	&= \mathbb{E}[\{g(A_n)\}^2] - \left( \mathbb{E}[g(A_n)] \right)^2 \\
	&=2^{-L} \sum_{i=2}^{\infty} (1-2^{-L})^{i-1} \left( \sum_{k=1}^{i-1} \frac{\log_2 \mathrm{e}}{k} \right)^2 -L^2.
\end{split}
\end{align}
To obtain the second equality in the above equations, we use the relation $\mathbb{E}[g(A_n)]=\mathbb{E}[f(x^n)]=L$.
% The expectation of the test function defined by Eq. () as input a sequence $U_S^n$ generated by an ergodic stationary source $S$ is equal to the entropy of $L$ bit blocks of $S$.
% %
% The mapping from $f_C(x^n)$ to the p-value $p_C$ is given by
% \begin{align}
% 	% p\mathchar`- \mathrm{value}_M 
% 	p_C
% 	= \mathrm{erfc} \left( \left| \frac{f_C(x^n) - L\times H(q)}{\sqrt{2} \sigma_C} \right| \right),
% \end{align}
% where $\sigma_C^2$ denotes the variance of $f_C(R^n)$ under null hypothesis. It has shown in [] that $\sigma_C^2$ is given by
% \begin{align}
% 	\sigma_C^2 = c_C(L,K)^2 \times \frac{\mathrm{Var}[g(A_n)]}{K},
% \end{align}
% where
% \begin{align}
% 	c_C(L,K)^2 \simeq \tilde{d}(L) + \frac{\tilde{e}(L)\times2^L}{K}.
% \end{align}
% Eq. (\ref{eq:coron_expected_value_truly_random}) shows that the expected value is exactly equal to the entropy of blocks. 
%
% Let us consider a binary sequence $U_s^n \in B^n$ generated by an ergodic stationary source $S$. For $U_s^n$, it has been shown that the following relation holds.
% \begin{align}
% 	\mathbb{E}[f_C(U_s^n)] = K_L,
% \end{align}
% where $K_L$ is given in (\ref{eq:K_L}).
%
When we consider a binary sequence $U_{\mathrm{BMS}_p}^n$ generated by a binary memoryless source $\mathrm{BMS}_p$, we exactly have
% Note that a sequence $U_{\mathrm{BMS}_p}^n$ follows a uniform distribution on $\{0,1\}$ taking each bit to be ``$1$'' with probability $p\in (0,1)$ independently. We can show that the relation $K_L=L\times H(p)$ holds for a binary memoryless source.
% Considering the binary memoryless source $\mathrm{BMS}_p$, the entropy of $L$ bit blocks $K_L$ is equal to $L\times H(p)$. Therefore we have
%
% In the case of a sequence $U_{\mathrm{BMS}_p}^n$ generated by $\mathrm{BMS}_p$, the entropy of $L$ bit blocks $K_L$ is equal to $L \times H(p)$. Therefore, we have
\begin{align}\label{eq:E_BMS}
	\mathbb{E}[f_C(U_{\mathrm{BMS}_p}^n)] = L\times H(p).
\end{align}
%
% It is necessary to derive the mean and variance of the reference distribution for a sequence of length $n$ generated by a binary symmetric source (BSS).
% %
% The expected value of the reference distribution for a BSS has been obtained as
% \begin{align}
% 	\mathbb{E}[f_C(R^n)] = L.
% \end{align}
% From Eq. (), the mean of the reference distribution for a truly random source is equal to the length of blocks.
% %
% \par
% %
% On the other hand, the variance of the reference distribution for a BSS has been obtained as
% \begin{align}
% 	\sigma_C^2 = c_C(L,K)^2 \times \frac{\mathrm{Var}[g(A_n)]}{K},
% \end{align}
% where $c_C(L,K)$ is correction factor by which $\sigma_M^2$ is reduced compared with what it would be if the terms $A_n$ were independent. The expression of $c_C(L,K)$ is similar to the one of $c_M(L,K)$, and can be approximated for $K\geq 33\times2^L$ by
% \begin{align}
% 	c_C(L,K)^2 \simeq d(L) + \frac{e(L)\times 2^L}{K},
% \end{align}
% where $d(L)$ and $e(L)$ are listed in \cite{coron1999security} for $L=3,4,\dots,16$. Note that $\mathrm{Var}[g(A_n)]$ can be calculated by
% \begin{align}
% \begin{split}
% 	\mathrm{Var}[g(A_n)] 
% 	% &= \mathbb{E}[\{g(A_n)\}^2] - \left( \mathbb{E}[g(A_n)] \right)^2 \\
% 	&=2^{-L} \sum_{i=2}^{\infty} (1-2^{-L})^{i-1} \left( \sum_{k=1}^{i-1} \frac{\log_2 \mathrm{e}}{k} \right)^2 -L^2,
% \end{split}
% \end{align}
% and the value of $\mathrm{Var}[g(A_n)]$ are also listed in \cite{coron1999security} for $L=3,4,\dots,16$.
%-------------------------------------------------------------------%
%-------------------------------------------------------------------%
%-------------------------------------------------------------------%
%-------------------------------------------------------------------%
%-------------------------------------------------------------------%
%-------------------------------------------------------------------%
%-------------------------------------------------------------------%
%-------------------------------------------------------------------%
%-------------------------------------------------------------------%
%-------------------------------------------------------------------%
%-------------------------------------------------------------------%
%-------------------------------------------------------------------%
%-------------------------------------------------------------------%
%-------------------------------------------------------------------%
%-------------------------------------------------------------------%
%-------------------------------------------------------------------%
\subsection{Highly sensitive universal statistical test}
% In the previous subsections, we have seen that Maurer's and Coron's universal statistical test can detect the non-randomness of a binary sequence.
% % We have seen that the non-randomness of a binary sequence can be detected by Maurer's and Coron's universal statistical test described in the previous subsections. 
% Both tests evaluate the non-randomness of a binary sequence whether the relation (\ref{eq:maurer_asymptotic_R}) or (\ref{eq:coron_expected_value_truly_random}) holds for $q=0.5$, where $q=\mathrm{Pr}[x_i=1]$. It has been suggested in \cite{yamamoto2016highly} that the deviation from $q=0.5$ cannot be detected with high sensitivity since the derivative of the binary entropy function is equal to $0$ at $q=0.5$, i.e.,
% \begin{align}
%   \left.\frac{\mathrm{d}}{\mathrm{d}q}H(q) \right|_{q=0.5} = \left.\log_2\frac{1-q}{q}\right|_{q=0.5} = 0.
% \end{align}
In the previous subsections, we have seen that Maurer's and Coron's universal statistical test can detect the non-randomness of a binary sequence.
% We have seen that the non-randomness of a binary sequence can be detected by Maurer's and Coron's universal statistical test described in the previous subsections. 
Both tests evaluate the non-randomness of a binary sequence whether the relation (\ref{eq:maurer_asymptotic_R}) or (\ref{eq:coron_expected_value_truly_random}) holds for $p=0.5$, where $q=\mathrm{Pr}[x_i=1]$. It has been suggested in \cite{yamamoto2016highly} that the deviation from $p=0.5$ cannot be detected with high sensitivity since the derivative of the binary entropy function is equal to $0$ at $p=0.5$, i.e.,
\begin{align}
  \left.\frac{\mathrm{d}}{\mathrm{d}p}H(p) \right|_{p=0.5} = \left.\log_2\frac{1-p}{p}\right|_{p=0.5} = 0.
\end{align}
\par
In \cite{yamamoto2016highly}, the universal test called ``highly sensitive universal statistical test'' has been proposed. This test is constructed on the basis of Maurer's and Coron's universal statistical tests.
%
In the highly sensitive test, a given binary sequence $x_1,x_2,\dots$ with $q \simeq 0.5$ is converted into another binary sequence $\hat{x}_1,\hat{x}_2,\dots$ with $\hat{q} \neq 0.5$ by
%
\begin{align}\label{eq:convert}
\begin{split}
  \mathrm{Pr}[\hat{x}_i = 0 \mid x_i=0] &= 1, \\
  \mathrm{Pr}[\hat{x}_i = 1 \mid x_i=1] &= \beta,
\end{split}
\end{align}
%
where $\hat{q}=\mathrm{Pr}[\hat{x}_i=1]$ and $\beta \in (0,1)$. Note that ``$0$'' in a sequence is not converted, and ``$1$'' in a sequence is flipped into ``$0$'' in probability with $1-\beta$. By Eq. (\ref{eq:convert}), a given sequence is converted into another binary sequence with $\hat{q}=0.5\beta$. In \cite{yamamoto2016highly}, the non-randomness can be detected by applying Coron's universal test for a converted binary sequence.
Numerical experiments show the effectiveness of the highly sensitive test and $\beta=0.66$ maximizes the effectiveness.
% The effectiveness of Highly sensitive test has been presented by some numerical experiments. 
% The optimal $\beta$ to attain a sensitive $\hat{q}$ is shown as $\beta=0.66$.
%
% It is suggested in \cite{yamamoto2016highly} that the highly sensitive test can detect the non-randomness much more sensitive than Maurer's or Coron's tests by numerical simulations. In the highly sensitive test, a given binary sequence $x^n$ is converted into another binary sequence $\hat{x}^n$ with $\hat{q}$ by switching each bit $x_i$ to $\hat{x}_i$ by
% %
% where $\alpha \in (0,1)$. Note that the bit ``$0$'' in a sequence is not converted, and the bit ``$1$'' in a sequence is converted into ``$0$'' in probability of $1-\alpha$. By Eq. (\ref{eq:convert}), a given sequence is converted into a binary sequence with $\hat{q}=0.5\alpha$. It also has been reported in \cite{yamamoto2016highly} that the optimal $\alpha$ to attain a sensitive $\hat{q}$ is $\alpha=0.66$. 
%
\par
The null hypothesis under the highly sensitive test $\mathcal{H}_0$ is that a binary sequence of length $n$ being tested follows a uniform distribution on $\{0,1\}^n$. As an implicit assumption, an additional hypothesis $\widetilde{\mathcal{H}}_0$ that a random number used for flipping is ideal needs to be considered. Hence, the null hypothesis under the highly sensitive test $\overline{\mathcal{H}}_0:=\mathcal{H}_0 \land \widetilde{\mathcal{H}}_0$ is that a converted binary sequence is considered to be generated by a distribution on $\{0,1\}^n$ taking each bit to be ``$1$'' with probability $\hat{q}$ independently. If the null hypothesis $\overline{\mathcal{H}}_0$ is rejected, then the hypothesis $\mathcal{H}_0$ would be rejected. Otherwise, there is no evidence for rejecting the null hypothesis $\overline{\mathcal{H}}_0$. The algorithm of the highly sensitive test is described in Algorithm \ref{alg:highly}.
%
\begin{algorithm}[h]
\caption{The procedure of highly sensitive universal statistical test}
\label{alg:highly}
\begin{algorithmic}[1]
\State Set parameters $\alpha,\,L,\, Q,\,K$ and $\beta$.
\State Convert a given binary sequence $s^n$ into $\hat{s}^n$ by Eq. (\ref{eq:convert}).
\State Divide a converted binary sequence $\hat{s}^n$ into adjacent non-overlapping blocks of length $L$, and compute $A_n(\hat{s}^n)$ for $n=Q+1,Q+2,\dots,Q+K$ by Eq.(\ref{eq:An}). \label{state:divide}
% The recommendation is $L$ as between $6$ and $16$, $Q \geq 10\times2^L$, $K \geq 1000\times2^L$. 
% \State Divide a binary sequence $x^n$ into adjacent non-overlapping blocks of length $L$ and calculate $A_n$ from Eq. (\ref{eq:An}) for $n=Q+1,Q+2,\dots,Q+K$.
\State Compute a test statistic value $f_C(\hat{s}^n)$ by Eq. (\ref{eq:fC}).
\State Compute a $p \, \mathchar`- \mathrm{value}$ by
\begin{align}
	 p\mathchar`- \mathrm{value} = \mathrm{erfc} \left( \left| \frac{f_C(\hat{s}^n) - L\times H(0.5\beta)}{\sqrt{2} \sigma_C(0.5\beta)} \right| \right).
\end{align}
% \State If the computed p-value is less than $\alpha$, then conclude the sequence is non-random. Otherwise, conclude the sequence is random.
\State Reject $\mathcal{H}_0$ if $p\mathchar`- \mathrm{value} < \alpha$; else accept $\overline{H}_0$.
\end{algorithmic}
\end{algorithm}
%
\par
% Let $\mathcal{U}_{\hat{q}}^{n}$ be a distribution on $\{0,1\}^n$ whose the probability of taking ``$1$'' is equal to $\hat{q}$ that random variables of length $n$ follows.
%
To implement the highly sensitive test, it is necessary to derive the expected value and the variance of reference distribution of a binary sequence.
%
By Eq. (\ref{eq:E_BMS}), the expected value has been obtained, whereas the variance has not been analyzed theoretically. Then, a value obtained by simulation is used as the variance in \cite{yamamoto2016highly}. The accurate variance should be derived to improve the reliability of the highly sensitive test.
% It should be derived the accurate value of variance to improve the reliability of highly sensitive test.
%
% For the sequence, calculate the variable $A_n$ as the same, and calculate the p-value by
% \begin{align}
% 	p\mathrm{\mathchar`- value} = \mathrm{erfc} \left( \left| \frac{f_C(\hat{x}^n) - L\times H(\hat{q})}{\sqrt{2} \sigma_C(\hat{q})} \right| \right),
% \end{align}
% where $\sigma_C(\hat{q})^2$ is the variance of $f_C(\hat{R}^n)$ under null hypothesis.
%
% \par
% To execute highly sensitive test, it is necessary to derive $\sigma(\hat{q})$ for all $\hat{q} \in (0,1)$. In \cite{yamamoto2016highly}, a value obtained by simulation has used since it is complicated to derive $\sigma({\hat{q}})$ theoretically. 
%
% \begin{algorithm}[h]
% \caption{The procedure of highly sensitive universal statistical test}
% \label{alg:highly}
% \begin{algorithmic}[1]
% \State Set the positive integers $L,\, Q,\,K$ and $\alpha$ as a parameter, and a significance level $\rho$.
% \State Convert a given binary sequence $s^n$ into $\hat{s}^n$ by Eq. (\ref{eq:convert}).
% \State Divide a converted binary sequence $\hat{s}^n$ into adjacent non-overlapping blocks of length $L$, and compute $A_n(\hat{s}^n)$ for $n=Q+1,Q+2,\dots,Q+K$ by Eq.(\ref{eq:An}). \label{state:divide}
% % The recommendation is $L$ as between $6$ and $16$, $Q \geq 10\times2^L$, $K \geq 1000\times2^L$. 
% % \State Divide a binary sequence $x^n$ into adjacent non-overlapping blocks of length $L$ and calculate $A_n$ from Eq. (\ref{eq:An}) for $n=Q+1,Q+2,\dots,Q+K$.
% \State Compute a test statistic value $f_C(\hat{s}^n)$ by Eq. (\ref{eq:fC}).
% \State Compute a $p \, \mathchar`- \mathrm{value}$ by
% \begin{align}
% 	 p\mathchar`- \mathrm{value} = \mathrm{erfc} \left( \left| \frac{f_C(\hat{s}^n) - L\times H(0.5\alpha)}{\sqrt{2} \sigma_C(0.5\alpha)} \right| \right).
% \end{align}
% % \State If the computed p-value is less than $\alpha$, then conclude the sequence is non-random. Otherwise, conclude the sequence is random.
% \State Reject $\mathcal{H}_0$ if $p\mathchar`- \mathrm{value} < \rho$; else accept $\overline{H}_0$.
% \end{algorithmic}
% \end{algorithm}
%
%-------------------------------------------------------------------%
%-------------------------------------------------------------------%
%-------------------------------------------------------------------%
%-------------------------------------------------------------------%
%-------------------------------------------------------------------%
%-------------------------------------------------------------------%
%-------------------------------------------------------------------%
%-------------------------------------------------------------------%
%-------------------------------------------------------------------%
%-------------------------------------------------------------------%
%-------------------------------------------------------------------%
%-------------------------------------------------------------------%
%-------------------------------------------------------------------%
%-------------------------------------------------------------------%
%-------------------------------------------------------------------%
%-------------------------------------------------------------------%
%-------------------------------------------------------------------%
%-------------------------------------------------------------------%
%-------------------------------------------------------------------%
%-------------------------------------------------------------------%
%-------------------------------------------------------------------%
%-----------------------------------------------------------------------------------------------%
\newpage
\section{Distribution}\label{sec:distribution}
In this section, we consider the distribution of $A_n(\hat{x}^n)$, where $\hat{x}^n$ is an $n$-bit random variable. For each $i$, we have
\begin{align}
  \mathrm{Pr}[(\hat{x}^n)_i = 1] = \hat{q},
\end{align}
where $(\hat{x}^n)_i$ is the $i$-th bit of $\hat{x}^n$. Each bit $(\hat{x}^n)_i$ is independent from other bits.
For simplicity, we write $A_n(\hat{x}^n)$ as $A_n$ unless specified.
In the following, we consider an assumption $Q\to\infty$, and due to the situation, the index of $A_n$ should be replaced as illustrated in Figure \ref{fig:replace}. Then, a sequence of $\{A_k\}_{k=1}^{K}$ follows a stationary ergodic process, that is, the joint distribution of $\{A_k\}_{k=n}^{n+m}$ only depends on $m$.  
In this section, we derive a marginal distribution of $A_n$ and a joint distribution of $(A_n,A_{n+k})$ necessary for calculating the variance of the reference distribution of the highly sensitive test.
%-----------------------------------------------------------------------------------------------%
\begin{figure}[b]
\centering
\begin{tikzpicture}
\draw (-1.0,0) rectangle (0,1);
\draw node at (-0.5,0.2) [above] {$A_1$};
\draw (0,0) rectangle (1,1);
\draw node at (0.5,0.2) [above] {$A_2$};
\draw (1,0) rectangle (2,1);
\draw node at (1.5,0.2) [above] {$A_3$};
\draw (2,0) rectangle (3,1);
\draw node at (2.5,0.2) [above] {$A_4$};
\draw (3,0)--(3.5,0);
\draw (3,1)--(3.5,1);
\draw[loosely dotted, very thick] (3.75,0.5)--(4.25,0.5);
\draw (4.5,0)--(5.0,0);
\draw (4.5,1)--(5.0,1);
\draw (5,0) rectangle (6,1);
\draw node at (5.5,0.2) [above] {$A_{Q}$};
\draw (6,0) rectangle (7,1);
\draw node at (6.5,0.2) [above] {$A_{Q+1}$};
\draw (7,0) rectangle (8,1);
\draw node at (7.5,0.2) [above] {$A_{Q+2}$};
\draw (8,0) rectangle (9,1);
\draw node at (8.5,0.2) [above] {$A_{Q+3}$};
\draw (9,0) rectangle (10,1);
\draw (10,0)--(10.5,0);
\draw (10,1)--(10.5,1);
%
\draw[->] (4.0, -0.5) -- (4.0, -1.0);
%
\draw (-1.0,-2.5) rectangle (0,-1.5);
\draw node at (-0.5,-2.3) [above] {$A_{1-Q}$};
\draw (0,-2.5) rectangle (1,-1.5);
\draw node at (0.5,-2.3) [above] {$A_{2-Q}$};
\draw (1,-2.5) rectangle (2,-1.5);
\draw node at (1.5,-2.3) [above] {$A_{3-Q}$};
\draw (2,-2.5) rectangle (3,-1.5);
\draw node at (2.5,-2.3) [above] {$A_{4-Q}$};
\draw (3,-2.5)--(3.5,-2.5);
\draw (3,-1.5)--(3.5,-1.5);
\draw[loosely dotted, very thick] (3.75,-2.0)--(4.25,-2.0);
\draw (4.5,-2.5)--(5.0,-2.5);
\draw (4.5,-1.5)--(5.0,-1.5);
\draw (5,-2.5) rectangle (6,-1.5);
\draw node at (5.5,-2.3) [above] {$A_{0}$};
\draw (6,-2.5) rectangle (7,-1.5);
\draw node at (6.5,-2.3) [above] {$A_{1}$};
\draw (7,-2.5) rectangle (8,-1.5);
\draw node at (7.5,-2.3) [above] {$A_{2}$};
\draw (8,-2.5) rectangle (9,-1.5);
\draw node at (8.5,-2.3) [above] {$A_{3}$};
\draw (9,-2.5) rectangle (10,-1.5);
\draw (10,-2.5)--(10.5,-2.5);
\draw (10,-1.5)--(10.5,-1.5);
\end{tikzpicture}
\caption{Replacement of the index}
\label{fig:replace}
\end{figure}
%-----------------------------------------------------------------------------------------------%
%-----------------------------------------------------------------------------------------------%
%-----------------------------------------------------------------------------------------------%
\subsection{Derivation of marginal distribution}
We consider the event of $\left< A_n=i \right>$ for $i\geq 1$. The event occurs when $n$-th block coincides $(n-i)$-th block and do not coincide other blocks between $n$-th and $(n-i)$-th blocks as illustrated in Figure \ref{fig:A_n=i}. Let $\mathcal{M}$ be such an event. Then, $\mathcal{M}$ is written as
\begin{align}
  \mathcal{M} = \left< b_{n-i} = b_{n}, b_{n-i+1} \neq b_{n} , \dots, b_{n-1} \neq b_{n}  \right>,
\end{align}
where $b_k$ is the $k$-th block of a sequence $\hat{x}^n$.
Then, we can derive the probability of occurring $\left< A_n = i \right>$ under the assumption that the blocks are statistically independent and identically distributed as
\begin{align}
  \label{eq:conditional_probability}
  \mathrm{Pr}[A_n=i] = \sum_{r=0}^{L} \mathrm{Pr}[\mathcal{M} \mid \ell(b_n) = r] \times \mathrm{Pr} [\ell(b_n) = r],
\end{align}
where $\ell(b)$ denotes the number of ``$1$'' included in the block $b \in \{0,1\}^L$. We also have
\begin{align}
  \mathrm{Pr}[\mathcal{M} \mid \ell(b_n) = r] &= w_r \times ( 1 - w_r )^{i-1},\label{eq:probability_M_mid} \\
  \mathrm{Pr} [\ell(b_n) = r] &= \binom{L}{r} w_r, \label{eq:probability_l_r}
\end{align}
where $w_r = \hat{q}^r (1-\hat{q})^{L-r}$ and $\binom{L}{r} = \frac{L!}{r!(L-r)!}$ is a binomial coefficient. 
%
By combining Eqs. (\ref{eq:conditional_probability})--(\ref{eq:probability_l_r}), the following relation is obtained as
\begin{align}\label{eq:A_n=i}
  \mathrm{Pr}[A_n=i] = \sum_{r=0}^{L} \binom{L}{r} w_r^2 ( 1 - w_r )^{i-1},
\end{align} 
for $i\geq 1$.
%
\begin{figure}
\centering
\begin{tikzpicture}
\draw (-1.5,0)--(-1.0,0);
\draw (-1.5,1)--(-1.0,1);
\draw (-1.0,0) rectangle (0,1);
\draw (0,0) rectangle (1,1);
\filldraw [draw=black, fill=pink] (1,0) rectangle (2,1) node(A) at (1.5, 1.0) [above] {$b_{n-i}$};
\filldraw [pattern = north east lines] (2,0) rectangle (3,1);
\filldraw [pattern = north east lines] (3,0) rectangle (4,1);
\filldraw [pattern = north east lines] (4,0) rectangle (5,1);
\filldraw [pattern = north east lines] (5,0) rectangle (6,1);
\filldraw [pattern = north east lines] (6,0) rectangle (7,1);
\filldraw [draw=black, fill=pink] (7,0) rectangle (8,1) node(AA) at (7.5, 1.0) [above] {$b_n$};
\draw (8,0) rectangle (9,1);
\draw (9,0) rectangle (10,1);
\draw (10,0)--(10.5,0);
\draw (10,1)--(10.5,1);
%
\draw[<->] (A) to[bend left=20] node [above] {\small $b_{n-i} = b_{n+k-j}$} (AA);
\end{tikzpicture}
\caption{The arrangement of blocks in the case of $A_n=i$.}
\label{fig:A_n=i}
\end{figure}
%
%-----------------------------------------------------------------------------------------------%
\subsection{Derivation of joint distribution}\label{subsec:4-2}
We consider the event $\left< A_n=i,\, A_{n+k}=j \right>$ for $i\geq 1$ and $j\geq 1$. 
It has been shown in \cite{coron1998accurate} that the following relation holds for a truly random sequence $R^N$
%
\begin{align}\begin{split}
  &\mathrm{Pr}[A_n(R^N)=i,\, A_{n+k}(R^N)=j] \\
   &=\left\{ \begin{array}{ll}
    2^{-2L}(1-2^{-L})^{i+j-2} & (1 \leq j \leq k-1) \\
    2^{-2L}(1-2^{-L})^{i+k-2} & (j=k) \\
    2^{-2L}(1-2^{-L})^{i-j+2k-1} \left( 1 - 2^{-L+1} \right)^{j-k-1} & (k+1 \leq j \leq k+i-1) \\
    0 & (j=k+i) \\
    2^{-2L}(1-2^{-L})^{-i+j-1} \left( 1 - 2^{-L+1} \right)^{i-1} & (j \geq k+i+1) 
  \end{array} \right..
\end{split}\end{align}
In this subsection, we derive the joint distribution of $(A_n(\hat{x}^n),A_{n+k}(\hat{x}^n))$ holding for any $\hat{q} \in (0,1)$.
%-----------------------------------------------------------------------------------------------%
%-----------------------------------------------------------------------------------------------%
%-----------------------------------------------------------------------------------------------%
\subsubsection{Case of $1 \leq j \leq k-1$}
When $1 \leq j \leq k-1$, the events $\left< A_n=i \right>$ and $\left< A_{n+k}=j \right>$ are independent each other as illustrated in Figure \ref{fig:case1}, since there are no overlapping between blocks from $b_{n-i}$ to $b_{n}$ and blocks from $b_{n+k-j}$ to $b_{n+k}$. Thus, we obtain the joint distribution as
\begin{align}
\begin{split}
  \label{eq:joint1}
  \mathrm{Pr}[A_n=i, A_{n+k}=j] =& \mathrm{Pr}[A_n=i] \times \mathrm{Pr}[A_{n+k}=j]\\
  =&\left( \sum_{r=0}^{L} \binom{L}{r}w_r^2 (1-w_r)^{i-1} \right) \times \left( \sum_{r=0}^{L} \binom{L}{r}w_r^2 (1-w_r)^{j-1} \right).
  \end{split}
\end{align}
In the above relations, the second equality is obtained from Eq. (\ref{eq:A_n=i}). Recall that $w_r = \hat{q}^r(1-\hat{q})^{L-r}$ where $\hat{q} \in (0,1)$.
\begin{figure}
\centering
  \begin{tikzpicture}
    \draw (-0.3,0)--(0,0);
    \draw (-0.3,0.6)--(0,0.6);
    \draw (0,0) rectangle (0.6,0.6);
    \filldraw [draw=black, fill=pink] (0.6,0) rectangle (0.6*2, 0.6) node (A) at (0.9, 0.6) [above] {$b_{n-i}$};
    \filldraw [pattern = north east lines] (0.6*2, 0) rectangle (0.6*3, 0.6);
    \filldraw [pattern = north east lines] (0.6*3, 0) rectangle (0.6*4, 0.6);
    \filldraw [pattern = north east lines] (0.6*4, 0) rectangle (0.6*5, 0.6);
    \filldraw [draw=black, fill=pink] (0.6*5, 0) rectangle (0.6*6, 0.6) node (AA) at (3.3, 0.6) [above] {$b_{n+k-j}$};
    \filldraw [draw=black, fill=white] (0.6*6, 0) rectangle (0.6*7, 0.6);
    \filldraw [draw=black, fill=white] (0.6*7, 0) rectangle (0.6*8, 0.6);
    \filldraw [draw=black, fill=yellow!60] (0.6*8, 0) rectangle (0.6*9, 0.6) node (B) at (5.1, 0.6) [above]{$b_n$};
    \filldraw [pattern = north west lines] (0.6*9, 0) rectangle (0.6*10, 0.6);
    \filldraw [pattern = north west lines] (0.6*10, 0) rectangle (0.6*11, 0.6);
    \filldraw [pattern = north west lines] (0.6*11, 0) rectangle (0.6*12, 0.6);
    \filldraw [pattern = north west lines] (0.6*12, 0) rectangle (0.6*13, 0.6);
    \filldraw [pattern = north west lines] (0.6*13, 0) rectangle (0.6*14, 0.6);
    \filldraw [draw=black, fill=yellow!60] (0.6*14, 0) rectangle (0.6*15, 0.6) node (BB) at (8.7, 0.6) [above]{$b_{n+k}$};
    \draw (0.6*15, 0) rectangle (0.6*16, 0.6);
    \draw (0.6*16, 0)--(0.6*16.5, 0);
    \draw (0.6*16, 0.6)--(0.6*16.5, 0.6);
    %
    \draw[<->] (A) to[bend left=30] node [above] {\small $b_{n-i} = b_{n+k-j}$} (AA);
    \draw[<->] (B) to[bend left=30] node [above] {\small $b_{n} = b_{n+k}$} (BB);
  \end{tikzpicture}
  \caption{An example of the arrangement of blocks in the case of $1\leq j \leq k-1$}
  \label{fig:case1}
\end{figure}
%-----------------------------------------------------------------------------------------------%
\subsubsection{Case of $j=k$}
For every $b \in B^{L}$, we consider the event $e_2(b)=\left< A_n=i ,\, A_{n+k}=j,\,b_n=b\right>$ for $j=k$. 
An example for the arrangement of blocks is illustrated in Figure \ref{fig:case2}.
The event $e_2(b)$ can be written as
\begin{align}\label{eq:e_2}
\begin{split}
  e_2(b) 
    = &\left< b_{n-i} = b , b_{n} = b, b_{n+k} = b \right> \\
    &\land \left< b_{n-i+1} \neq b , \dots , b_{n-1} \neq b \right> \\
    &\land \left< b_{n+1} \neq b , \dots , b_{n+k-1} \neq b \right>.
\end{split}
\end{align}
Since the blocks are statistically independent and uniformly distributed, we have
\begin{align}\label{eq:probability_e_2}
  \mathrm{Pr} \left[ e_2(b) \right] 
  =w_r^3 \times (1-w_{r})^{i+k-2}.
\end{align}
%
We define $\mathcal{E}_2$ as the event of occurring $\left< A_n=i ,\, A_{n+k}=j\right>$ in the case of $j=k$. Then, $\mathcal{E}_2$ can be written by using Eq. (\ref{eq:e_2}) as
\begin{align}\label{eq:E_2}
  \mathcal{E}_2 = \bigvee_{b\in B^L} e_2(b).
\end{align}
%
Therefore, the joint distribution is derived as follows:
\begin{align}
\begin{split}\label{eq:joint_dist_2}
  \mathrm{Pr}[A_n=i ,\, A_{n+k}=j] &=
  \mathrm{Pr}[\mathcal{E}_2] \\
  &=\mathrm{Pr}\left[ \bigvee_{b \in B^L} e_2(b) \right] \\
  &= \sum_{b\in B^L} \mathrm{Pr} [e_2(b)] \\
  &= \sum_{b\in B_0^L \cup \dots \cup B_L^L} \mathrm{Pr} [e_2(b)] \\
  &= \sum_{r=0}^{L} \sum_{b\in B_{r}^L} \mathrm{Pr} \left[ e_2(b) \right] \\
  &= \sum_{r=0}^{L} \left(\sharp B_{r}^L \right) \mathrm{Pr} \left[ e_2(b) \right] \\
  &= \sum_{r=0}^{L} \dbinom{L}{r} w_{r}^3 (1-w_{r})^{i+k-2},
\end{split}
\end{align}
where $B_r^L := \{ b\in B^L \mid \ell(b)=r \}$ and $\sharp \mathcal{S}$ be the number of elements included in a finite set $\mathcal{S}$.
The sixth equality in Eq. (\ref{eq:joint_dist_2}) has been obtained with the fact that $\mathrm{Pr}[e_2(b)]$ depends only on $\ell (b)$.
%
\begin{figure}
\centering
  \begin{tikzpicture}
    \draw (-0.3,0)--(0,0);
    \draw (-0.3,0.6)--(0,0.6);
    \draw (0,0) rectangle (0.6,0.6);
    \filldraw [draw=black, fill=pink] (0.6,0) rectangle (0.6*2, 0.6) node (A) at (0.9, 0.6) [above]{$b_{n-i}$};
    \filldraw [pattern = north east lines] (0.6*2, 0) rectangle (0.6*3, 0.6);
    \filldraw [pattern = north east lines] (0.6*3, 0) rectangle (0.6*4, 0.6);
    \filldraw [pattern = north east lines] (0.6*4, 0) rectangle (0.6*5, 0.6);
    \filldraw [pattern = north east lines] (0.6*5, 0) rectangle (0.6*6, 0.6);
    \filldraw [draw=black, fill=pink] (0.6*6, 0) rectangle (0.6*7, 0.6) node (B) at (3.9, 0.6) [above]{$b_{n}$};
    \filldraw [pattern = north east lines] (0.6*7, 0) rectangle (0.6*8, 0.6);
    \filldraw [pattern = north east lines] (0.6*8, 0) rectangle (0.6*9, 0.6);
    \filldraw [pattern = north east lines] (0.6*9, 0) rectangle (0.6*10, 0.6);
    \filldraw [pattern = north east lines] (0.6*10, 0) rectangle (0.6*11, 0.6);
    \filldraw [pattern = north east lines] (0.6*11, 0) rectangle (0.6*12, 0.6);
    \filldraw [pattern = north east lines] (0.6*12, 0) rectangle (0.6*13, 0.6);
    \filldraw [pattern = north east lines] (0.6*13, 0) rectangle (0.6*14, 0.6);
    \filldraw [draw=black, fill=pink] (0.6*14, 0) rectangle (0.6*15, 0.6) node (C) at (8.7, 0.6) [above]{$b_{n+k}$};
    \draw (0.6*15, 0) rectangle (0.6*16, 0.6);
    \draw (0.6*16, 0)--(0.6*16.5, 0);
    \draw (0.6*16, 0.6)--(0.6*16.5, 0.6);
    %
    \draw[<->] (A) to[bend left=20] node [above] {\small $b_{n-i} = b_{n}$} (B);
    \draw[<->] (B) to[bend left=20] node [above] {\small $b_{n} = b_{n+k}$} (C);
  \end{tikzpicture}
  \caption{An example of the arrangement of blocks in the case of $j = k$}
  \label{fig:case2}
\end{figure}
%-----------------------------------------------------------------------------------------------%
\subsubsection{Case of $k+1 \leq j \leq k+i-1$}%case3
For every $b,\, b^\prime \in B^L$, we consider the event $e_3(b,b^\prime) \left< A_n=i ,\, A_{n+k}=j,\,b_n=b,\,b_{n+k}=b^\prime\right>$ for $k+1 \leq j \leq k+i-1$. 
An example for the arrangement of blocks is illustrated in Figure \ref{fig:case3}.
The event $e_3(b,b^\prime)$ can be written as
\begin{align}
\begin{split}
  \label{eq:e_3}
  e_3 \left(b,b^\prime\right) = 
  &\left< b_{n-i} = b , b_{n} = b , b_{n+k-j} = b^\prime , b_{n+k} = b^\prime \right> \\
  &\land \left< b_{n-i+1} \neq b, \dots, b_{n+k-j-1} \neq b \right> \\
  &\land \left< b_{n+k-j+1} \neq b, \dots, b_{n-1} \neq b \right> \\
  &\land \left< b_{n+k-j+1} \neq b^\prime, \dots, b_{n-1} \neq b^\prime \right> \\
  &\land \left< b_{n+1} \neq b^\prime , \dots, b_{n+k-1} \neq b^\prime \right>.
\end{split}
\end{align}
Since the blocks are statistically independent and uniformly distributed, we have
\begin{align}
\begin{split}
  \label{eq:probability_e3}
  \mathrm{Pr} \left[ e_3 \left(b,b^\prime\right) \right] 
  =& w_{r_1}^2 w_{r_2}^2 
  (1-w_{r_1})^{i-j+k-1} 
  (1-w_{r_1}-w_{r_2})^{j-k-1}
  (1-w_{r_2})^{k-1}.
\end{split}
\end{align}
We define $\mathcal{E}_3$ as the event $\left< A_n=i ,\, A_{n+k}=j \right>$ in the case of $k+1 \leq j \leq k+i-1$, $\mathcal{E}_3$ can be written by using $e_3(b,b^\prime)$ as
\begin{align}\label{eq:E_3}
  \mathcal{E}_3 = \bigvee_{b \in B^L} \bigvee_{b^\prime \in B^L \setminus \{b\}} e_3 \left(b,b^\prime\right).
\end{align}
Therefore, we can derive the joint distribution as
\begin{align}\label{eq:joint_dist_3}
\begin{split}
  &\mathrm{Pr} [A_n=i ,\, A_{n+k}=j] \\
  &=\mathrm{Pr} [\mathcal{E}_3] \\
  &= \mathrm{Pr} \left[ \bigvee_{b_1 \in B^L} \bigvee_{b_2 \in B^L \setminus \{b_1\}}
  e_3(b,b^\prime) \right] \\
  %
  &=\sum_{b_1 \in B^L} \sum_{b_2 \in B^{L} \setminus \{ b_1 \}} \mathrm{Pr} \left[ e_3(b,b^\prime) \right] \\
  %
  &= \sum_{r_1=0}^{L} \sum_{b_1 \in B^L_{r_1}} \sum_{r_2=0}^{L} \sum_{b_2 \in B^L_{r_2} \setminus \{ b_1 \}} \mathrm{Pr} \left[ e_3(b,b^\prime) \right] \\
  %
  &= \sum_{r_1=0}^{L} \sum_{r_2 \neq r_1} \sum_{b_1 \in B^L_{r_1}} \sum_{b_2 \in B^L_{r_2}} \mathrm{Pr} \left[e_3(b,b^\prime) \right] 
  %
  + \sum_{r_1=0}^{L} \sum_{r_2 \in \{r_1\}} \sum_{b_1 \in B^L_{r_1}} \sum_{b_2 \in B^L_{r_1} \setminus \{ b_1 \}} \mathrm{Pr} \left[e_3(b,b^\prime) \right] \\
  %
  &= \sum_{r_1=0}^{L} \sum_{r_2 \neq r_1} \dbinom{L}{r_1} \dbinom{L}{r_2} \mathrm{Pr} \left[e_3(b,b^\prime) \right] 
  %
  + \sum_{r_1=0}^{L}\sum_{r_2\in\{r_1\}} \dbinom{L}{r_1} \left\{ \dbinom{L}{r_1} -1 \right\} \mathrm{Pr} \left[e_3(b,b^\prime) \right],
\end{split}
\end{align}
where $\mathrm{Pr} \left[e_3(b,b^\prime) \right]$ is given in Eq. (\ref{eq:probability_e3}).
The last equality in Eq. (\ref{eq:joint_dist_3}) holds since $\mathrm{Pr} [A_n=i ,\, A_{n+k}=j,\,b_n=b,\,b_{n+k}=b^\prime]$ depends only on $\ell (b)$ and $\ell (b^\prime)$.
%
\begin{figure}
\centering
  \begin{tikzpicture}
    \draw (-0.3,0)--(0,0);
    \draw (-0.3,0.6)--(0,0.6);
    \draw (0,0) rectangle (0.6,0.6);
    \filldraw [draw=black, fill=pink] (0.6,0) rectangle (0.6*2, 0.6) node (A) at (0.9,0.6) [above]{$b_{n-i}$};
    \filldraw [pattern = north east lines] (0.6*2, 0) rectangle (0.6*3, 0.6);
    \filldraw [pattern = north east lines] (0.6*3, 0) rectangle (0.6*4, 0.6);
    \filldraw [draw=black, fill=yellow!60] (0.6*4, 0) rectangle (0.6*5, 0.6) node (B) at (2.7,0.6) [above] {$b_{n+k-j}$};
    \filldraw [pattern = crosshatch] (0.6*5, 0) rectangle (0.6*6, 0.6);
    \filldraw [pattern = crosshatch] (0.6*6, 0) rectangle (0.6*7, 0.6);
    \filldraw [pattern = crosshatch] (0.6*7, 0) rectangle (0.6*8, 0.6);
    \filldraw [draw=black, fill=pink] (0.6*8, 0) rectangle (0.6*9, 0.6)node (AA) at (5.1,0.6)[above] {$b_n$};
    \filldraw [pattern = north east lines] (0.6*9, 0) rectangle (0.6*10, 0.6);
    \filldraw [pattern = north east lines] (0.6*10, 0) rectangle (0.6*11, 0.6);
    \filldraw [pattern = north east lines] (0.6*11, 0) rectangle (0.6*12, 0.6);
    \filldraw [pattern = north east lines] (0.6*12, 0) rectangle (0.6*13, 0.6);
    \filldraw [pattern = north east lines] (0.6*13, 0) rectangle (0.6*14, 0.6);
    \filldraw [draw=black, fill=yellow!60] (0.6*14, 0) rectangle (0.6*15, 0.6)node (BB) at (8.7,0.6)[above]{$b_{n+k}$};
    \draw (0.6*15, 0) rectangle (0.6*16, 0.6);
    \draw (0.6*16, 0)--(0.6*16.5, 0);
    \draw (0.6*16, 0.6)--(0.6*16.5, 0.6);
    %
    \draw[<->] (A) to[bend left=20] node [above] {\small $b_{n-i} = b_{n}$} (AA);
    \draw[<->] (B) to[bend left=20] node [above] {\small $b_{n+k-j} = b_{n+k}$} (BB);
  \end{tikzpicture}
  \caption{An example of the arrangement of blocks in the case of $k+1 \leq j \leq k+i-1$}
  \label{fig:case3}
\end{figure}
%-----------------------------------------------------------------------------------------------%
\subsubsection{Case of $j=k+i$}%case4
When $j=k+i$, the events $\left< A_n=i \right>$ and $\left< A_{n+k}=j \right>$ do not occur coincidentally as illustrated in Figure \ref{fig:case4}. Thus, the joint distribution is obtained as
\begin{align}
  \label{eq:joint4}
  \mathrm{Pr}[A_n=i,A_{n+k}=j] = 0.
\end{align}
%
\begin{figure}
\centering
\begin{tikzpicture}
  \draw (-0.3,0)--(0,0);
  \draw (-0.3,0.6)--(0,0.6);
  \draw (0,0) rectangle (0.6,0.6);
  \filldraw [draw=black, fill=pink] (0.6,0) rectangle (0.6*2, 0.6) node (A) at (0.9,0.6)[above]{$b_{n-i}$};
  \filldraw [pattern = north east lines] (0.6*2, 0) rectangle (0.6*3, 0.6);
  \filldraw [pattern = north east lines] (0.6*3, 0) rectangle (0.6*4, 0.6);
  \filldraw [pattern = north east lines] (0.6*4, 0) rectangle (0.6*5, 0.6);
  \filldraw [pattern = north east lines] (0.6*5, 0) rectangle (0.6*6, 0.6);
  \filldraw [pattern = north east lines] (0.6*6, 0) rectangle (0.6*7, 0.6);
  \filldraw [pattern = north east lines] (0.6*7, 0) rectangle (0.6*8, 0.6);
  \filldraw [draw=black, fill=pink] (0.6*8, 0) rectangle (0.6*9, 0.6) node (B) at (5.1,0.6)[above]{$b_{n}$};
  \filldraw [pattern = north east lines] (0.6*9, 0) rectangle (0.6*10, 0.6);
  \filldraw [pattern = north east lines] (0.6*10, 0) rectangle (0.6*11, 0.6);
  \filldraw [pattern = north east lines] (0.6*11, 0) rectangle (0.6*12, 0.6);
  \filldraw [pattern = north east lines] (0.6*12, 0) rectangle (0.6*13, 0.6);
  \filldraw [pattern = north east lines] (0.6*13, 0) rectangle (0.6*14, 0.6);
  \filldraw [draw=black, fill=pink] (0.6*14, 0) rectangle (0.6*15, 0.6)node (C) at (8.7,0.6)[above]{$b_{n+k}$};
  \draw (0.6*15, 0) rectangle (0.6*16, 0.6);
  \draw (0.6*16, 0)--(0.6*16.5, 0);
  \draw (0.6*16, 0.6)--(0.6*16.5, 0.6);
  %
  \draw[<->] (A) to[bend left=10] node [above] {\small $b_{n-i} = b_{n}$} (B);
  \draw[<->] (B) to[bend left=10] node [above] {\small $b_{n} \neq b_{n+k}$} (C);
  \draw[<->, dashed] (A) to[bend left=30] node [above] {\small $b_{n-i} = b_{n+k}$} (C);
\end{tikzpicture}
  \caption{An example of the arrangement of blocks in the case of $j=k+i$}
  \label{fig:case4}
\end{figure}
%-----------------------------------------------------------------------------------------------%
%-----------------------------------------------------------------------------------------------%
%-----------------------------------------------------------------------------------------------%
\subsubsection{Case of $j \geq k+i+1$}
For every $b,\, b^\prime \in B^L$, we consider the event $\left< A_n=i ,\, A_{n+k}=j,b_n=b,\,b_{n+k}=b^\prime\right>$ occurs when $j \geq k+i+1$. An example for the arrangement of blocks is illustrated in Figure \ref{fig:case5}. Let $e_5(b_1,b_2)$ be the event of occurring $\left< A_n=i ,\, A_{n+k}=j,b_n=b,\,b_{n+k}=b^\prime\right>$ which is written as
\begin{align}
\begin{split}
  \label{eq:e_5}
  e_5 (b,b^\prime) := 
  &\left< b_{n+k-j} = b , b_{n-i} = b^\prime , b_{n} = b^\prime , b_{n+k} = b \right> \\
  &\land \left< b_{n+k-j+1} \neq b, \dots, b_{n-i-1} \neq b \right> \\
  &\land \left< b_{n-i+1} \neq b, \dots, b_{n-1} \neq b \right> \\
  &\land \left< b_{n-i+1} \neq b^\prime, \dots, b_{n-1} \neq b^\prime \right> \\
  &\land \left< b_{n+1} \neq b , \dots, b_{n+k-1} \neq b \right>.
\end{split}
\end{align}
Since the blocks are statistically independent and uniformly distributed, we have
\begin{align}
\begin{split}
  \label{eq:probability_e5}
  \mathrm{Pr} \left[ e_5(b_1,b_2) \right] 
  =& w_{r_1}^2  w_{r_2}^2 
  (1-w_{r_1})^{-i+j-k-1} 
  (1-w_{r_1}-w_{r_2})^{i-1}
  (1-w_{r_2})^{k-1}. 
\end{split}
\end{align}
We define $\mathcal{E}_5$ as the event $\left< A_n=i ,\, A_{n+k}=j\right>$ for $j \geq k+i+1$. Then, $\mathcal{E}_5$ can be written by using $e_5(b_1,b_2)$ as
\begin{align}\label{eq:E_5}
  \mathcal{E}_5 = \bigvee_{b \in B^L} \bigvee_{b^\prime \in B^L \setminus \{b_1\}} e_5(b,b^\prime).
\end{align}
%
Therefore, we can derive the joint distribution as
\begin{align}\label{eq:joint_dist_5}
\begin{split}
  &\mathrm{Pr} [A_n=i,\, A_{n+k}=j] \\
  &=\mathrm{Pr} [\mathcal{E}_5] \\
  &= \mathrm{Pr} \left[ \bigvee_{b \in B^L} \bigvee_{b^\prime \in B^L \setminus \{b\}}
  e_5(b,b^\prime) \right] \\
  %
  &=\sum_{b \in B^L} \sum_{b^\prime \in B^{L} \setminus \{ b \}} \mathrm{Pr} \left[ e_5(b,b^\prime) \right] \\
  %
  &= \sum_{r_1=0}^{L} \sum_{b \in B^L_{r_1}} \sum_{r_2=0}^{L} \sum_{b^\prime \in B^L_{r_2} \setminus \{ b \}} \mathrm{Pr} \left[ e_5(b,b^\prime) \right] \\
  %
  &= \sum_{r_1=0}^{L} \sum_{r_2 \neq r_1} \sum_{b \in B^L_{r_1}} \sum_{b^\prime \in B^L_{r_2}} \mathrm{Pr} \left[e_5(b_1,b_2) \right] 
  %
  + \sum_{r_1=0}^{L} \sum_{r_2 \in \{r_1\}} \sum_{b \in B^L_{r_1}} \sum_{b^\prime \in B^L_{r_1} \setminus \{ b \}} \mathrm{Pr} \left[e_5(b,b^\prime) \right] \\
  %
  &= \sum_{r_1=0}^{L} \sum_{r_2 \neq r_1} \dbinom{L}{r_1} \dbinom{L}{r_2} \mathrm{Pr} \left[e_5(b,b^\prime) \right] 
  %
  + \sum_{r_1=0}^{L}\sum_{r_2\in\{r_1\}} \dbinom{L}{r_1} \left\{ \dbinom{L}{r_1} -1 \right\} \mathrm{Pr} \left[e_5(b,b^\prime) \right],
\end{split}
\end{align}
where $\mathrm{Pr} \left[e_5(b_1,b_2) \right]$ is obtained in Eq. (\ref{eq:probability_e5}).
The above relations have been obtained in the same manner in the case of $k+1 \leq j \leq k+i-1$.
%
\begin{figure}
\centering
  \begin{tikzpicture}
    \draw (-0.3,0)--(0,0);
    \draw (-0.3,0.6)--(0,0.6);
    \draw (0,0) rectangle (0.6,0.6);
    \filldraw [draw=black, fill=yellow!60] (0.6,0) rectangle (0.6*2, 0.6)node (A) at (0.9,0.6) [above]{$b_{n+k-j}$};
    \filldraw [pattern = north east lines] (0.6*2, 0) rectangle (0.6*3, 0.6);
    \filldraw [pattern = north east lines] (0.6*3, 0) rectangle (0.6*4, 0.6);
    \filldraw [draw=black, fill=pink] (0.6*4, 0) rectangle (0.6*5, 0.6)node (B) at (2.7,0.6) [above]{$b_{n-i}$};
    \filldraw [pattern = crosshatch] (0.6*5, 0) rectangle (0.6*6, 0.6);
    \filldraw [pattern = crosshatch] (0.6*6, 0) rectangle (0.6*7, 0.6);
    \filldraw [pattern = crosshatch] (0.6*7, 0) rectangle (0.6*8, 0.6);
    \filldraw [pattern = crosshatch] (0.6*8, 0) rectangle (0.6*9, 0.6);
    \filldraw [pattern = crosshatch] (0.6*9, 0) rectangle (0.6*10, 0.6);
    \filldraw [draw=black, fill=pink] (0.6*10, 0) rectangle (0.6*11, 0.6) node (BB) at (6.3,0.6)[above]{$b_n$};
    \filldraw [pattern = north east lines] (0.6*11, 0) rectangle (0.6*12, 0.6);
    \filldraw [pattern = north east lines] (0.6*12, 0) rectangle (0.6*13, 0.6);
    \filldraw [pattern = north east lines] (0.6*13, 0) rectangle (0.6*14, 0.6);
    \filldraw [draw=black, fill=yellow!60] (0.6*14, 0) rectangle (0.6*15, 0.6)node (AA) at (8.7,0.6)[above]{$b_{n+k}$};
    \draw (0.6*15, 0) rectangle (0.6*16, 0.6);
    \draw (0.6*16, 0)--(0.6*16.5, 0);
    \draw (0.6*16, 0.6)--(0.6*16.5, 0.6);
    %
    \draw[<->] (A) to[bend left=25] node [above] {\small $b_{n+k-j} = b_{n+k}$} (AA);
    \draw[<->] (B) to[bend left=15] node [above] {\small $b_{n-i} = b_{n}$} (BB);
  \end{tikzpicture}
  \caption{An example of the arrangement of blocks in the case of $j \geq k+i+1$}
  \label{fig:case5}
\end{figure}
%-----------------------------------------------------------------------------------------------%
%-----------------------------------------------------------------------------------------------%
%-----------------------------------------------------------------------------------------------%
\subsubsection{Summary of the results}
%これまでのサブセクションで系列Aに関する周辺分布と同時分布について導出してきた.周辺分布は,An=iである事象を考えることによって導出され,式()で与えられる,一方,同時分布はブロックの配置によって5つの場合に分けて解析する必要があった.ここに,結果を再掲する.
In the previous subsections, we have derived the marginal distribution of $A_n$ and the joint distribution of $(A_n,A_{n+k})$ under the assumption of $Q\to\infty$. 
We give the summary of the result of the joint distribution that we have obtained in subsection \ref{subsec:4-2}.
The joint distribution of $(A_n,A_{n+k})$ is written as follows:
\begin{align}\begin{split}\label{eq:joint_distribution}
  &\mathrm{Pr}[A_n=i,\,A_{n+k}=j] \\
  %%%---case1---
  &= \left\{ \begin{array}{ll}
  % \displaystyle\left( \sum_{r=0}^{L} \binom{L}{r}\mathcal{Q}_r^2 (1-\mathcal{Q}_r)^{i-1} \right) \times \left( \sum_{r=0}^{L} \binom{L}{r}\mathcal{Q}_r^2 (1-\mathcal{Q}_r)^{j-1} \right) & (1\leq j \leq k-1)\\
  \displaystyle\left( \sum_{r=0}^{L} \binom{L}{r}w_r^2 (1-w_r)^{i-1} \right) \times \left( \sum_{r=0}^{L} \binom{L}{r}w_r^2 (1-w_r)^{j-1} \right) & (1\leq j \leq k-1)\\
  %%%---case2---
  % \displaystyle\sum_{r=0}^{L} \dbinom{L}{r} \mathcal{Q}_{r}^3 (1-\mathcal{Q}_{r})^{i+k-2} & (j=k) \\
  \displaystyle\sum_{r=0}^{L} \dbinom{L}{r} w_{r}^3 (1-w_{r})^{i+k-2} & (j=k) \\
  %%%---case3---
  \displaystyle\sum_{r_1=0}^{L} \sum_{r_2 \neq r_1} \dbinom{L}{r_1} \dbinom{L}{r_2} \mathrm{Pr} \left[e_3(b,b^\prime) \right] \\
  %
  + \displaystyle\sum_{r_1=0}^{L}\sum_{r_2\in\{r_1\}} \dbinom{L}{r_1} \left\{ \dbinom{L}{r_1} -1 \right\} \mathrm{Pr} \left[e_3(b,b^\prime) \right] & (k+1 \leq j \leq k+i-1)\\
  %%%---case4---
  0 & (j=k+i) \\
  %%%---case5---
  \displaystyle\sum_{r_1=0}^{L} \sum_{r_2 \neq r_1} \dbinom{L}{r_1} \dbinom{L}{r_2} \mathrm{Pr} \left[e_5(b,b^\prime) \right] \\
  %
  \displaystyle+ \sum_{r_1=0}^{L}\sum_{r_2\in\{r_1\}} \dbinom{L}{r_1} \left\{ \dbinom{L}{r_1} -1 \right\} \mathrm{Pr} \left[e_5(b,b^\prime) \right] & (j \geq k+i+1)
  \end{array}\right.,
\end{split}\end{align}
where $w_r=\hat{q}^r(1-\hat{q})^{L-r}$. In Eq.(\ref{eq:joint_distribution}), $\mathrm{Pr} \left[e_3(b,b^\prime) \right]$ and $\mathrm{Pr} \left[e_5(b,b^\prime) \right]$ are given in Eqs. (\ref{eq:probability_e3}) and (\ref{eq:probability_e5}), respectively.
%-----------------------------------------------------------------------------------------------%
%-----------------------------------------------------------------------------------------------%
%-----------------------------------------------------------------------------------------------%
%-----------------------------------------------------------------------------------------------%
%-----------------------------------------------------------------------------------------------%
%-----------------------------------------------------------------------------------------------%
%-----------------------------------------------------------------------------------------------%
%-----------------------------------------------------------------------------------------------%
%-----------------------------------------------------------------------------------------------%
%-----------------------------------------------------------------------------------------------%
%-----------------------------------------------------------------------------------------------%
%-----------------------------------------------------------------------------------------------%
%-----------------------------------------------------------------------------------------------%
%-----------------------------------------------------------------------------------------------%
%-----------------------------------------------------------------------------------------------%
%-----------------------------------------------------------------------------------------------%
%-----------------------------------------------------------------------------------------------%
%-----------------------------------------------------------------------------------------------%
\newpage
\section{The variance of references distribution}\label{sec:3}
In Section \ref{sec:distribution}, we have obtained the marginal distribution of $A_n$ and the joint distribution of $(A_n,\,A_{n+k})$ for any $\hat{q}\in (0,1)$.
%
In this section, we provide a theoretical deviation for the variance of reference distribution of the highly sensitive test under the null hypothesis using the results given in Section \ref{sec:distribution}.
%-----------------------------------------------------------------------------------------------%
%-----------------------------------------------------------------------------------------------%
%-----------------------------------------------------------------------------------------------%
\subsection{Theoretical derivation of the variance}
The variance of a random variable $X$ is defined by
\begin{align}
\begin{split}\label{eq:var_def}
  \mathrm{Var}[X] &= \mathbb{E}[(X - \mathbb{E}[X])^2] \\
  &=\mathbb{E}[X^2] - (\mathbb{E}[X])^2,
\end{split}
\end{align}
where $\mathbb{E}[X]$ is the expected value of $X$ and $\mathrm{Var}[X]$ is the variance of $X$.
In general, for any random variables $X_1,X_2,\dots,X_n$, the variance of the sum of $n$ variables is obtained by
\begin{align}\label{eq:sum_var}
  \mathrm{Var}\left[\sum_{i=1}^{n} X_i \right] = \sum_{i=1}^{n} \mathrm{Var}[X_i] + 2 \sum_{1\leq i < j \leq n}\mathrm{Cov}[X_i, X_j],
\end{align}
where $\mathrm{Cov}[X,Y]$ is the covariance defined by
\begin{align}\label{eq:cov_def}
  \mathrm{Cov}[X, Y] = \mathbb{E}[XY] - \mathbb{E}[X] \times \mathbb{E}[Y].
\end{align}
Let $\sigma_{C,\hat{q}}(K)^2$ be the variance for the reference distribution of the highly sensitive test with $\hat{q}$.
Using Eq. (\ref{eq:sum_var}), $\sigma_{C,\hat{q}}(K)^2$ is written as
\begin{align}
\begin{split}\label{eq:var_fC}
  \sigma_{C,\hat{q}}(K)^2 =& \mathrm{Var} [f_C(\hat{x}^n)] \\
  =& \mathrm{Var} \left[ \frac{1}{K} \sum_{n=1}^{K} g(A_n) \right] \\
  =& \frac{1}{K^2} \left( \sum_{n=1}^{K} \mathrm{Var} [g(A_n)] + 2 \sum_{1 \leq i < j \leq K} \mathrm{Cov} [g(A_{i}), g(A_{j})] \right) \\
  =& \frac{1}{K^2} \Biggl( K \times \mathrm{Var} [g(A_n)] + 2\sum_{k=1}^{K-1}(K-k) \times \mathrm{Cov}[g(A_n),g(A_{n+k})] \Biggr).
\end{split}
\end{align}
The last equality in Eq. (\ref{eq:var_fC}) has been obtained with the fact that the sequence of $\{A_k\}_{k=1}^{K}$ is stationary ergodic under the assumption $Q\to\infty$.
\par
In the next place, we derive $\mathrm{Var} [g(A_n)]$ and $\mathrm{Cov}[g(A_n),g(A_{n+k})]$ in Eq. (\ref{eq:var_fC}). First, $\mathrm{Var} [g(A_n)]$ can be written as
\begin{align}\label{eq:var_g_def}
  \mathrm{Var}[g(A_n)] 
  &= \mathbb{E}[\{g(A_n)\}^2] - (\mathbb{E}[g(A_n)])^2 
\end{align}
by Eq. (\ref{eq:var_def}). From the definition of expected value, the first term of the right hand side of Eq. (\ref{eq:var_g_def}) can be calculated as
\begin{align}\begin{split}\label{eq:expectation_g_An_square}
  \mathbb{E}[\{g(A_n)\}^2] &= \sum_{i=1}^{\infty} \{g(i)\}^2 \mathrm{Pr}[A_n=i]\\
  &=\sum_{i=2}^{\infty} \left[\left\{ (\log_2 \mathrm{e}) \sum_{k=1}^{i-1} \frac{1}{k} \right\}^2 \times \sum_{r=0}^{L} \binom{L}{r} w_r^2 (1-w_r)^{i-1}\right]. 
\end{split}\end{align}
The second term of the right hand side of Eq. (\ref{eq:var_g_def}) is equal to $\{L \times H(\hat{q})\}^2$ from Eq. (\ref{eq:E_BMS}).
%
Secondly, $\mathrm{Cov}[g(A_n),g(A_{n+k})]$ in Eq. (\ref{eq:var_fC}) can be calculated as
\begin{align}\label{eq:covariance_g_g}
\begin{split}
  \mathrm{Cov}[g(A_n),g(A_{n+k})] 
  &= \mathbb{E}[g(A_n) g(A_{n+k})] - \mathbb{E}[g(A_n)]\times\mathbb{E}[g(A_{n+k})] \\
  &= \sum_{i=1}^{\infty}\sum_{j=1}^{\infty}g(i)g(j)\mathrm{Pr}[A_n=i, \, A_{n+k}=j] - \left\{L \times H(\hat{q})\right\}^2
\end{split}
\end{align}
where $\mathrm{Pr}[A_n=i, \, A_{n+k}=j]$ has been obtained in Eq. (\ref{eq:joint_distribution}).
The first equality in Eq. (\ref{eq:covariance_g_g}) has been obtained from Eq. (\ref{eq:cov_def}). The second equality in Eq. (\ref{eq:covariance_g_g}) has been obtained from the definition of the expected value and Eq. (\ref{eq:E_BMS}).
\par
By combining Eqs. (\ref{eq:var_fC})--(\ref{eq:covariance_g_g}), $\sigma_{C,q}(K)^2$ is rewritten as
\begin{align}\begin{split}\label{eq:sigma_C^2}
  \sigma_{C,\hat{q}}(K)^2  
  =& \frac{1}{K} \left( \sum_{i=2}^{\infty} \left[\left\{ (\log_2 \mathrm{e}) \sum_{k=1}^{i-1} \frac{1}{k} \right\}^2 \times \sum_{r=0}^{L} \binom{L}{r} w_r^2 (1-w_r)^{i-1}\right] - \{L \times H(\hat{q})\}^2 \right)\\
  &+ \frac{2}{K^2}\sum_{k=1}^{K-1}(K-k) \left\{\sum_{i=1}^{\infty}\sum_{j=1}^{\infty}g(i)g(j)\mathrm{Pr}[A_n=i, \, A_{n+k}=j] - \left\{L \times H(\hat{q})\right\}^2\right\}.
\end{split}\end{align}
%-----------------------------------------------------------------------------------------------%
%-----------------------------------------------------------------------------------------------%
%-----------------------------------------------------------------------------------------------%
\clearpage
\subsection{Numerical results}\label{subsec:numerical_exp_L4}
In this subsection, we show some results of experiments. 
In the following, we approximate the infinite double sum as finite sum when we compute $\sigma_{C,\hat{q}} (K)^2$ by Eq. (\ref{eq:sigma_C^2}) since it is unable to compute an infinite summation by computer. Even if computing infinite sum is inevitable, we explore the covariance given in (\ref{eq:covariance_g_g}) to calculate the value more efficiently by computational experiment in Appendix \ref{appendix:B}.
%
\subsubsection{Experiment 1}
%
We confirm that $\sigma_{C,\hat{q}} (K)^2$ can be computed accurately by Eq. (\ref{eq:sigma_C^2}) when $L=4$. In the numerical computation, we set $Q=10 \times 2^L$.
Figure \ref{fig:1} shows the result of computed variance of reference distribution when $\hat{q}=0.33,\, 0.4$ and $0.5$. Note that we set $\hat{q}=0.33$ since it is said to be optimal for detecting the deviation of a binary sequence in \cite{yamamoto2016highly}. As seen in the Figure \ref{fig:1}, the variance decreases by $\mathcal{O}(\frac{1}{K})$. We can express the variance as $\frac{D_K(\hat{q})}{K}$. Table \ref{tab:1} shows the coefficient $D_K(\hat{q})$ when we approximate variance by $\frac{D_K(\hat{q})}{K}$. 
\par
In the next place, we show the result of the numerical experiment for computing an unbiased variance of binary sequences generated by a pseudo random generator. The procedure of the experiment is as follows.
\begin{enumerate}[Step1:]
  \item Set $L,\,Q,\,K,\,\hat{q},\,M$ and $N$.
  \item Generate $M$ pieces of binary sequences $x^{n,1},\dots x^{n,M}$ by pseudo random number generator, where $x^{n,i}$ for $i=1,2,\dots,M$ is a binary sequence of length $n=L\times(Q+K)$.
  \item Convert each binary sequence $x^{n,i}$ into $\hat{x}^{n,i}$ with $\hat{q}$ from Eq. (\ref{eq:convert}) by using pseudo random number generator.
  \item For each converted binary sequence $\hat{x}^{n,i}$, compute the test statistical value $f_i=f_C(\hat{x}^{n,i})$ from Eq. (\ref{eq:fC}).
  \item Compute an unbiased variance defined by
  \begin{align}
    u^2 = \frac{1}{M-1}\sum_{i=1}^{M}(f_i - \overline{f})^2,
  \end{align}
  where $\overline{f}$ is the arithmetic mean of $f_1,f_2,\dots,f_M$.
  \item Repeat Step2 to Step4 in $N$ times, and obtain $N$ unbiased variances $u_1^2,u_2^2,\dots,u_N^2$. Then, compute the arithmetic mean value of unbiased variances by
  \begin{align}
    \overline{u}^2 = \frac{1}{N} \sum_{i=1}^{N} u_i^2.
  \end{align}
\end{enumerate}
%
In the numerical simulation, we set $L=4,\,Q=10\times 2^L,\,\hat{q}=0.33,\, M=1000$ and $N=30$. We also set $K$ as $10^3,\,2\times 10^3 ,\, 4\times 10^3,\, 6\times 10^3,\, 8\times 10^3,\, 10\times 10^3,\,12\times 10^3,\,14\times 10^3$ and $16 \times 10^3$. We used Mersenne Twister as the pseudo random number generator in Step2 and Step3.
Figure \ref{fig:2} shows the result of the experiment, and we can confirm that the simulated unbiased variance coincides with the result obtained by Eq. (\ref{eq:sigma_C^2}) in precisely.
%
\begin{figure}[htbp]
  \centering
    \includegraphics[width=0.8\linewidth]{./figure/fig3.pdf}
    \caption{The variance of the reference distribution computed based on Eq. (\ref{eq:sigma_C^2}) with $\hat{q}=0.33,\, 0.4$ and $0.5$ (Copyright(C)2020 IEICE, \cite{hikima2020} Figure 1)}
    \label{fig:1}
\end{figure}
%
\begin{table}[htbp]
  \centering
  \caption{$D_K(\hat{q})$ for different values of $\hat{q}$ and $K$}
  \begin{tabular}{ccccc} \hline
    $\hat{q}$ & $D_{10000}(\hat{q})$ & $D_{20000}(\hat{q})$ & $D_{30000}(\hat{q})$ & $D_{40000}(\hat{q})$  \\ \hline 
    $0.33$    & $1.867364$     & $1.865492$     & $1.864868$     & $1.864556$\\
    $0.4$     & $1.328692$     & $1.327430$     & $1.327009$     & $1.326799$\\
    $0.5$     & $1.028395$     & $1.027449$     & $1.027134$     & $1.026976$\\ \hline
  \end{tabular}
  \label{tab:1}
\end{table}
%
\begin{figure}[htbp]
  \centering
    \includegraphics[width=0.7\linewidth]{./figure/fig4.pdf}
    \caption{The variance of the reference distribution computed in the experiment. The solid blue line and the broken red line are the variances computed based on Eq. (\ref{eq:sigma_C^2}) with $\hat{q}=0.5$ and $\hat{q}=0.33$, respectively. The block points show the arithmetic mean of the unbiased variance with $\hat{q}=0.33$ using Mersenne Twister. (Copyright(C)2020 IEICE, \cite{hikima2020} Figure 2)}
    \label{fig:2}
\end{figure}
%-----------------------------------------------------------------------------------------------%
%-----------------------------------------------------------------------------------------------%
%-----------------------------------------------------------------------------------------------%
\clearpage
\subsubsection{Experiment 2}\label{subsec:4-3}
We have seen that the variance of reference distribution of the highly sensitive test can be computed when $L=4$. We consider the case of $L=8$ and $\hat{q}=0.33$ which are recommended in \cite{yamamoto2016highly}. 
However, the computational cost for $L=8$ is too high to compute directly since the recommendation value for $K$ is $1000\times 2^8 = 256000$.
To overcome the obstacle, we derive the fitted curve. 
We approximate the variance of the reference distribution as
\begin{align}\label{eq:approx_sigma}
  \sigma_{C,\hat{q}}(K)^2 = \frac{1}{K} \left( a + \frac{b}{K} \right),
\end{align}
where $a$ and $b$ are real valued constants. 
These constants can be obtained with any two points $(K_1, \sigma_{C,\hat{q}}(K_1)^2)$ and $(K_2, \sigma_{C,\hat{q}}(K_2)^2)$ by
\begin{align}\begin{split}\label{eq:keisuu_ab}
  a &= \frac{1}{K_1-K_2} \left( K_1^2 \sigma_{C,\hat{q}}(K_1)^2 - K_2^2 \sigma_{C,\hat{q}}(K_2)^2  \right), \\
  b &= \frac{K_1K_2}{K_2-K_1} \left( K_1 \sigma_{C,\hat{q}}(K_1)^2 - K_2 \sigma_{C,\hat{q}}(K_2)^2 \right).
\end{split}\end{align}
%
Table \ref{tab:2} shows the pairs of $(a,b)$ obtained from $(K_1,K_2)=(40000,45000)$ and the standard deviation $\tilde{\sigma}_{C,\hat{q}}(K)$ for $K=1000\times2^L$ obtained from Eq. (\ref{eq:approx_sigma}).
%
Figure \ref{eq:approx_sigma} show the fitted curve.
Using the fitted curve, we obtained the standard deviation for $K=1000\times 2^8$ as
\begin{align}\label{eq:proposed_value}
  \sigma_{C,0.33} (1000\times 2^8) = 0.003488600339.
\end{align}
%
%
\par
To confirm the accuracy of Eq. (\ref{eq:proposed_value}), we computed $\sigma_{C,0.33} (1000\times 2^8)$ using MT in 10 times. For each trial, we used $4\times 10^6$ pieces of binary sequences and set $Q=10\times 2^8$.
Table \ref{tab:2} and Figure \ref{fig:comparison_yamamoto} show the results of this experiment.
These results support that the value represented in Eq. (\ref{eq:proposed_value}) is more accurate than the value in previous study.
%
\begin{table}[htb]
  \centering
  \caption{The pairs of $(a,b)$ when $(K_1,K_2)=(40000,45000)$ and the standard deviation obtained from Eq. (\ref{eq:approx_sigma})}
  \begin{tabular}{ccc} \hline
    $(K_1,K_2)$      & $(a,b)$                            & $\tilde{\sigma}_{C,q}(K)$   \\ \hline 
    % $(20000,30000)$  & $(3.112120333856, 897.0609512838)$ & $0.003488611201$      \\
    % $(30000,40000)$  & $(3.112100900335, 897.6439269103)$ & $0.003488601597$      \\
    $(40000,45000)$  & $(3.112098237555, 897.7504381251)$ & $0.003488600339$      \\ \hline
    % $(45000,46000)$  & $(3.112097857417, 897.7675443506)$ & $0.003488600164$      \\ \hline
  \end{tabular}
  \label{tab:2}
\end{table}
%
\begin{figure}[htbp]
  \centering
    \includegraphics[width=0.7\linewidth]{./figure/approx_varf_L8_10000.pdf}
    \caption{The fitted curve}
    \label{fig:fitted}
\end{figure}
%
%-----------------------------------------------------------------------------------------------%
%
\begin{table}[tbp]
  \centering
  \caption{Value of $\sigma_{C,0.33}(1000\times 2^8)$ computed using the Mersenne Twister}
  \begin{tabular}{cc} \hline
    Trial No.   & $\sigma_{C,0.33}(1000\times 2^8)$           \\ \hline 
    1           & $0.00348911$       \\
    2           & $0.00348889$       \\
    3           & $0.00348837$       \\ 
    4           & $0.00349002$       \\ 
    5           & $0.00348612$       \\ 
    6           & $0.00348572$       \\ 
    7           & $0.00348889$       \\ 
    8           & $0.00348767$       \\ 
    9           & $0.00348672$       \\ 
    10          & $0.00349002$       \\ \hline 
    Total       & $0.00348816 \pm 2.44 \times 10^{-6}$ \\ \hline
  \end{tabular}
  \label{tab:3}
\end{table}
%
\begin{figure}[tbp]
  \centering
  \includegraphics[width=0.7\linewidth]{./figure/unbiased_variance.pdf}
  \caption{The variance $\sigma_{C,0.33}(1000\times 2^8)$. Each point shows an unbiased variance derived empirically and the red broken line shows the arithmetic mean. The blue broken line shows the value given in \cite{yamamoto2016highly}. The black broken line represented in Eq. (\ref{eq:proposed_value}).}
  \label{fig:comparison_yamamoto}
\end{figure}
%-----------------------------------------------------------------------------------------------%
\clearpage
\subsubsection{Experiment 3}
We investigated the difference between the value derived in Experiment 2 and the value given in \cite{yamamoto2016highly}. In the follows, we refer to the value represented in Eq. (\ref{eq:proposed_value}) as proposed value and the value given in \cite{yamamoto2016highly} as Yamamoto's value. We used MT \cite{matsumoto1998mersenne} and AES-128 CTR \cite{rijmen2001advanced} as pseudo random number generators. 
\par
Each test was performed for $10^5$ sequences of length $n=2068480$-bit. We divided them into $100$ sets of $1000$ sequences. We assigned pass or rejected for each set based on two-level tests, ``proportion test'' and ``uniformity test'', described in subsection \ref{subsec:1-2}.
%
Recall that the significance level for uniformity test is $0.0001$. 
Since the significance level for uniformity test is so small that we cannot observe the difference, we executed uniformity test with the significance levels $0.01$ and $0.05$.
By the same reason, we additionally performed proportion test with $\xi=2$ as well as $\xi=3$.
\par
The results of Experiment 2 imply that more sequences should be used to observe the differences between the highly sensitive test with the proposed value and the test with the Yamamoto's value. We cannot expect to get any meaningful results because there are some approximation errors. For instance, we assume that the p-value takes any real value in $[0,1]$, but practically it can take only discrete values. When we use an enormous number of sequences, we cannot avoid the effect of such errors.
%
Thus, the purpose of the experiments is only to confirm that the proposed value does not cause any problem in practical situation.
\par
Tables \ref{tab:proportion_1} and \ref{tab:proportion_2} show the results of proportion test with $\xi=3$ and with $\xi=2$, respectively. Tables \ref{tab:uniformity_1}, \ref{tab:uniformity_2} and \ref{tab:uniformity_3} show the results of uniformity test with the significance levels $0.0001$, $0.01$ and $0.05$, respectively. Note that ``MT/AES'' means that a tested sequence is generated by Mersenne Twister and AES-128 CTR is used for flipping each bit. ``AES/MT'' and ``AES/AES'' are defined in the same manner.
The results in Tables \ref{tab:proportion_1}, \ref{tab:proportion_2}, \ref{tab:uniformity_1}, \ref{tab:uniformity_2} and \ref{tab:uniformity_3} support that the proposed value is robust.
%-----------------------------------------------------------------------------------------------%
\begin{table}[htb]
  \centering
  \caption{Number of sets rejected by proportion test with $\xi=3$}
  \begin{tabular}{ccc} \hline
              & Yamamoto \cite{yamamoto2016highly}  & Proposed \\ \hline 
    MT/AES    & 0         & 0        \\
    AES/MT    & 0         & 0        \\
    AES/AES   & 0         & 0        \\ \hline 
  \end{tabular}
  \label{tab:proportion_1}
\end{table}
%-----------------------------------------------------------------------------------------------%
\begin{table}[htb]
  \centering
  \caption{Number of sets rejected by proportion test with $\xi=2$}
  \begin{tabular}{ccc} \hline
              & Yamamoto \cite{yamamoto2016highly} & Proposed \\ \hline 
    MT/AES    & 2         & 3        \\
    AES/MT    & 4         & 4        \\
    AES/AES   & 6         & 6        \\ \hline 
  \end{tabular}
  \label{tab:proportion_2}
\end{table}
%-----------------------------------------------------------------------------------------------%
\begin{table}[t]
  \centering
  \caption{Number of sets rejected by uniformity test with the significance level $0.0001$}
  \begin{tabular}{ccc} \hline
              & Yamamoto \cite{yamamoto2016highly} & Proposed \\ \hline 
    MT/AES    & 0         & 0        \\
    AES/MT    & 0         & 0        \\
    AES/AES   & 0         & 0        \\ \hline 
  \end{tabular}
  \label{tab:uniformity_1}
\end{table}
%-----------------------------------------------------------------------------------------------%
\begin{table}[t]
  \centering
  \caption{Number of sets rejected by uniformity test with the significance level $0.01$}
  \begin{tabular}{ccc} \hline
              & Yamamoto \cite{yamamoto2016highly} & Proposed \\ \hline 
    MT/AES    & 0         & 0        \\
    AES/MT    & 0         & 1        \\
    AES/AES   & 0         & 0        \\ \hline 
  \end{tabular}
  \label{tab:uniformity_2}
\end{table}
%-----------------------------------------------------------------------------------------------%
\begin{table}[t]
  \centering
  \caption{Number of sets rejected by uniformity test with the significance level $0.05$}
  \begin{tabular}{ccc} \hline
              & Yamamoto \cite{yamamoto2016highly} & Proposed \\ \hline 
    MT/AES    & 2         & 2        \\
    AES/MT    & 3         & 6        \\
    AES/AES   & 4         & 5        \\ \hline 
  \end{tabular}
  \label{tab:uniformity_3}
\end{table}
%-----------------------------------------------------------------------------------------------%
%-----------------------------------------------------------------------------------------------%
%-----------------------------------------------------------------------------------------------%
%-----------------------------------------------------------------------------------------------%
%-----------------------------------------------------------------------------------------------%
%-----------------------------------------------------------------------------------------------%
%-----------------------------------------------------------------------------------------------%
%-----------------------------------------------------------------------------------------------%
%-----------------------------------------------------------------------------------------------%
\clearpage
\section{Conclusion}\label{sec:conclusion}
In this thesis, we have theoretically derived the variance for the reference distribution of highly sensitive universal statistical test. 
%
In deriving process, the marginal distribution of $A_n$ and the joint distribution of $(A_n,A_{n+k})$ have been derived theoretically with the fact that the sequence of $\{A_k\}_{k=1}^{K}$ can be considered as stationary ergodic under the assumption of $Q\to\infty$. 
%
We have shown that the value of the variance can be numerically computed for the block size $L=4$.
%
Because of computational cost, we used an fitted curve to get the numerical value of the variance for $L=8$.
%
Since the value obtained by the fitted curve may have some error, we have compared with the unbiased variance computed from binary sequences generated by a pseudo random number generator. By this experiment, we have confirmed that the value obtained from the fitted curve is more consistent with the experimental result than the existing value which has been obtained by a numerical experiment.
%
We can state that the value of the variance in this thesis is superior subject to the existing one. 
%
Thus, we recommend that the value obtained in this thesis should be used when the highly sensitive universal statistical test is performed so that randomness of binary sequences can be tested more precisely.
%-- Acknowledgments -------------------------------------------------------------
\clearpage
\acknowledgment
The author would like to express his sincere gratitude to Professor Ken Umeno for his guidance and helpful advice.
%
I would also like to express my appreciation for Assistant Professor Atsushi Iwasaki for his support and encouragement in keeping my progress.
%
\par
Furthermore, I would like to extend my thanks to the member of Physical Statistical Laboratory for their support during my study. 
%
I would especially grateful to Dr. Hirofumi Tsuda and Mr. Shinji Kakinaka for their extended discussions and valuable suggestions which have greatly contributed to the improvement of this thesis.
%
\par
Finally, I wish to thank my family for their support and encouragement throughout my study.
%-- References ------------------------------------------------------------------
\clearpage
\addcontentsline{toc}{section}{\refname} % Add to the table of contents.
                                         % Delete if you use chapter option.
\bibliographystyle{ieeetr}
\bibliography{cite}
%-- Appendix ---------------------------------------------------------------------
%%% If you don't need appendices, delete the below.
\clearpage
\appendix
\section{Proof of $C=-\frac{\ln 2}{\gamma}$}\label{appendix:A}
In this appendix, we provide the proof of the following theorem.
\begin{theorem}
For any $x\in(0,1)$, we have the following relation
\begin{align}
	\lim_{x\to 0+} \left[ v(x) + \log_2 x \right] = -\frac{\gamma}{\ln 2} = -0.832746\cdots,
\end{align}
where
\begin{align}
	v(x) = x\sum_{i=1}^{\infty} (1-x)^{i-1}\log_2 i,
\end{align}
and $\gamma$ is Euler's constant defined by
\begin{align}\label{eq:gamma_def}
	\gamma := \lim_{n\to\infty}\left(\sum_{i=1}^{n} \frac{1}{i}-\ln n\right).
\end{align}
\end{theorem}
%
\begin{proof}
Let $s=1-x$. We have
\begin{align}\begin{split}
	v(1-s) + \log_2 (1-s) 
	&= (1-s)\sum_{i=1}^{\infty} s^{i-1}\log_2 i + \log_2 (1-s) \\
	&= \frac{1}{\ln2} \left\{ (1-s) \sum_{i=1}^{\infty}s^{i-1}\ln i + \ln(1-s) \right\} \\
	&= \frac{1}{\ln2} \left\{ (1-s)\times \frac{1}{1-s} \sum_{i=1}^{\infty}s^{i} \times \ln \frac{i+1}{i} - \sum_{i=1}^{\infty} \frac{s^i}{i} \right\} \\
	&= \frac{1}{\ln2} \sum_{i=1}^{\infty}s^{i} \left\{ \ln \left(1+\frac{1}{i}\right) - \frac{1}{i} \right\}.
\end{split}\end{align}
From the above equations, we have
\begin{align}\begin{split}
	\lim_{s\to 1-} \left[ v(1-s) + \log_2 (1-s) \right] 
	&=\frac{1}{\ln2} \lim_{s\to 1-} \left[ \sum_{i=1}^{\infty} s^{i} \left\{ \ln \left(1+\frac{1}{i}\right) - \frac{1}{i} \right\} \right]\\
	&=\frac{1}{\ln2} \sum_{i=1}^{\infty} \lim_{s\to 1-} \left[ s^{i} \left\{ \ln \left(1+\frac{1}{i}\right) - \frac{1}{i} \right\} \right] \\
	&=\frac{1}{\ln2} \sum_{i=1}^{\infty}\left\{ \ln \left(1+\frac{1}{i}\right) - \frac{1}{i} \right\}.
\end{split}\end{align}
For the derivation of the second equation, we have used the fact that the infinite series converges absolutely which allows us to exchange the limits. 
%
Since $\ln n$ can be written as 
\begin{align}
	\ln n = \ln \left( \frac{2}{1}\cdot\frac{3}{2}\cdot\frac{4}{3}\cdot\cdots\frac{n}{n-1}\right) = \sum_{k=1}^{n-1}\ln\left(1+\frac{1}{k}\right),
\end{align}
Euler's constant $\gamma$ defined by Eq. (\ref{eq:gamma_def}) can also be represented as
\begin{align}\begin{split}
	\gamma &= \lim_{n\to\infty} \left[ \sum_{k=1}^{n-1}\left\{ \frac{1}{k} - \ln \left(1+\frac{1}{k}\right)\right\} + \frac{1}{n} \right] \\
	&=\sum_{n=1}^{\infty}\left\{ \frac{1}{n} - \ln \left(1+\frac{1}{n}\right) \right\}.
\end{split}\end{align}
Therefore, we arrive at the following result
\begin{align}
	\lim_{s\to 1-} \left[ v(1-s) + \log_2 (1-s) \right]  =- \frac{\gamma}{\ln 2}.
\end{align}
The theorem is obtained if we substitute $s$ into $1-x$.
\end{proof}
\clearpage
% \newpage
\section{Exploration of the covariance given in Eq. (\ref{eq:covariance_g_g})}\label{appendix:B}
In this appendix, we explore the covariance given in Eq. (\ref{eq:covariance_g_g}) to calculate the value more efficiently by computational experiments. Notice that the covariance is written as 
\begin{align}\label{eq:cov_saikei}
	\mathrm{Cov}[g(A_n),\, g(A_{n+k})] 
	= \sum_{i=1}^{\infty} \sum_{j=1}^{\infty} g(i) g(j) \mathrm{Pr} \left[ A_n=i,\,A_{n+k}=j \right] - \{L\times H(\hat{q})\}^2,
\end{align}
where $g$ is given in Eq. (\ref{eq:function_g}), $\mathrm{Pr} \left[ A_n=i,\,A_{n+k}=j \right]$ is given in Eq. (\ref{eq:joint_distribution}), and $H$ is a binary entropy function.
%
Since the term $\{L\times H(\hat{q})\}^2$ is irrelevant to an infinite series, we write the first term of the right hand side (r.h.s.) of Eq. (\ref{eq:cov_saikei}) as
%
\begin{align}\label{eq:Sk}
	\overline{S}_{k} := \sum_{i=1}^{\infty} \sum_{j=1}^{\infty} g(i) g(j) \mathrm{Pr} \left[ A_n=i,\,A_{n+k}=j \right].
\end{align}
%
Firstly, we show the following Lemma concerning an infinite series to calculate the value expressed Eq. (\ref{eq:Sk}). 
%
\begin{lemma}\label{lemma:1}
% abcdefghijklmnopqrstuvwxyz
For any $z \in (0,1)$, the following relation holds
\begin{align}\label{eq:infinite_series}
	\sum_{i=1}^{\infty} g(i) \times (1-z)^i = -\frac{1}{\ln 2} \times \frac{1-z}{z} \times \ln z,
\end{align}
where
\begin{align}
	g(m) = (\log_2 \mathrm{e}) \sum_{k=1}^{m-1} \frac{1}{k}.
\end{align}
\end{lemma}
%-----------------------------------------------------------------------------------------------%
\begin{proof}
Let $t=1-z$, and $\overline{S}$ be the left hand side of Eq. (\ref{eq:infinite_series}), that is,
\begin{align}\begin{split}\label{eq:S}
  \overline{S} &= \sum_{i=1}^{\infty} g(i) \times t^i \\
    % \Leftrightarrow S &= g(1)\times A +  \sum_{i=2}^{\infty} g(i) \times A^{i} \label{eq:S}.
    &= g(1)\times t +  \sum_{i=2}^{\infty} g(i) \times t^{i}.
\end{split}\end{align}
Multiplying the both sides of Eq. (\ref{eq:S}) by $t(\neq 0)$, we obtain the following relation
\begin{align}\begin{split}\label{eq:AS}
  t\times \overline{S} &= \sum_{i=1}^{\infty} g(i) \times t^{i+1} \\
  &= \sum_{i=2}^{\infty} g(i-1) \times t^{i}.
\end{split}\end{align}
Subtracting Eq. (\ref{eq:AS}) from Eq. (\ref{eq:S}), we obtain the following relation
\begin{align}\begin{split}\label{eq:(1-A)S}
  (1-t)\overline{S} &= g(1)\times t^{1} + \sum_{i=2}^{\infty} \left\{ g(i)-g(i-1) \right\} \times t^{i} \\
  &=0 + \sum_{i=2}^{\infty} (\log_2 \mathrm{e}) \times \frac{1}{i-1} \times t^{i} \\
  &=(\log_2 \mathrm{e})\times t \times \sum_{i=1}^{\infty} \frac{t^{i}}{i} \\
  &=(\log_2 \mathrm{e})\times t \times \left\{ - \ln (1-t) \right\}.
\end{split}\end{align}
To obtain the last equality in the above equations, we have used the Taylor series for $|t| < 1$.
Dividing both sides of Eq. (\ref{eq:(1-A)S}) by $1-t (\neq 0)$, we arrive at the following result
\begin{align}
  \overline{S} &= -\frac{1}{\ln 2} \times \frac{t}{1-t} \times \ln (1-t).
\end{align}
The lemma is obtained if we substitute $t$ into $1-z$.
\end{proof}
%-----------------------------------------------------------------------------------------------%
%-----------------------------------------------------------------------------------------------%
%-----------------------------------------------------------------------------------------------%
\par
In the next place, we calculate the value given in Eq. (\ref{eq:Sk}) more in details.
\subsection{Case of $1 \leq j \leq k-1$}
For $1 \leq j \leq k-1$, Eq. (\ref{eq:Sk}) is written as
\begin{align}\begin{split}
  \overline{S}_{k} 
  &:= \sum_{i=1}^{\infty} \sum_{j=1}^{k-1} g(i) g(j) \mathrm{Pr} \left[ A_n=i,\,A_{n+k}=j \right] \\
  &= \sum_{i=1}^{\infty} \sum_{j=1}^{k-1} g(i) g(j) \left( \sum_{r=0}^{L} \binom{L}{r}w_r^2 (1-w_r)^{i-1} \right) \times \left( \sum_{r=0}^{L} \binom{L}{r}w_r^2 (1-w_r)^{j-1} \right) \\ 
  &= \sum_{r=0}^{L} \left\{\binom{L}{r} \frac{w_r^2}{1-w_r} \sum_{i=1}^{\infty} g(i)(1-w_r)^{i} \right\}
  \times \sum_{r=0}^{L} \left\{ \binom{L}{r} \frac{w_r^2}{1-w_r} \sum_{j=1}^{k-1} g(j)(1-w_r)^{j} \right\}.
\end{split}\end{align}
From Lemma \ref{lemma:1}, the infinite series of $i$ in the above equations is written as
\begin{align}
	\sum_{i=1}^{\infty} g(i) (1 - w_r)^{i} = - \frac{1}{\ln 2} \times \frac{1-w_r}{w_r}\times\ln w_r.
\end{align}
%
Therefore, we obtain the following relation
\begin{align}\begin{split}
  \overline{S}_{k}
  =& \sum_{r=0}^{L} \left\{ \dbinom{L}{r} \frac{w_r^2}{1-w_r}  \times \left(- \frac{1}{\ln 2} \times \frac{1-w_r}{w_r}\times\ln w_r \right) \right\} \\
  &\times \sum_{r=0}^{L} \left\{ \dbinom{L}{r} \frac{w_r^2}{1-w_r} \times \sum_{j=1}^{k-1} (1-w_r)^{j} \right\} \\
  =&-\frac{1}{\ln 2}\sum_{r=0}^{L} \left\{ \dbinom{L}{r} w_r \ln w_r \right\} \times \sum_{r=0}^{L} \left\{ \dbinom{L}{r} \frac{w_r^2}{1-w_r} \sum_{j=1}^{k-1} (1-w_r)^{j} \right\}.
\end{split}\end{align}
%
\subsection{Case of $j=k$}
In the case of $j=k$, Eq. (\ref{eq:Sk}) is written as
\begin{align}\begin{split}\label{eq:app_case2}
  \overline{S}_{k} 
  &= \sum_{i=1}^{\infty} g(i) \sum_{r=0}^{L} \binom{L}{r} w_r^3 (1-w_r)^{k+i-2} \sum_{j \in \{k\}} g(j)\\
  &= \sum_{r=0}^{L} \binom{L}{r} w_r^3 (1-w_r)^{k-2} \sum_{i=1}^{\infty} g(i) (1-w_r)^{i} \times g(k) \\
  &= g(k) \times \sum_{r=0}^{L} \binom{L}{r} w_r^3 (1-w_r)^{k-2} \left( -\frac{1}{\ln 2} \times \frac{1-w_{r}}{w_{r}} \times \ln w_{r} \right) \\
  &= -\frac{g(k)}{\ln 2} \times \sum_{r=0}^{L} \binom{L}{r} w_r^2 (1-w_r)^{k-1} \ln w_{r}.
\end{split}\end{align}
The third equation in Eq. (\ref{eq:app_case2}) has been obtained from Lemma \ref{lemma:1}.
%-----------------------------------------------------------------------------------------------%
%-----------------------------------------------------------------------------------------------%
%-----------------------------------------------------------------------------------------------%
\subsection{Case of $k+1 \leq j \leq k+i-1$}
Recall that the joint distribution for $k+1 \leq j \leq k+i-1$ is written as
\begin{align}\begin{split}
	\mathrm{Pr}[A_n=i,\, A_{n+k}=j] 
	=& \sum_{r_1=0}^{L} \sum_{r_2 \neq r_1} \binom{L}{r_1}\binom{L}{r_2}\mathrm{Pr}[e_3(b_1,b_2)] \\
	&+ \sum_{r_1=0}^{L} \sum_{r_2 \in \{r_1\}} \binom{L}{r_1}\left\{\binom{L}{r_1}-1\right\}\mathrm{Pr}[e_3(b_1,b_2)],
\end{split}\end{align}
where $\mathrm{Pr}[e_3(b_1,b_2)]$ is expressed as
\begin{align}\begin{split}
	\mathrm{Pr}[e_3(b_1,b_2)]
  =& w_{r_1}^2  \times w_{r_2}^2 
  \times (1-w_{r_1})^{i-j+k-1} 
  \times (1-w_{r_1}-w_{r_2})^{j-k-1}
  \times (1-w_{r_2})^{k-1} \\
  =&\phi_k(r_1,r_2)\times (1-w_{r_1})^{i} \times \left(1-\frac{w_{r_2}}{1-w_{r_1}} \right)^{j},
\end{split}\end{align}
%
where
\begin{align}\label{eq:phi_k}
	\phi_k(r_1,r_2) = w_{r_1}^2 w_{r_2}^2 
  	(1-w_{r_1})^{k-1} 
  	(1-w_{r_1}-w_{r_2})^{-k-1}
  	(1-w_{r_2})^{k-1}.
\end{align}
%
Then, Eq. (\ref{eq:Sk}) is expressed as
\begin{align}\begin{split}\label{eq:Sk_case3}
	\overline{S}_k 
	=& \sum_{i=1}^{\infty}\sum_{j=k+1}^{k+i-1} g(i)g(j) \sum_{r_1=0}^{L} \sum_{r_2 \neq r_1} \binom{L}{r_1}\binom{L}{r_2}\mathrm{Pr}[e_3(b_1,b_2)]\\ 
	&+ \sum_{i=1}^{\infty}\sum_{j=k+1}^{k+i-1} g(i)g(j) \sum_{r_1=0}^{L} \sum_{r_2 \in \{r_1\}} \binom{L}{r_1} \left\{\binom{L}{r_1}-1 \right\}\mathrm{Pr}[e_3(b_1,b_2)].
\end{split}\end{align}
%
Now, we consider the first term in r.h.s. of Eq. (\ref{eq:Sk_case3}). Let $\overline{A}_1$ be this term of Eq. (\ref{eq:Sk_case3}). We have
\begin{align}\begin{split}\label{eq:A_1}
	\overline{A}_1
	&=\sum_{i=1}^{\infty}\sum_{j=k+1}^{k+i-1} g(i)g(j) \sum_{r_1=0}^{L} \sum_{r_2 \neq r_1} \binom{L}{r_1}\binom{L}{r_2}\mathrm{Pr}[e_3(b_1,b_2)] \\
	&=\sum_{r_1=0}^{L} \sum_{r_2 \neq r_1} \binom{L}{r_1}\binom{L}{r_2}\phi_k(r_1,r_2)
	\sum_{i=1}^{\infty} \sum_{j=1}^{k+i-1} g(i)g(j)(1-w_{r_1})^{i} \times \left(1-\frac{w_{r_2}}{1-w_{r_1}} \right)^{j} \\
	&=\sum_{r_1=0}^{L} \sum_{r_2 \neq r_1} \binom{L}{r_1}\binom{L}{r_2}\phi_k(r_1,r_2)
	\sum_{j=k+1}^{\infty} \sum_{i=k+1}^{k+i-1} g(i)g(j)(1-w_{r_1})^{i} \times \left(1-\frac{w_{r_2}}{1-w_{r_1}} \right)^{j} \\
	&=\sum_{r_1=0}^{L} \sum_{r_2 \neq r_1} \binom{L}{r_1}\binom{L}{r_2}\phi_k(r_1,r_2)
	\sum_{j=k+1}^{\infty} \left\{ g(j) \left(1-\frac{w_{r_2}}{1-w_{r_1}} \right)^{j} \times \sum_{i=j-k+1}^{\infty} g(i)(1-w_{r_1})^{i} \right\}.
\end{split}\end{align}
In the course of the derivation of the above relations, the second equation has been obtained from Eq. (\ref{eq:phi_k}). The third equation has been obtained by exchanging the summation over $i$ and $j$.
%
Then, the infinite series with respect to $i$ in the last equation of Eq. (\ref{eq:A_1}) can be calculated as
\begin{align}\begin{split}
	\sum_{i=j-k+1}^{\infty} g(i)(1-w_{r_1})^{i} 
	&= \sum_{i=1}^{\infty} g(i)(1-w_{r_1})^{i} - \sum_{i=1}^{j-k} g(i)(1-w_{r_1})^{i} \\
	&= -\frac{1}{\ln 2} \times \frac{1-w_{r_1}}{w_{r_1}} \times \ln w_{r_1} - \sum_{i=1}^{j-k} g(i)(1-w_{r_1})^{i}.
\end{split}\end{align}
In the above relations, we have used the result of Lemma \ref{lemma:1}.
Hence, the first term in r.h.s. of Eq. (\ref{eq:Sk_case3}) can be expressed as
\begin{align}
	\overline{A}_1 
	=& \sum_{r_1=0}^{L} \sum_{r_2 \neq r_1} \binom{L}{r_1}\binom{L}{r_2}\phi_k(r_1,r_2)\\
	&\times\sum_{j=k+1}^{\infty} \left[ g(j) \left(1-\frac{w_{r_2}}{1-w_{r_1}} \right)^{j} \times \left\{ -\frac{1}{\ln 2} \times \frac{1-w_{r_1}}{w_{r_1}} \times \ln w_{r_1} - \sum_{i=1}^{j-k} g(i)(1-w_{r_1})^{i} \right\} \right].
\end{align}
We can derive the second term in r.h.s. of Eq. (\ref{eq:Sk_case3}) in the same way as the first term. Let $\overline{A}_2$ be the second term of Eq. (\ref{eq:Sk_case3}). Then, we have
\begin{align}\begin{split}
	\overline{A}_2 =& \sum_{r_1=0}^{L} \sum_{r_2 \in \{r_1\}} \binom{L}{r_1}\left\{\binom{L}{r_1}-1\right\} \phi(r_1,r_2) \\
	&\times\sum_{j=k+1}^{\infty} \left[ g(j) \left(1-\frac{w_{r_2}}{1-w_{r_1}} \right)^{j} \times \left\{ -\frac{1}{\ln 2} \times \frac{1-w_{r_1}}{w_{r_1}} \times \ln w_{r_1} - \sum_{i=1}^{j-k} g(i)(1-w_{r_1})^{i} \right\} \right].
\end{split}\end{align}
%
Therefore, Eq. (\ref{eq:Sk_case3}) can be written as
\begin{align}\begin{split}
	\overline{S}_k =& \overline{A}_1 + \overline{A}_2 \\
	=& \sum_{r_1=0}^{L} \sum_{r_2 \neq r_1} \binom{L}{r_1}\binom{L}{r_2}\phi_k(r_1,r_2)\\
	&\times\sum_{j=k+1}^{\infty} \left[ g(j) \left(1-\frac{w_{r_2}}{1-w_{r_1}} \right)^{j} \times \left\{ -\frac{1}{\ln 2} \times \frac{1-w_{r_1}}{w_{r_1}} \times \ln w_{r_1} - \sum_{i=1}^{j-k} g(i)(1-w_{r_1})^{i} \right\} \right] \\
	&+\sum_{r_1=0}^{L} \sum_{r_2 \in \{r_1\}} \binom{L}{r_1}\left\{\binom{L}{r_1}-1\right\} \phi(r_1,r_2) \\
	&\times\sum_{j=k+1}^{\infty} \left[ g(j) \left(1-\frac{w_{r_2}}{1-w_{r_1}} \right)^{j} \times \left\{ -\frac{1}{\ln 2} 
	\times \frac{1-w_{r_1}}{w_{r_1}} \times \ln w_{r_1} - \sum_{i=1}^{j-k} g(i)(1-w_{r_1})^{i} \right\} \right],
\end{split}\end{align}
where $\phi_k(r_1,r_2)$ is given in Eq. (\ref{eq:phi_k}).
%-----------------------------------------------------------------------------------------------%
%-----------------------------------------------------------------------------------------------%
%-----------------------------------------------------------------------------------------------%
\subsection{Case of $j=k+i$}
Equation (\ref{eq:Sk}) in the case of $j=k+i$ is equal to $0$ from Eq. (\ref{eq:joint_distribution}).
%-----------------------------------------------------------------------------------------------%
%-----------------------------------------------------------------------------------------------%
%-----------------------------------------------------------------------------------------------%
\subsection{Case of $j \geq k+i+1$}
%--------------------------------------------------------------------------------%
Recall that the joint distribution for $j \geq k+i+1$ is written as
\begin{align}\begin{split}
  \mathrm{Pr}[A_n=i,\, A_{n+k}=j] 
  =& \sum_{r_1=0}^{L} \sum_{r_2 \neq r_1} \binom{L}{r_1}\binom{L}{r_2}\mathrm{Pr}[e_5(b_1,b_2)] \\
  &+ \sum_{r_1=0}^{L} \sum_{r_2 \in \{r_1\}} \binom{L}{r_1}\left\{\binom{L}{r_1}-1\right\}\mathrm{Pr}[e_5(b_1,b_2)],
\end{split}\end{align}
where $\mathrm{Pr}[e_5(b_1,b_2)]$ is expressed as
\begin{align}\begin{split}
  \mathrm{Pr}[e_5(b_1,b_2)]
  =& w_{r_1}^2  \times w_{r_2}^2 
  \times (1-w_{r_1})^{-i+j-k-1} 
  \times (1-w_{r_1}-w_{r_2})^{i-1}
  \times (1-w_{r_1})^{k-1} \\
  =&\psi_k(r_1,r_2)\times \left(1-\frac{w_{r_2}}{1-w_{r_1}} \right)^{i} \times (1-w_{r_1})^j,
\end{split}\end{align}
%
where
\begin{align}\begin{split}\label{eq:psi_k}
  \psi_k(r_1,r_2) 
  &= w_{r_1}^2 w_{r_2}^2 
    (1-w_{r_1})^{-k-1} 
    (1-w_{r_1}-w_{r_2})^{-1}
    (1-w_{r_1})^{k-1} \\
  &= w_{r_1}^2 w_{r_2}^2 (1-w_{r_1})^{-2} (1-w_{r_1}-w_{r_2})^{-1}.
\end{split}\end{align}
%
Then, Eq. (\ref{eq:Sk}) is expressed as
\begin{align}\begin{split}\label{eq:Sk_case5}
  \overline{S}_k 
  =& \sum_{i=1}^{\infty}\sum_{j=k+i+1}^{\infty} g(i)g(j) \sum_{r_1=0}^{L} \sum_{r_2 \neq r_1} \binom{L}{r_1}\binom{L}{r_2}\mathrm{Pr}[e_5(b_1,b_2)]\\ 
  &+ \sum_{i=1}^{\infty}\sum_{j=k+i+1}^{\infty} g(i)g(j) \sum_{r_1=0}^{L} \sum_{r_2 \in \{r_1\}} \binom{L}{r_1} \left\{\binom{L}{r_1}-1 \right\}\mathrm{Pr}[e_5(b_1,b_2)].
\end{split}\end{align}
%
Firstly, we consider the first term in the r.h.s. of Eq. (\ref{eq:Sk_case3}). Let $\overline{B}_1$ be the first term of Eq. (\ref{eq:Sk_case3}). Then, $\overline{B}_1$ is written as
\begin{align}\begin{split}\label{eq:B_1}
  \overline{B}_1
  &=\sum_{i=1}^{\infty}\sum_{j=k+i+1}^{\infty} g(i)g(j) \sum_{r_1=0}^{L} \sum_{r_2 \neq r_1} \binom{L}{r_1}\binom{L}{r_2}\mathrm{Pr}[e_5(b_1,b_2)] \\
  &=\sum_{r_1=0}^{L} \sum_{r_2 \neq r_1} \binom{L}{r_1}\binom{L}{r_2}\psi_k(r_1,r_2)
  \sum_{i=1}^{\infty} \sum_{j=k+i+1}^{\infty} g(i)g(j) \left(1-\frac{w_{r_2}}{1-w_{r_1}} \right)^{i} \times (1-w_{r_1})^j \\
  &=\sum_{r_1=0}^{L} \sum_{r_2 \neq r_1} \binom{L}{r_1}\binom{L}{r_2}\psi_k(r_1,r_2)
  \sum_{i=1}^{\infty} g(i)\left(1-\frac{w_{r_2}}{1-w_{r_1}} \right)^{i} \times \sum_{j=k+i+1}^{\infty} g(j) (1-w_{r_1})^j.
\end{split}\end{align}
In the course of the derivation of the above relations, the second equation has been obtained from Eq. (\ref{eq:phi_k}). The third equation has been obtained by exchange of infinite series.
%
Then, the infinite sum with respect to $j$ in the last equation of Eq. (\ref{eq:B_1}) can be calculated as
\begin{align}\begin{split}
  \sum_{j=k+i+1}^{\infty} g(j) (1-w_{r_1})^j 
  &= \sum_{j=1}^{\infty} g(j)(1-w_{r_1})^{j} - \sum_{j=1}^{k+i} g(j)(1-w_{r_1})^{j} \\
  &= -\frac{1}{\ln 2} \times \frac{1-w_{r_1}}{w_{r_1}} \times \ln w_{r_1} - \sum_{j=1}^{k+i} g(j)(1-w_{r_1})^{j}.
\end{split}\end{align}
In the above relations, we use the result of Lemma \ref{lemma:1}.
Hence, the first term of Eq. (\ref{eq:Sk_case5}) can be expressed as
\begin{align}
  \overline{B}_1 
  =& \sum_{r_1=0}^{L} \sum_{r_2 \neq r_1} \binom{L}{r_1}\binom{L}{r_2}\psi_k(r_1,r_2)\\
  &\times\sum_{i=1}^{\infty} \left[ g(i) \left(1-\frac{w_{r_2}}{1-w_{r_1}} \right)^{i} \times \left\{ -\frac{1}{\ln 2} \times \frac{1-w_{r_1}}{w_{r_1}} \times \ln w_{r_1} - \sum_{j=1}^{k+i} g(j)(1-w_{r_1})^{j} \right\} \right].
\end{align}
We can derive the second term of Eq. (\ref{eq:Sk_case5}) in the same way as the first term. Let $\overline{B}_2$ be the second term of Eq. (\ref{eq:Sk_case5}). Then, we have
\begin{align}\begin{split}
  \overline{B}_2 =& \sum_{r_1=0}^{L} \sum_{r_2 \in \{r_1\}} \binom{L}{r_1}\left\{\binom{L}{r_1}-1\right\} \psi(r_1,r_2) \\
  &\times\sum_{i=1}^{\infty} \left[ g(i) \left(1-\frac{w_{r_2}}{1-w_{r_1}} \right)^{i} \times \left\{ -\frac{1}{\ln 2} \times \frac{1-w_{r_1}}{w_{r_1}} \times \ln w_{r_1} - \sum_{j=1}^{k+i} g(j)(1-w_{r_1})^{j} \right\} \right].
\end{split}\end{align}
%
Therefore, Eq. (\ref{eq:Sk_case5}) can be expressed as
\begin{align}\begin{split}
  \overline{S}_k =& \overline{B}_1 + \overline{B}_2 \\
  =& \sum_{r_1=0}^{L} \sum_{r_2 \neq r_1} \binom{L}{r_1}\binom{L}{r_2}\psi_k(r_1,r_2)\\
  &\times\sum_{i=1}^{\infty} \left[ g(i) \left(1-\frac{w_{r_2}}{1-w_{r_1}} \right)^{i} \times \left\{ -\frac{1}{\ln 2} \times \frac{1-w_{r_1}}{w_{r_1}} \times \ln w_{r_1} - \sum_{j=1}^{k+i} g(j)(1-w_{r_1})^{j} \right\} \right] \\
  &+\sum_{r_1=0}^{L} \sum_{r_2 \in \{r_1\}} \binom{L}{r_1}\left\{\binom{L}{r_1}-1\right\} \psi(r_1,r_2) \\
  &\times\sum_{i=1}^{\infty} \left[ g(i) \left(1-\frac{w_{r_2}}{1-w_{r_1}} \right)^{i} \times \left\{ -\frac{1}{\ln 2} \times \frac{1-w_{r_1}}{w_{r_1}} \times \ln w_{r_1} - \sum_{j=1}^{k+i} g(j)(1-w_{r_1})^{j} \right\} \right],
\end{split}\end{align}
where $\psi_k(r_1,r_2)$ is given in Eq. (\ref{eq:psi_k}).



















\newpage
\section{Exploration of the covariance given in Eq. (\ref{eq:covariance_g_g})}\label{appendix:B}
In this appendix, we explore the covariance given in Eq. (\ref{eq:covariance_g_g}) to calculate the value more efficiently by computational experiments. Notice that the covariance is written as 
\begin{align}\label{eq:cov_saikei}
	\mathrm{Cov}[g(A_n),\, g(A_{n+k})] 
	= \sum_{i=1}^{\infty} \sum_{j=1}^{\infty} g(i) g(j) \mathrm{Pr} \left[ A_n=i,\,A_{n+k}=j \right] - \{L\times H(\hat{q})\}^2,
\end{align}
where $g$ is given in Eq. (\ref{eq:function_g}), $\mathrm{Pr} \left[ A_n=i,\,A_{n+k}=j \right]$ is given in Eq. (\ref{eq:joint_distribution}), and $H$ is a binary entropy function.
%
Since the term $\{L\times H(\hat{q})\}^2$ is irrelevant to an infinite series, we write the first term of the right hand side (r.h.s.) of Eq. (\ref{eq:cov_saikei}) as
%
\begin{align}\label{eq:Sk}
	\overline{S}_{k} := \sum_{i=1}^{\infty} \sum_{j=1}^{\infty} g(i) g(j) \mathrm{Pr} \left[ A_n=i,\,A_{n+k}=j \right].
\end{align}
%
Firstly, we show the following Lemma concerning an infinite series to calculate the value expressed Eq. (\ref{eq:Sk}). 
%
\begin{lemma}\label{lemma:1}
% abcdefghijklmnopqrstuvwxyz
For any $z \in (0,1)$, the following relation holds
\begin{align}\label{eq:infinite_series}
	\sum_{i=1}^{\infty} g(i) \times (1-z)^i = -\frac{1}{\ln 2} \times \frac{1-z}{z} \times \ln z,
\end{align}
where
\begin{align}
	g(m) = (\log_2 \mathrm{e}) \sum_{k=1}^{m-1} \frac{1}{k}.
\end{align}
\end{lemma}
%-----------------------------------------------------------------------------------------------%
\begin{proof}
Let $t=1-z$, and $\overline{S}$ be the left hand side of Eq. (\ref{eq:infinite_series}), that is,
\begin{align}\begin{split}\label{eq:S}
  \overline{S} &= \sum_{i=1}^{\infty} g(i) \times t^i \\
    % \Leftrightarrow S &= g(1)\times A +  \sum_{i=2}^{\infty} g(i) \times A^{i} \label{eq:S}.
    &= g(1)\times t +  \sum_{i=2}^{\infty} g(i) \times t^{i}.
\end{split}\end{align}
Multiplying the both sides of Eq. (\ref{eq:S}) by $t(\neq 0)$, we obtain the following relation
\begin{align}\begin{split}\label{eq:AS}
  t\times \overline{S} &= \sum_{i=1}^{\infty} g(i) \times t^{i+1} \\
  &= \sum_{i=2}^{\infty} g(i-1) \times t^{i}.
\end{split}\end{align}
Subtracting Eq. (\ref{eq:AS}) from Eq. (\ref{eq:S}), we obtain the following relation
\begin{align}\begin{split}\label{eq:(1-A)S}
  (1-t)\overline{S} &= g(1)\times t^{1} + \sum_{i=2}^{\infty} \left\{ g(i)-g(i-1) \right\} \times t^{i} \\
  &=0 + \sum_{i=2}^{\infty} (\log_2 \mathrm{e}) \times \frac{1}{i-1} \times t^{i} \\
  &=(\log_2 \mathrm{e})\times t \times \sum_{i=1}^{\infty} \frac{t^{i}}{i} \\
  &=(\log_2 \mathrm{e})\times t \times \left\{ - \ln (1-t) \right\}.
\end{split}\end{align}
To obtain the last equality in the above equations, we have used the Taylor series for $|t| < 1$.
Dividing both sides of Eq. (\ref{eq:(1-A)S}) by $1-t (\neq 0)$, we arrive at the following result
\begin{align}
  \overline{S} &= -\frac{1}{\ln 2} \times \frac{t}{1-t} \times \ln (1-t).
\end{align}
The lemma is obtained if we substitute $t$ into $1-z$.
\end{proof}
%-----------------------------------------------------------------------------------------------%
%-----------------------------------------------------------------------------------------------%
%-----------------------------------------------------------------------------------------------%
\par
In the next place, we calculate the value given in Eq. (\ref{eq:Sk}) more in details.
\subsection{Case of $1 \leq j \leq k-1$}
For $1 \leq j \leq k-1$, Eq. (\ref{eq:Sk}) is written as
\begin{align}\begin{split}
  \overline{S}_{k} 
  &:= \sum_{i=1}^{\infty} \sum_{j=1}^{k-1} g(i) g(j) \mathrm{Pr} \left[ A_n=i,\,A_{n+k}=j \right] \\
  &= \sum_{i=1}^{\infty} \sum_{j=1}^{k-1} g(i) g(j) \left( \sum_{r=0}^{L} \binom{L}{r}w_r^2 (1-w_r)^{i-1} \right) \times \left( \sum_{r=0}^{L} \binom{L}{r}w_r^2 (1-w_r)^{j-1} \right) \\ 
  &= \sum_{r=0}^{L} \left\{\binom{L}{r} \frac{w_r^2}{1-w_r} \sum_{i=1}^{\infty} g(i)(1-w_r)^{i} \right\}
  \times \sum_{r=0}^{L} \left\{ \binom{L}{r} \frac{w_r^2}{1-w_r} \sum_{j=1}^{k-1} g(j)(1-w_r)^{j} \right\}.
\end{split}\end{align}
From Lemma \ref{lemma:1}, the infinite series of $i$ in the above equations is written as
\begin{align}
	\sum_{i=1}^{\infty} g(i) (1 - w_r)^{i} = - \frac{1}{\ln 2} \times \frac{1-w_r}{w_r}\times\ln w_r.
\end{align}
%
Therefore, we obtain the following relation
\begin{align}\begin{split}
  \overline{S}_{k}
  =& \sum_{r=0}^{L} \left\{ \dbinom{L}{r} \frac{w_r^2}{1-w_r}  \times \left(- \frac{1}{\ln 2} \times \frac{1-w_r}{w_r}\times\ln w_r \right) \right\} \\
  &\times \sum_{r=0}^{L} \left\{ \dbinom{L}{r} \frac{w_r^2}{1-w_r} \times \sum_{j=1}^{k-1} (1-w_r)^{j} \right\} \\
  =&-\frac{1}{\ln 2}\sum_{r=0}^{L} \left\{ \dbinom{L}{r} w_r \ln w_r \right\} \times \sum_{r=0}^{L} \left\{ \dbinom{L}{r} \frac{w_r^2}{1-w_r} \sum_{j=1}^{k-1} (1-w_r)^{j} \right\}.
\end{split}\end{align}
%
\subsection{Case of $j=k$}
In the case of $j=k$, Eq. (\ref{eq:Sk}) is written as
\begin{align}\begin{split}\label{eq:app_case2}
  \overline{S}_{k} 
  &= \sum_{i=1}^{\infty} g(i) \sum_{r=0}^{L} \binom{L}{r} w_r^3 (1-w_r)^{k+i-2} \sum_{j \in \{k\}} g(j)\\
  &= \sum_{r=0}^{L} \binom{L}{r} w_r^3 (1-w_r)^{k-2} \sum_{i=1}^{\infty} g(i) (1-w_r)^{i} \times g(k) \\
  &= g(k) \times \sum_{r=0}^{L} \binom{L}{r} w_r^3 (1-w_r)^{k-2} \left( -\frac{1}{\ln 2} \times \frac{1-w_{r}}{w_{r}} \times \ln w_{r} \right) \\
  &= -\frac{g(k)}{\ln 2} \times \sum_{r=0}^{L} \binom{L}{r} w_r^2 (1-w_r)^{k-1} \ln w_{r}.
\end{split}\end{align}
The third equation in Eq. (\ref{eq:app_case2}) has been obtained from Lemma \ref{lemma:1}.
%-----------------------------------------------------------------------------------------------%
%-----------------------------------------------------------------------------------------------%
%-----------------------------------------------------------------------------------------------%
\subsection{Case of $k+1 \leq j \leq k+i-1$}
Recall that the joint distribution for $k+1 \leq j \leq k+i-1$ is written as
\begin{align}\begin{split}
	\mathrm{Pr}[A_n=i,\, A_{n+k}=j] 
	=& \sum_{r_1=0}^{L} \sum_{r_2 \neq r_1} \binom{L}{r_1}\binom{L}{r_2}\mathrm{Pr}[e_3(b_1,b_2)] \\
	&+ \sum_{r_1=0}^{L} \sum_{r_2 \in \{r_1\}} \binom{L}{r_1}\left\{\binom{L}{r_1}-1\right\}\mathrm{Pr}[e_3(b_1,b_2)],
\end{split}\end{align}
where $\mathrm{Pr}[e_3(b_1,b_2)]$ is expressed as
\begin{align}\begin{split}
	\mathrm{Pr}[e_3(b_1,b_2)]
  =& w_{r_1}^2  \times w_{r_2}^2 
  \times (1-w_{r_1})^{i-j+k-1} 
  \times (1-w_{r_1}-w_{r_2})^{j-k-1}
  \times (1-w_{r_2})^{k-1} \\
  =&\phi_k(r_1,r_2)\times (1-w_{r_1})^{i} \times \left(1-\frac{w_{r_2}}{1-w_{r_1}} \right)^{j},
\end{split}\end{align}
%
where
\begin{align}\label{eq:phi_k}
	\phi_k(r_1,r_2) = w_{r_1}^2 w_{r_2}^2 
  	(1-w_{r_1})^{k-1} 
  	(1-w_{r_1}-w_{r_2})^{-k-1}
  	(1-w_{r_2})^{k-1}.
\end{align}
%
Then, Eq. (\ref{eq:Sk}) is expressed as
\begin{align}\begin{split}\label{eq:Sk_case3}
	\overline{S}_k 
	=& \sum_{i=1}^{\infty}\sum_{j=k+1}^{k+i-1} g(i)g(j) \sum_{r_1=0}^{L} \sum_{r_2 \neq r_1} \binom{L}{r_1}\binom{L}{r_2}\mathrm{Pr}[e_3(b_1,b_2)]\\ 
	&+ \sum_{i=1}^{\infty}\sum_{j=k+1}^{k+i-1} g(i)g(j) \sum_{r_1=0}^{L} \sum_{r_2 \in \{r_1\}} \binom{L}{r_1} \left\{\binom{L}{r_1}-1 \right\}\mathrm{Pr}[e_3(b_1,b_2)].
\end{split}\end{align}
%
Now, we consider the first term in r.h.s. of Eq. (\ref{eq:Sk_case3}). Let $\overline{A}_1$ be this term of Eq. (\ref{eq:Sk_case3}). We have
\begin{align}\begin{split}\label{eq:A_1}
	\overline{A}_1
	&=\sum_{i=1}^{\infty}\sum_{j=k+1}^{k+i-1} g(i)g(j) \sum_{r_1=0}^{L} \sum_{r_2 \neq r_1} \binom{L}{r_1}\binom{L}{r_2}\mathrm{Pr}[e_3(b_1,b_2)] \\
	&=\sum_{r_1=0}^{L} \sum_{r_2 \neq r_1} \binom{L}{r_1}\binom{L}{r_2}\phi_k(r_1,r_2)
	\sum_{i=1}^{\infty} \sum_{j=1}^{k+i-1} g(i)g(j)(1-w_{r_1})^{i} \times \left(1-\frac{w_{r_2}}{1-w_{r_1}} \right)^{j} \\
	&=\sum_{r_1=0}^{L} \sum_{r_2 \neq r_1} \binom{L}{r_1}\binom{L}{r_2}\phi_k(r_1,r_2)
	\sum_{j=k+1}^{\infty} \sum_{i=k+1}^{k+i-1} g(i)g(j)(1-w_{r_1})^{i} \times \left(1-\frac{w_{r_2}}{1-w_{r_1}} \right)^{j} \\
	&=\sum_{r_1=0}^{L} \sum_{r_2 \neq r_1} \binom{L}{r_1}\binom{L}{r_2}\phi_k(r_1,r_2)
	\sum_{j=k+1}^{\infty} \left\{ g(j) \left(1-\frac{w_{r_2}}{1-w_{r_1}} \right)^{j} \times \sum_{i=j-k+1}^{\infty} g(i)(1-w_{r_1})^{i} \right\}.
\end{split}\end{align}
In the course of the derivation of the above relations, the second equation has been obtained from Eq. (\ref{eq:phi_k}). The third equation has been obtained by exchanging the summation over $i$ and $j$.
%
Then, the infinite series with respect to $i$ in the last equation of Eq. (\ref{eq:A_1}) can be calculated as
\begin{align}\begin{split}
	\sum_{i=j-k+1}^{\infty} g(i)(1-w_{r_1})^{i} 
	&= \sum_{i=1}^{\infty} g(i)(1-w_{r_1})^{i} - \sum_{i=1}^{j-k} g(i)(1-w_{r_1})^{i} \\
	&= -\frac{1}{\ln 2} \times \frac{1-w_{r_1}}{w_{r_1}} \times \ln w_{r_1} - \sum_{i=1}^{j-k} g(i)(1-w_{r_1})^{i}.
\end{split}\end{align}
In the above relations, we have used the result of Lemma \ref{lemma:1}.
Hence, the first term in r.h.s. of Eq. (\ref{eq:Sk_case3}) can be expressed as
\begin{align}
	\overline{A}_1 
	=& \sum_{r_1=0}^{L} \sum_{r_2 \neq r_1} \binom{L}{r_1}\binom{L}{r_2}\phi_k(r_1,r_2)\\
	&\times\sum_{j=k+1}^{\infty} \left[ g(j) \left(1-\frac{w_{r_2}}{1-w_{r_1}} \right)^{j} \times \left\{ -\frac{1}{\ln 2} \times \frac{1-w_{r_1}}{w_{r_1}} \times \ln w_{r_1} - \sum_{i=1}^{j-k} g(i)(1-w_{r_1})^{i} \right\} \right].
\end{align}
We can derive the second term in r.h.s. of Eq. (\ref{eq:Sk_case3}) in the same way as the first term. Let $\overline{A}_2$ be the second term of Eq. (\ref{eq:Sk_case3}). Then, we have
\begin{align}\begin{split}
	\overline{A}_2 =& \sum_{r_1=0}^{L} \sum_{r_2 \in \{r_1\}} \binom{L}{r_1}\left\{\binom{L}{r_1}-1\right\} \phi(r_1,r_2) \\
	&\times\sum_{j=k+1}^{\infty} \left[ g(j) \left(1-\frac{w_{r_2}}{1-w_{r_1}} \right)^{j} \times \left\{ -\frac{1}{\ln 2} \times \frac{1-w_{r_1}}{w_{r_1}} \times \ln w_{r_1} - \sum_{i=1}^{j-k} g(i)(1-w_{r_1})^{i} \right\} \right].
\end{split}\end{align}
%
Therefore, Eq. (\ref{eq:Sk_case3}) can be written as
\begin{align}\begin{split}
	\overline{S}_k =& \overline{A}_1 + \overline{A}_2 \\
	=& \sum_{r_1=0}^{L} \sum_{r_2 \neq r_1} \binom{L}{r_1}\binom{L}{r_2}\phi_k(r_1,r_2)\\
	&\times\sum_{j=k+1}^{\infty} \left[ g(j) \left(1-\frac{w_{r_2}}{1-w_{r_1}} \right)^{j} \times \left\{ -\frac{1}{\ln 2} \times \frac{1-w_{r_1}}{w_{r_1}} \times \ln w_{r_1} - \sum_{i=1}^{j-k} g(i)(1-w_{r_1})^{i} \right\} \right] \\
	&+\sum_{r_1=0}^{L} \sum_{r_2 \in \{r_1\}} \binom{L}{r_1}\left\{\binom{L}{r_1}-1\right\} \phi(r_1,r_2) \\
	&\times\sum_{j=k+1}^{\infty} \left[ g(j) \left(1-\frac{w_{r_2}}{1-w_{r_1}} \right)^{j} \times \left\{ -\frac{1}{\ln 2} 
	\times \frac{1-w_{r_1}}{w_{r_1}} \times \ln w_{r_1} - \sum_{i=1}^{j-k} g(i)(1-w_{r_1})^{i} \right\} \right],
\end{split}\end{align}
where $\phi_k(r_1,r_2)$ is given in Eq. (\ref{eq:phi_k}).
%-----------------------------------------------------------------------------------------------%
%-----------------------------------------------------------------------------------------------%
%-----------------------------------------------------------------------------------------------%
\subsection{Case of $j=k+i$}
Equation (\ref{eq:Sk}) in the case of $j=k+i$ is equal to $0$ from Eq. (\ref{eq:joint_distribution}).
%-----------------------------------------------------------------------------------------------%
%-----------------------------------------------------------------------------------------------%
%-----------------------------------------------------------------------------------------------%
\subsection{Case of $j \geq k+i+1$}
%--------------------------------------------------------------------------------%
Recall that the joint distribution for $j \geq k+i+1$ is written as
\begin{align}\begin{split}
  \mathrm{Pr}[A_n=i,\, A_{n+k}=j] 
  =& \sum_{r_1=0}^{L} \sum_{r_2 \neq r_1} \binom{L}{r_1}\binom{L}{r_2}\mathrm{Pr}[e_5(b_1,b_2)] \\
  &+ \sum_{r_1=0}^{L} \sum_{r_2 \in \{r_1\}} \binom{L}{r_1}\left\{\binom{L}{r_1}-1\right\}\mathrm{Pr}[e_5(b_1,b_2)],
\end{split}\end{align}
where $\mathrm{Pr}[e_5(b_1,b_2)]$ is expressed as
\begin{align}\begin{split}
  \mathrm{Pr}[e_5(b_1,b_2)]
  =& w_{r_1}^2  \times w_{r_2}^2 
  \times (1-w_{r_1})^{-i+j-k-1} 
  \times (1-w_{r_1}-w_{r_2})^{i-1}
  \times (1-w_{r_1})^{k-1} \\
  =&\psi_k(r_1,r_2)\times \left(1-\frac{w_{r_2}}{1-w_{r_1}} \right)^{i} \times (1-w_{r_1})^j,
\end{split}\end{align}
%
where
\begin{align}\begin{split}
  \psi_k(r_1,r_2) 
  &= w_{r_1}^2 w_{r_2}^2 
    (1-w_{r_1})^{-k-1} 
    (1-w_{r_1}-w_{r_2})^{-1}
    (1-w_{r_1})^{k-1} \\
  &= w_{r_1}^2 w_{r_2}^2 (1-w_{r_1})^{-2} (1-w_{r_1}-w_{r_2})^{-1}.
  \label{eq:psi_k}
\end{split}\end{align}
%
Then, Eq. (\ref{eq:Sk}) is expressed as
\begin{align}\begin{split}\label{eq:Sk_case5}
  \overline{S}_k 
  =& \sum_{i=1}^{\infty}\sum_{j=k+i+1}^{\infty} g(i)g(j) \sum_{r_1=0}^{L} \sum_{r_2 \neq r_1} \binom{L}{r_1}\binom{L}{r_2}\mathrm{Pr}[e_5(b_1,b_2)]\\ 
  &+ \sum_{i=1}^{\infty}\sum_{j=k+i+1}^{\infty} g(i)g(j) \sum_{r_1=0}^{L} \sum_{r_2 \in \{r_1\}} \binom{L}{r_1} \left\{\binom{L}{r_1}-1 \right\}\mathrm{Pr}[e_5(b_1,b_2)].
\end{split}\end{align}
%
Firstly, we consider the first term in the r.h.s. of Eq. (\ref{eq:Sk_case3}). Let $\overline{B}_1$ be the first term of Eq. (\ref{eq:Sk_case3}). Then, $\overline{B}_1$ is written as
\begin{align}\begin{split}\label{eq:B_1}
  \overline{B}_1
  &=\sum_{i=1}^{\infty}\sum_{j=k+i+1}^{\infty} g(i)g(j) \sum_{r_1=0}^{L} \sum_{r_2 \neq r_1} \binom{L}{r_1}\binom{L}{r_2}\mathrm{Pr}[e_5(b_1,b_2)] \\
  &=\sum_{r_1=0}^{L} \sum_{r_2 \neq r_1} \binom{L}{r_1}\binom{L}{r_2}\psi_k(r_1,r_2)
  \sum_{i=1}^{\infty} \sum_{j=k+i+1}^{\infty} g(i)g(j) \left(1-\frac{w_{r_2}}{1-w_{r_1}} \right)^{i} \times (1-w_{r_1})^j \\
  &=\sum_{r_1=0}^{L} \sum_{r_2 \neq r_1} \binom{L}{r_1}\binom{L}{r_2}\psi_k(r_1,r_2)
  \sum_{i=1}^{\infty} g(i)\left(1-\frac{w_{r_2}}{1-w_{r_1}} \right)^{i} \times \sum_{j=k+i+1}^{\infty} g(j) (1-w_{r_1})^j.
\end{split}\end{align}
In the course of the derivation of the above relations, the second equation has been obtained from Eq. (\ref{eq:phi_k}). The third equation has been obtained by exchange of infinite series.
%
Then, the infinite sum with respect to $j$ in the last equation of Eq. (\ref{eq:B_1}) can be calculated as
\begin{align}\begin{split}
  \sum_{j=k+i+1}^{\infty} g(j) (1-w_{r_1})^j 
  &= \sum_{j=1}^{\infty} g(j)(1-w_{r_1})^{j} - \sum_{j=1}^{k+i} g(j)(1-w_{r_1})^{j} \\
  &= -\frac{1}{\ln 2} \times \frac{1-w_{r_1}}{w_{r_1}} \times \ln w_{r_1} - \sum_{j=1}^{k+i} g(j)(1-w_{r_1})^{j}.
\end{split}\end{align}
In the above relations, we use the result of Lemma \ref{lemma:1}.
Hence, the first term of Eq. (\ref{eq:Sk_case5}) can be expressed as
\begin{align}
  \overline{B}_1 
  =& \sum_{r_1=0}^{L} \sum_{r_2 \neq r_1} \binom{L}{r_1}\binom{L}{r_2}\psi_k(r_1,r_2)\\
  &\times\sum_{i=1}^{\infty} \left[ g(i) \left(1-\frac{w_{r_2}}{1-w_{r_1}} \right)^{i} \times \left\{ -\frac{1}{\ln 2} \times \frac{1-w_{r_1}}{w_{r_1}} \times \ln w_{r_1} - \sum_{j=1}^{k+i} g(j)(1-w_{r_1})^{j} \right\} \right].
\end{align}
We can derive the second term of Eq. (\ref{eq:Sk_case5}) in the same way as the first term. Let $\overline{B}_2$ be the second term of Eq. (\ref{eq:Sk_case5}). Then, we have
\begin{align}\begin{split}
  \overline{B}_2 =& \sum_{r_1=0}^{L} \sum_{r_2 \in \{r_1\}} \binom{L}{r_1}\left\{\binom{L}{r_1}-1\right\} \psi(r_1,r_2) \\
  &\times\sum_{i=1}^{\infty} \left[ g(i) \left(1-\frac{w_{r_2}}{1-w_{r_1}} \right)^{i} \times \left\{ -\frac{1}{\ln 2} \times \frac{1-w_{r_1}}{w_{r_1}} \times \ln w_{r_1} - \sum_{j=1}^{k+i} g(j)(1-w_{r_1})^{j} \right\} \right].
\end{split}\end{align}
%
Therefore, Eq. (\ref{eq:Sk_case5}) can be expressed as
\begin{align}\begin{split}
  \overline{S}_k =& \overline{B}_1 + \overline{B}_2 \\
  =& \sum_{r_1=0}^{L} \sum_{r_2 \neq r_1} \binom{L}{r_1}\binom{L}{r_2}\psi_k(r_1,r_2)\\
  &\times\sum_{i=1}^{\infty} \left[ g(i) \left(1-\frac{w_{r_2}}{1-w_{r_1}} \right)^{i} \times \left\{ -\frac{1}{\ln 2} \times \frac{1-w_{r_1}}{w_{r_1}} \times \ln w_{r_1} - \sum_{j=1}^{k+i} g(j)(1-w_{r_1})^{j} \right\} \right] \\
  &+\sum_{r_1=0}^{L} \sum_{r_2 \in \{r_1\}} \binom{L}{r_1}\left\{\binom{L}{r_1}-1\right\} \psi(r_1,r_2) \\
  &\times\sum_{i=1}^{\infty} \left[ g(i) \left(1-\frac{w_{r_2}}{1-w_{r_1}} \right)^{i} \times \left\{ -\frac{1}{\ln 2} \times \frac{1-w_{r_1}}{w_{r_1}} \times \ln w_{r_1} - \sum_{j=1}^{k+i} g(j)(1-w_{r_1})^{j} \right\} \right],
\end{split}\end{align}
where $\psi_k(r_1,r_2)$ is given in Eq. (\ref{eq:psi_k}).
%
%-- End of body -------------------------------------------------------------------
\fi
\ifoutputcover
\cleardoublepage
%-- Covers and abstract for submission --------------------------------------------
\makecover                      % Cover
\makespine[\numberofspines]     % Spine
\fi
\ifoutputabstractforsubmission
\makeabstractforsubmission      % Abstract for submission
\fi
\end{document}
%----------------------------------------------------------------------------------
%----------------------------------------------------------------------------------
%----------------------------------------------------------------------------------
%----------------------------------------------------------------------------------
%----------------------------------------------------------------------------------
%----------------------------------------------------------------------------------
%----------------------------------------------------------------------------------
%----------------------------------------------------------------------------------
%----------------------------------------------------------------------------------
%----------------------------------------------------------------------------------
%----------------------------------------------------------------------------------
%----------------------------------------------------------------------------------
%----------------------------------------------------------------------------------
%----------------------------------------------------------------------------------
%----------------------------------------------------------------------------------
%----------------------------------------------------------------------------------
%----------------------------------------------------------------------------------
%----------------------------------------------------------------------------------
%----------------------------------------------------------------------------------
%----------------------------------------------------------------------------------
%----------------------------------------------------------------------------------
%----------------------------------------------------------------------------------