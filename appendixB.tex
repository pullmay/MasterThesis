\newpage
\section{Exploration of the covariance given in Eq. (\ref{eq:covariance_g_g})}\label{appendix:B}
In this appendix, we explore the covariance given in Eq. (\ref{eq:covariance_g_g}) to calculate the value more efficiently by computational experiments. Notice that the covariance is written as 
\begin{align}\label{eq:cov_saikei}
	\mathrm{Cov}[g(A_n),\, g(A_{n+k})] 
	= \sum_{i=1}^{\infty} \sum_{j=1}^{\infty} g(i) g(j) \mathrm{Pr} \left[ A_n=i,\,A_{n+k}=j \right] - \{L\times H(\hat{q})\}^2,
\end{align}
where $g$ is given in Eq. (\ref{eq:function_g}), $\mathrm{Pr} \left[ A_n=i,\,A_{n+k}=j \right]$ is given in Eq. (\ref{eq:joint_distribution}), and $H$ is a binary entropy function.
%
Since the term $\{L\times H(\hat{q})\}^2$ is irrelevant to an infinite series, we write the first term of the right hand side (r.h.s.) of Eq. (\ref{eq:cov_saikei}) as
%
% Let us denote the following quantity as $\mathcal{P}_k$
%
\begin{align}\label{eq:Sk}
	\overline{S}_{k} := \sum_{i=1}^{\infty} \sum_{j=1}^{\infty} g(i) g(j) \mathrm{Pr} \left[ A_n=i,\,A_{n+k}=j \right].
\end{align}
% From the result discussed in Section \ref{sec:distribution}, the above value depends on the joint distribution.
%
Firstly, we show the following Lemma concerning an infinite series to calculate the value expressed Eq. (\ref{eq:Sk}). 
%
\begin{lemma}\label{lemma:1}
% abcdefghijklmnopqrstuvwxyz
For any $z \in (0,1)$, the following relation holds
\begin{align}\label{eq:infinite_series}
	\sum_{i=1}^{\infty} g(i) \times (1-z)^i = -\frac{1}{\ln 2} \times \frac{1-z}{z} \times \ln z,
\end{align}
where
\begin{align}
	g(m) = (\log_2 \mathrm{e}) \sum_{k=1}^{m-1} \frac{1}{k}.
\end{align}
\end{lemma}
%
% \begin{proof}
% Let $A=1-z$, and $S$ be the left hand side of Eq. (\ref{eq:infinite_series}), that is,
% \begin{align}\begin{split}\label{eq:S}
% 	S &= \sum_{i=1}^{\infty} g(i) \times A^i \\
%     % \Leftrightarrow S &= g(1)\times A +  \sum_{i=2}^{\infty} g(i) \times A^{i} \label{eq:S}.
%     &= g(1)\times A +  \sum_{i=2}^{\infty} g(i) \times A^{i}.
% \end{split}\end{align}
% Multiplying the both sides of Eq. (\ref{eq:S}) by $A(\neq 0)$, we obtain the following relation
% \begin{align}\begin{split}\label{eq:AS}
% 	A\times S &= \sum_{i=1}^{\infty} g(i) \times A^{i+1} \\
% 	% \Leftrightarrow A\times S &= \sum_{i=2}^{\infty} g(i-1) \times A^{i} \label{eq:AS}.
%   &= \sum_{i=2}^{\infty} g(i-1) \times A^{i}.
% \end{split}\end{align}
% Subtracting Eq. (\ref{eq:AS}) from Eq. (\ref{eq:S}), we obtain the following relation
% \begin{align}\begin{split}\label{eq:(1-A)S}
% 	(1-A)S &= g(1)\times A^{1} + \sum_{i=2}^{\infty} \left\{ g(i)-g(i-1) \right\} \times A^{i} \\
% 	&=0 + \sum_{i=2}^{\infty} (\log_2 \mathrm{e}) \times \frac{1}{i-1} \times A^{i} \\
% 	&=(\log_2 \mathrm{e})\times A \times \sum_{i=1}^{\infty} \frac{A^{i}}{i} \\
% 	&=(\log_2 \mathrm{e})\times A \times \left\{ - \ln (1-A) \right\}.
% \end{split}\end{align}
% To obtain the last equality in the above equations, we have used the Taylor series for $|A| < 1$.
% % In the above relations, the last equality refers to the result of Taylor series for $|A| < 1$.
% Dividing both sides of Eq. (\ref{eq:(1-A)S}) by $1-A (\neq 0)$, we arrive at the following result
% % the following relations are obtained
% \begin{align}
% 	S &= -\frac{1}{\ln 2} \times \frac{A}{1-A} \times \ln (1-A).
% 	% &= -\frac{1}{\ln 2} \times \frac{1-z}{z} \times \ln z.
% \end{align}
% The lemma is obtained if we substitute $A$ into $1-z$.
% % In the above relations, the last equality is obtained from the fact $A=1-z$.
% \end{proof}
%-----------------------------------------------------------------------------------------------%
\begin{proof}
Let $t=1-z$, and $\overline{S}$ be the left hand side of Eq. (\ref{eq:infinite_series}), that is,
\begin{align}\begin{split}\label{eq:S}
  \overline{S} &= \sum_{i=1}^{\infty} g(i) \times t^i \\
    % \Leftrightarrow S &= g(1)\times A +  \sum_{i=2}^{\infty} g(i) \times A^{i} \label{eq:S}.
    &= g(1)\times t +  \sum_{i=2}^{\infty} g(i) \times t^{i}.
\end{split}\end{align}
Multiplying the both sides of Eq. (\ref{eq:S}) by $t(\neq 0)$, we obtain the following relation
\begin{align}\begin{split}\label{eq:AS}
  t\times \overline{S} &= \sum_{i=1}^{\infty} g(i) \times t^{i+1} \\
  % \Leftrightarrow A\times S &= \sum_{i=2}^{\infty} g(i-1) \times A^{i} \label{eq:AS}.
  &= \sum_{i=2}^{\infty} g(i-1) \times t^{i}.
\end{split}\end{align}
Subtracting Eq. (\ref{eq:AS}) from Eq. (\ref{eq:S}), we obtain the following relation
\begin{align}\begin{split}\label{eq:(1-A)S}
  (1-t)\overline{S} &= g(1)\times t^{1} + \sum_{i=2}^{\infty} \left\{ g(i)-g(i-1) \right\} \times t^{i} \\
  &=0 + \sum_{i=2}^{\infty} (\log_2 \mathrm{e}) \times \frac{1}{i-1} \times t^{i} \\
  &=(\log_2 \mathrm{e})\times t \times \sum_{i=1}^{\infty} \frac{t^{i}}{i} \\
  &=(\log_2 \mathrm{e})\times t \times \left\{ - \ln (1-t) \right\}.
\end{split}\end{align}
To obtain the last equality in the above equations, we have used the Taylor series for $|t| < 1$.
% In the above relations, the last equality refers to the result of Taylor series for $|A| < 1$.
Dividing both sides of Eq. (\ref{eq:(1-A)S}) by $1-t (\neq 0)$, we arrive at the following result
% the following relations are obtained
\begin{align}
  \overline{S} &= -\frac{1}{\ln 2} \times \frac{t}{1-t} \times \ln (1-t).
  % &= -\frac{1}{\ln 2} \times \frac{1-z}{z} \times \ln z.
\end{align}
The lemma is obtained if we substitute $t$ into $1-z$.
% In the above relations, the last equality is obtained from the fact $A=1-z$.
\end{proof}
%-----------------------------------------------------------------------------------------------%
%-----------------------------------------------------------------------------------------------%
%-----------------------------------------------------------------------------------------------%
% In the case of $1 \leq j \leq k-1$, $\mathcal{P}_k$ is calculated as
% \begin{align}
% 	\mathcal{P}_{k,\, j\in J_1} = -\frac{1}{\ln 2}\sum_{r=0}^{L} \left\{ \dbinom{L}{r} \mathcal{Q}_r \ln \mathcal{Q}_r \right\} \times \sum_{r=0}^{L} \left\{ \dbinom{L}{r} \frac{\mathcal{Q}_r^2}{1-\mathcal{Q}_r} \sum_{j=1}^{k-1} (1-\mathcal{Q}_r)^{j} \right\}.
% \end{align}
% In the case of $j=k$,
% \begin{align}
% 	\mathcal{P}_{k,\, j\in J_2} = -\frac{g(k)}{\ln 2}\sum_{r=0}^{L} \left\{ \dbinom{L}{r} \mathcal{Q}_r^2 (1-\mathcal{Q})^{k-1} \ln \mathcal{Q}_r \right\}.
% \end{align}
% In the case of $k+1 \leq j \leq k+i-1$,
% \begin{align}
% 	\mathcal{P}_{k,\, j\in J_3} 
% 	=& \sum_{r_1=0}^{L} \sum_{r_2 \neq r_1} \binom{L}{r_1}\binom{L}{r_2} \phi_k(r_1,r_2) \\
% 	&\times \sum_{j=k+1}^{\infty} \left[ g(j)\left( 1-\frac{\mathcal{Q}_{r_2}}{1-\mathcal{Q}_{r_1}} \right)^{j} \left\{ -\frac{1}{\ln 2} \times \frac{1-\mathcal{Q}_{r_1}}{\mathcal{Q}_{r_1}} \times \ln \mathcal{Q}_{r_1} - \sum_{i=1}^{j-k} g(i)(1-\mathcal{Q})^i \right\} \right] \\
% 	&\times \sum_{r_1=0}^{L} \sum_{r_2 \in \{r_1\}} \binom{L}{r_1}\left\{\binom{L}{r_1}-1\right\} \phi_k(r_1,r_2) \\
% 	&\times \sum_{j=k+1}^{\infty} \left[ g(j)\left( 1-\frac{\mathcal{Q}_{r_2}}{1-\mathcal{Q}_{r_1}} \right)^{j} \left\{ -\frac{1}{\ln 2} \times \frac{1-\mathcal{Q}_{r_1}}{\mathcal{Q}_{r_1}} \times \ln \mathcal{Q}_{r_1} - \sum_{i=1}^{j-k} g(i)(1-\mathcal{Q})^i \right\} \right].
% \end{align}
% where $\phi_k$ is defined as follows:
% \begin{align}
% 	\phi_k(r_1,r_2) = \mathcal{Q}_{r_1}^2 \mathcal{Q}_{r_2}^2 (1-\mathcal{Q}_{r_1})^{k-1} (1-\mathcal{Q}_{r_1}-\mathcal{Q}_{r_2})^{-k-1} (1-\mathcal{Q}_{r_2})^{k-1}.
% \end{align}
% %
% In the case of $j=k+i$,
% \begin{align}
% 	\mathcal{P}_{k,\, j\in J_4} = 0
% \end{align}
% %
% In the case of $j\geq k+i+1$,
% \begin{align}
% 	\mathcal{P}_{k,\, j\in J_5} 
% 	=& \sum_{r_1=0}^{L} \sum_{r_2 \neq r_1} \binom{L}{r_1}\binom{L}{r_2} \psi(r_1,r_2) \\
% 	&\times \left[ \frac{1}{(\ln 2)^2} \times \mu(r_1,r_2) - \sum_{i=1}^{\infty}\left\{ g(i)\left( 1-\frac{\mathcal{Q}_{r_2}}{1-\mathcal{Q}_{r_1}} \right)^{i} \times \sum_{j=1}^{k+i} g(j) (1-\mathcal{Q}_{r_1})^{j} \right\} \right] \\
% 	&\times \sum_{r_1=0}^{L} \sum_{r_2 \in \{r_1\}} \binom{L}{r_1}\left\{\binom{L}{r_1}-1\right\} \psi(r_1,r_2) \\
% 	&\times \left[ \frac{1}{(\ln 2)^2} \times \mu(r_1,r_2) - \sum_{i=1}^{\infty}\left\{ g(i)\left( 1-\frac{\mathcal{Q}_{r_2}}{1-\mathcal{Q}_{r_1}} \right)^{i} \times \sum_{j=1}^{k+i} g(j) (1-\mathcal{Q}_{r_1})^{j} \right\} \right].
% \end{align}
% where $\psi$ and $\mu$ are defined as follows:
% \begin{align}
% 	\psi(r_1,r_2) &= \mathcal{Q}_{r_1}^2 \mathcal{Q}_{r_2}^2 (1-\mathcal{Q}_{r_1})^{-2} (1-\mathcal{Q}_{r_1}-\mathcal{Q}_{r_2})^{-1} \\
% 	\mu(r_1,r_2) &= \frac{1-\mathcal{Q}_{r_1}}{\mathcal{Q}_{r_1}} \times \frac{1-\mathcal{Q}_{r_1}-\mathcal{Q}_{r_2}}{\mathcal{Q}_{r_1}} \times \ln \mathcal{Q}_{r_1} \times \ln \frac{1-\mathcal{Q}_{r_2}}{\mathcal{Q}_{r_1}}.
% \end{align}
%-----------------------------------------------------------------------------------------------%
%-----------------------------------------------------------------------------------------------%
%-----------------------------------------------------------------------------------------------%
\par
In the next place, we calculate the value given in Eq. (\ref{eq:Sk}) more in details.
\subsection{Case of $1 \leq j \leq k-1$}
For $1 \leq j \leq k-1$, Eq. (\ref{eq:Sk}) is written as
\begin{align}\begin{split}
  % \mathcal{P}_{k,1} 
  \overline{S}_{k} 
  &:= \sum_{i=1}^{\infty} \sum_{j=1}^{k-1} g(i) g(j) \mathrm{Pr} \left[ A_n=i,\,A_{n+k}=j \right] \\
  &= \sum_{i=1}^{\infty} \sum_{j=1}^{k-1} g(i) g(j) \left( \sum_{r=0}^{L} \binom{L}{r}w_r^2 (1-w_r)^{i-1} \right) \times \left( \sum_{r=0}^{L} \binom{L}{r}w_r^2 (1-w_r)^{j-1} \right) \\
  % &= \sum_{r=0}^{L} \left\{\binom{L}{r} \mathcal{Q}_r^2 (1-\mathcal{Q}_r)^{-1} \sum_{i=1}^{\infty} g(i)(1-\mathcal{Q}_r)^{i} \right\}
  % \times \sum_{r=0}^{L} \left\{ \binom{L}{r} \mathcal{Q}_r^2 (1-\mathcal{Q}_r)^{-1} \sum_{j=1}^{k-1} g(j)(1-\mathcal{Q}_r)^{j} \right\} 
  &= \sum_{r=0}^{L} \left\{\binom{L}{r} \frac{w_r^2}{1-w_r} \sum_{i=1}^{\infty} g(i)(1-w_r)^{i} \right\}
  \times \sum_{r=0}^{L} \left\{ \binom{L}{r} \frac{w_r^2}{1-w_r} \sum_{j=1}^{k-1} g(j)(1-w_r)^{j} \right\}.
\end{split}\end{align}
From Lemma \ref{lemma:1}, the infinite series of $i$ in the above equations is written as
\begin{align}
	% \sum_{i=1}^{\infty} g(i) \times (1 - \mathcal{Q}_r)^{i} = - \frac{1}{\ln 2} \times \frac{1-\mathcal{Q}_r}{\mathcal{Q}_r}\times\ln \mathcal{Q}_r.
	\sum_{i=1}^{\infty} g(i) (1 - w_r)^{i} = - \frac{1}{\ln 2} \times \frac{1-w_r}{w_r}\times\ln w_r.
\end{align}
%
Therefore, we obtain the following relation
\begin{align}\begin{split}
  % \mathcal{P}_{k,1}
  \overline{S}_{k}
  % =& \sum_{r=0}^{L} \left\{ \dbinom{L}{r} \mathcal{Q}_r^2 \left\{ 1-\mathcal{Q}_r \right\}^{-1} \times \left(- \frac{1}{\ln 2} \times \frac{1-\mathcal{Q}_r}{\mathcal{Q}_r}\times\ln \mathcal{Q}_r \right) \right\} \\
  % &\times \sum_{r=0}^{L} \left\{ \dbinom{L}{r} \mathcal{Q}_r^2 \left\{ 1-\mathcal{Q}_r \right\}^{-1} \times \sum_{j=1}^{k-1} (1-\mathcal{Q}_r)^{j} \right\} \\
  % =&-\frac{1}{\ln 2}\sum_{r=0}^{L} \left\{ \dbinom{L}{r} \mathcal{Q}_r \ln \mathcal{Q}_r \right\} \times \sum_{r=0}^{L} \left\{ \dbinom{L}{r} \frac{\mathcal{Q}_r^2}{1-\mathcal{Q}_r} \sum_{j=1}^{k-1} (1-\mathcal{Q}_r)^{j} \right\}
  =& \sum_{r=0}^{L} \left\{ \dbinom{L}{r} \frac{w_r^2}{1-w_r}  \times \left(- \frac{1}{\ln 2} \times \frac{1-w_r}{w_r}\times\ln w_r \right) \right\} \\
  &\times \sum_{r=0}^{L} \left\{ \dbinom{L}{r} \frac{w_r^2}{1-w_r} \times \sum_{j=1}^{k-1} (1-w_r)^{j} \right\} \\
  =&-\frac{1}{\ln 2}\sum_{r=0}^{L} \left\{ \dbinom{L}{r} w_r \ln w_r \right\} \times \sum_{r=0}^{L} \left\{ \dbinom{L}{r} \frac{w_r^2}{1-w_r} \sum_{j=1}^{k-1} (1-w_r)^{j} \right\}.
\end{split}\end{align}
%
% Let S be an infinite series of $i$ denoted as:
% \begin{align}
% 	S = \sum_{i=1}^{\infty} g(i) (1-\mathcal{Q}_r)^i
% \end{align}
% %
% Letting $A = 1-\mathcal{Q}_r$, the above equation is denoted as: 
% \begin{align}
%   \label{eq:S}
%   S = \sum_{i=1}^{\infty} g(i) A^i = g(1)A^1 + \sum_{i=2}^{\infty} g(i) A^i
% \end{align}
% By multiplying the both sides of the equation (\ref{eq:S}) by $A(\neq 0)$, we obtain:
% \begin{align}
%   A\times S = \sum_{i=1}^{\infty} g(i) A^{i+1} = \sum_{i=2}^{\infty} g(i-1) A^i
% \end{align}
% By subtracting equation () from equation (), we obtain:
% \begin{align}
%   (1-A)S &= g(1) A^1 + \sum_{i=2}^{\infty} \left\{g(i) - g(i-1)\right\} A^i \\
%   &= 0 + \sum_{i=2}^{\infty} (\log_2 \mathrm{e}) \times \frac{1}{i-1} A^i \\
%   &= A \times (\log_2 \mathrm{e}) \sum_{i=1}^{\infty} \frac{A^i}{i} \\
%   &= A \times (\log_2 \mathrm{e}) \times \left\{ -\ln (1-A) \right\}
% \end{align}
% By dividing the both sides of the above equation by $1-A$, we obtain:
% \begin{align}
%   S &= \frac{-A}{1-A} \times (\log_2\mathrm{e}) \times \ln (1-A) \\
%   &= \frac{-A}{1-A} \times \frac{\ln (1-A)}{\ln 2} \\
%   &= -\frac{1-\mathcal{Q}_r}{\mathcal{Q}_r} \times \frac{\ln \mathcal{Q}_r}{\ln 2} \quad (\because A=1-\mathcal{Q}_r)\\
%   &= - \frac{1}{\ln 2} \times \frac{1-\mathcal{Q}_r}{\mathcal{Q}_r}\times\ln \mathcal{Q}_r 
% \end{align}
% Therefore, we obtain:
% \begin{align}
%   \sum_{i=1}^{\infty} g(i) (1 - \mathcal{Q}_r)^{i} = - \frac{1}{\ln 2} \times \frac{1-\mathcal{Q}_r}{\mathcal{Q}_r}\times\ln \mathcal{Q}_r
% \end{align}
% %
% Therefore, we obtain:
% \begin{align}
%   \mathcal{P}_{k,1}
%   =& \sum_{r=0}^{L} \left\{ \dbinom{L}{r} \mathcal{Q}_r^2 \left\{ 1-\mathcal{Q}_r \right\}^{-1} \times \left(- \frac{1}{\ln 2} \times \frac{1-\mathcal{Q}_r}{\mathcal{Q}_r}\times\ln \mathcal{Q}_r \right) \right\} \\
%   &\times \sum_{r=0}^{L} \left\{ \dbinom{L}{r} \mathcal{Q}_r^2 \left\{ 1-\mathcal{Q}_r \right\}^{-1} \times \sum_{j=1}^{k-1} (1-\mathcal{Q}_r)^{j} \right\} \\
%   =&-\frac{1}{\ln 2}\sum_{r=0}^{L} \left\{ \dbinom{L}{r} \mathcal{Q}_r \ln \mathcal{Q}_r \right\} \times \sum_{r=0}^{L} \left\{ \dbinom{L}{r} \frac{\mathcal{Q}_r^2}{1-\mathcal{Q}_r} \sum_{j=1}^{k-1} (1-\mathcal{Q}_r)^{j} \right\}
% \end{align}
\subsection{Case of $j=k$}
% Let $J_2=\{j\mid j=k\}$. 
In the case of $j=k$, Eq. (\ref{eq:Sk}) is written as
\begin{align}\begin{split}\label{eq:app_case2}
  \overline{S}_{k} 
  % &= \sum_{i=1}^{\infty} g(i) \sum_{r=0}^{L} \binom{L}{r} \mathcal{Q}_r^3 (1-\mathcal{Q}_r)^{k+i-2} \sum_{j \in J_2} g(j)\\
  % &= \sum_{r=0}^{L} \binom{L}{r} \mathcal{Q}_r^3 (1-\mathcal{Q}_r)^{k-2} \sum_{i=1}^{\infty} g(i) (1-\mathcal{Q}_r)^{i} \times g(k) \\
  % &= g(k) \times \sum_{r=0}^{L} \binom{L}{r} \mathcal{Q}_r^3 (1-\mathcal{Q}_r)^{k-2} \left( -\frac{1}{\ln 2} \times \frac{1-\mathcal{Q}_{r}}{\mathcal{Q}_{r}} \times \ln \mathcal{Q}_{r} \right) \\
  % &= -\frac{g(k)}{\ln 2} \times \sum_{r=0}^{L} \binom{L}{r} \mathcal{Q}_r^2 (1-\mathcal{Q}_r)^{k-1} \ln \mathcal{Q}_{r}.
  &= \sum_{i=1}^{\infty} g(i) \sum_{r=0}^{L} \binom{L}{r} w_r^3 (1-w_r)^{k+i-2} \sum_{j \in \{k\}} g(j)\\
  &= \sum_{r=0}^{L} \binom{L}{r} w_r^3 (1-w_r)^{k-2} \sum_{i=1}^{\infty} g(i) (1-w_r)^{i} \times g(k) \\
  &= g(k) \times \sum_{r=0}^{L} \binom{L}{r} w_r^3 (1-w_r)^{k-2} \left( -\frac{1}{\ln 2} \times \frac{1-w_{r}}{w_{r}} \times \ln w_{r} \right) \\
  &= -\frac{g(k)}{\ln 2} \times \sum_{r=0}^{L} \binom{L}{r} w_r^2 (1-w_r)^{k-1} \ln w_{r}.
\end{split}\end{align}
The third equation in Eq. (\ref{eq:app_case2}) has been obtained from Lemma \ref{lemma:1}.
%-----------------------------------------------------------------------------------------------%
%-----------------------------------------------------------------------------------------------%
%-----------------------------------------------------------------------------------------------%
\subsection{Case of $k+1 \leq j \leq k+i-1$}
Recall that the joint distribution for $k+1 \leq j \leq k+i-1$ is written as
% When $k+1 \leq j \leq k+i-1$, the joint distribution is provided as
\begin{align}\begin{split}
	\mathrm{Pr}[A_n=i,\, A_{n+k}=j] 
	=& \sum_{r_1=0}^{L} \sum_{r_2 \neq r_1} \binom{L}{r_1}\binom{L}{r_2}\mathrm{Pr}[e_3(b_1,b_2)] \\
	&+ \sum_{r_1=0}^{L} \sum_{r_2 \in \{r_1\}} \binom{L}{r_1}\left\{\binom{L}{r_1}-1\right\}\mathrm{Pr}[e_3(b_1,b_2)],
\end{split}\end{align}
where $\mathrm{Pr}[e_3(b_1,b_2)]$ is expressed as
\begin{align}\begin{split}
	\mathrm{Pr}[e_3(b_1,b_2)]
	% =& \mathcal{Q}_{r_1}^2  \times \mathcal{Q}_{r_2}^2 
 %  \times (1-\mathcal{Q}_{r_1})^{i-j+k-1} 
 %  \times (1-\mathcal{Q}_{r_1}-\mathcal{Q}_{r_2})^{j-k-1}
 %  \times (1-\mathcal{Q}_{r_2})^{k-1} \\
 %  =&\mathcal{Q}_{r_1}^2  \times \mathcal{Q}_{r_2}^2 
 %  \times (1-\mathcal{Q}_{r_1})^{k-1} 
 %  \times (1-\mathcal{Q}_{r_1}-\mathcal{Q}_{r_2})^{-k-1}
 %  \times (1-\mathcal{Q}_{r_2})^{k-1} \\
 %  &\times (1-\mathcal{Q}_{r_1})^{i} \times \left(1-\frac{\mathcal{Q}_{r_2}}{1-\mathcal{Q}_{r_1}} \right)^{j}
 	=& w_{r_1}^2  \times w_{r_2}^2 
  \times (1-w_{r_1})^{i-j+k-1} 
  \times (1-w_{r_1}-w_{r_2})^{j-k-1}
  \times (1-w_{r_2})^{k-1} \\
  % =&w_{r_1}^2  \times w_{r_2}^2 
  % \times (1-w_{r_1})^{k-1} 
  % \times (1-w_{r_1}-w_{r_2})^{-k-1}
  % \times (1-w_{r_2})^{k-1} \\
  =&\phi_k(r_1,r_2)\times (1-w_{r_1})^{i} \times \left(1-\frac{w_{r_2}}{1-w_{r_1}} \right)^{j},
\end{split}\end{align}
%
where
\begin{align}\label{eq:phi_k}
	\phi_k(r_1,r_2) = w_{r_1}^2 w_{r_2}^2 
  	(1-w_{r_1})^{k-1} 
  	(1-w_{r_1}-w_{r_2})^{-k-1}
  	(1-w_{r_2})^{k-1}.
\end{align}
%
Then, Eq. (\ref{eq:Sk}) is expressed as
\begin{align}\begin{split}\label{eq:Sk_case3}
	% \mathcal{P}_k 
	% =& \sum_{i=1}^{\infty}\sum_{j=k+1}^{k+i-1} g(i)g(j) \sum_{r_1=0}^{L} \sum_{r_2 \neq r_1} \binom{L}{r_1}\binom{L}{r_2}\mathrm{Pr}[e_3(b_1,b_2)]\\ 
	% &+ \sum_{i=1}^{\infty}\sum_{j=k+1}^{k+i-1} g(i)g(j) \sum_{r_1=0}^{L} \sum_{r_2 \in \{r_1\}} \binom{L}{r_1} \left\{\binom{L}{r_1}-1 \right\}\mathrm{Pr}[e_3(b_1,b_2)]
	\overline{S}_k 
	=& \sum_{i=1}^{\infty}\sum_{j=k+1}^{k+i-1} g(i)g(j) \sum_{r_1=0}^{L} \sum_{r_2 \neq r_1} \binom{L}{r_1}\binom{L}{r_2}\mathrm{Pr}[e_3(b_1,b_2)]\\ 
	&+ \sum_{i=1}^{\infty}\sum_{j=k+1}^{k+i-1} g(i)g(j) \sum_{r_1=0}^{L} \sum_{r_2 \in \{r_1\}} \binom{L}{r_1} \left\{\binom{L}{r_1}-1 \right\}\mathrm{Pr}[e_3(b_1,b_2)].
\end{split}\end{align}
%
Now, we consider the first term in r.h.s. of Eq. (\ref{eq:Sk_case3}). Let $\overline{A}_1$ be this term of Eq. (\ref{eq:Sk_case3}). We have
\begin{align}\begin{split}\label{eq:A_1}
	% \mathcal{S}_1
	% &=\sum_{i=1}^{\infty}\sum_{j=k+1}^{k+i-1} g(i)g(j) \sum_{r_1=0}^{L} \sum_{r_2 \neq r_1} \binom{L}{r_1}\binom{L}{r_2}\mathrm{Pr}[e_3(b_1,b_2)] \\
	% &=\sum_{r_1=0}^{L} \sum_{r_2 \neq r_1} \binom{L}{r_1}\binom{L}{r_2}\phi_k(r_1,r_2)
	% \sum_{i=1}^{\infty} \sum_{j=1}^{k+i-1} g(i)g(j)(1-\mathcal{Q}_{r_1})^{i} \times \left(1-\frac{\mathcal{Q}_{r_2}}{1-\mathcal{Q}_{r_1}} \right)^{j} \\
	% &=\sum_{r_1=0}^{L} \sum_{r_2 \neq r_1} \binom{L}{r_1}\binom{L}{r_2}\phi_k(r_1,r_2)
	% \sum_{j=k+1}^{\infty} \sum_{i=k+1}^{k+i-1} g(i)g(j)(1-\mathcal{Q}_{r_1})^{i} \times \left(1-\frac{\mathcal{Q}_{r_2}}{1-\mathcal{Q}_{r_1}} \right)^{j} \\
	% &=\sum_{r_1=0}^{L} \sum_{r_2 \neq r_1} \binom{L}{r_1}\binom{L}{r_2}\phi_k(r_1,r_2)
	% \sum_{j=k+1}^{\infty} \left\{ g(j) \left(1-\frac{\mathcal{Q}_{r_2}}{1-\mathcal{Q}_{r_1}} \right)^{j} \times \sum_{i=j-k+1}^{\infty} g(i)(1-\mathcal{Q}_{r_1})^{i} \right\} 
	\overline{A}_1
	&=\sum_{i=1}^{\infty}\sum_{j=k+1}^{k+i-1} g(i)g(j) \sum_{r_1=0}^{L} \sum_{r_2 \neq r_1} \binom{L}{r_1}\binom{L}{r_2}\mathrm{Pr}[e_3(b_1,b_2)] \\
	&=\sum_{r_1=0}^{L} \sum_{r_2 \neq r_1} \binom{L}{r_1}\binom{L}{r_2}\phi_k(r_1,r_2)
	\sum_{i=1}^{\infty} \sum_{j=1}^{k+i-1} g(i)g(j)(1-w_{r_1})^{i} \times \left(1-\frac{w_{r_2}}{1-w_{r_1}} \right)^{j} \\
	&=\sum_{r_1=0}^{L} \sum_{r_2 \neq r_1} \binom{L}{r_1}\binom{L}{r_2}\phi_k(r_1,r_2)
	\sum_{j=k+1}^{\infty} \sum_{i=k+1}^{k+i-1} g(i)g(j)(1-w_{r_1})^{i} \times \left(1-\frac{w_{r_2}}{1-w_{r_1}} \right)^{j} \\
	&=\sum_{r_1=0}^{L} \sum_{r_2 \neq r_1} \binom{L}{r_1}\binom{L}{r_2}\phi_k(r_1,r_2)
	\sum_{j=k+1}^{\infty} \left\{ g(j) \left(1-\frac{w_{r_2}}{1-w_{r_1}} \right)^{j} \times \sum_{i=j-k+1}^{\infty} g(i)(1-w_{r_1})^{i} \right\}.
\end{split}\end{align}
In the course of the derivation of the above relations, the second equation has been obtained from Eq. (\ref{eq:phi_k}). The third equation has been obtained by exchanging the summation over $i$ and $j$.
%
Then, the infinite series with respect to $i$ in the last equation of Eq. (\ref{eq:A_1}) can be calculated as
\begin{align}\begin{split}
	% \sum_{i=j-k+1}^{\infty} g(i)(1-\mathcal{Q}_{r_1})^{i} 
	% &= \sum_{i=1}^{\infty} g(i)(1-\mathcal{Q}_{r_1})^{i} - \sum_{i=1}^{j-k} g(i)(1-\mathcal{Q}_{r_1})^{i} \\
	% &= -\frac{1}{\ln 2} \times \frac{1-\mathcal{Q}_{r_1}}{\mathcal{Q}_{r_1}} \times \ln \mathcal{Q}_{r_1} - \sum_{i=1}^{j-k} g(i)(1-\mathcal{Q}_{r_1})^{i}.
	\sum_{i=j-k+1}^{\infty} g(i)(1-w_{r_1})^{i} 
	&= \sum_{i=1}^{\infty} g(i)(1-w_{r_1})^{i} - \sum_{i=1}^{j-k} g(i)(1-w_{r_1})^{i} \\
	&= -\frac{1}{\ln 2} \times \frac{1-w_{r_1}}{w_{r_1}} \times \ln w_{r_1} - \sum_{i=1}^{j-k} g(i)(1-w_{r_1})^{i}.
\end{split}\end{align}
In the above relations, we have used the result of Lemma \ref{lemma:1}.
Hence, the first term in r.h.s. of Eq. (\ref{eq:Sk_case3}) can be expressed as
\begin{align}
	% \mathcal{S}_1 
	% =& \sum_{r_1=0}^{L} \sum_{r_2 \neq r_1} \binom{L}{r_1}\binom{L}{r_2}\phi_k(r_1,r_2)\\
	% &\times\sum_{j=k+1}^{\infty} \left[ g(j) \left(1-\frac{\mathcal{Q}_{r_2}}{1-\mathcal{Q}_{r_1}} \right)^{j} \times \left\{ -\frac{1}{\ln 2} \times \frac{1-\mathcal{Q}_{r_1}}{\mathcal{Q}_{r_1}} \times \ln \mathcal{Q}_{r_1} - \sum_{i=1}^{j-k} g(i)(1-\mathcal{Q}_{r_1})^{i} \right\} \right].
	\overline{A}_1 
	=& \sum_{r_1=0}^{L} \sum_{r_2 \neq r_1} \binom{L}{r_1}\binom{L}{r_2}\phi_k(r_1,r_2)\\
	&\times\sum_{j=k+1}^{\infty} \left[ g(j) \left(1-\frac{w_{r_2}}{1-w_{r_1}} \right)^{j} \times \left\{ -\frac{1}{\ln 2} \times \frac{1-w_{r_1}}{w_{r_1}} \times \ln w_{r_1} - \sum_{i=1}^{j-k} g(i)(1-w_{r_1})^{i} \right\} \right].
\end{align}
We can derive the second term in r.h.s. of Eq. (\ref{eq:Sk_case3}) in the same way as the first term. Let $\overline{A}_2$ be the second term of Eq. (\ref{eq:Sk_case3}). Then, we have
\begin{align}\begin{split}
	\overline{A}_2 =& \sum_{r_1=0}^{L} \sum_{r_2 \in \{r_1\}} \binom{L}{r_1}\left\{\binom{L}{r_1}-1\right\} \phi(r_1,r_2) \\
	&\times\sum_{j=k+1}^{\infty} \left[ g(j) \left(1-\frac{w_{r_2}}{1-w_{r_1}} \right)^{j} \times \left\{ -\frac{1}{\ln 2} \times \frac{1-w_{r_1}}{w_{r_1}} \times \ln w_{r_1} - \sum_{i=1}^{j-k} g(i)(1-w_{r_1})^{i} \right\} \right].
\end{split}\end{align}
%
Therefore, Eq. (\ref{eq:Sk_case3}) can be written as
\begin{align}\begin{split}
	\overline{S}_k =& \overline{A}_1 + \overline{A}_2 \\
	=& \sum_{r_1=0}^{L} \sum_{r_2 \neq r_1} \binom{L}{r_1}\binom{L}{r_2}\phi_k(r_1,r_2)\\
	&\times\sum_{j=k+1}^{\infty} \left[ g(j) \left(1-\frac{w_{r_2}}{1-w_{r_1}} \right)^{j} \times \left\{ -\frac{1}{\ln 2} \times \frac{1-w_{r_1}}{w_{r_1}} \times \ln w_{r_1} - \sum_{i=1}^{j-k} g(i)(1-w_{r_1})^{i} \right\} \right] \\
	&+\sum_{r_1=0}^{L} \sum_{r_2 \in \{r_1\}} \binom{L}{r_1}\left\{\binom{L}{r_1}-1\right\} \phi(r_1,r_2) \\
	&\times\sum_{j=k+1}^{\infty} \left[ g(j) \left(1-\frac{w_{r_2}}{1-w_{r_1}} \right)^{j} \times \left\{ -\frac{1}{\ln 2} 
	\times \frac{1-w_{r_1}}{w_{r_1}} \times \ln w_{r_1} - \sum_{i=1}^{j-k} g(i)(1-w_{r_1})^{i} \right\} \right],
\end{split}\end{align}
where $\phi_k(r_1,r_2)$ is given in Eq. (\ref{eq:phi_k}).
% \begin{align}
% 	\phi_k(r_1,r_2) = w_{r_1}^2 w_{r_2}^2 (1-w_{r_1})^{k-1} (1-w_{r_1}-w_{r_2})^{-k-1} (1-w_{r_2})^{k-1}.
% \end{align}
%-----------------------------------------------------------------------------------------------%
%-----------------------------------------------------------------------------------------------%
%-----------------------------------------------------------------------------------------------%
\subsection{Case of $j=k+i$}
Equation (\ref{eq:Sk}) in the case of $j=k+i$ is equal to $0$ from Eq. (\ref{eq:joint_distribution}).
%-----------------------------------------------------------------------------------------------%
%-----------------------------------------------------------------------------------------------%
%-----------------------------------------------------------------------------------------------%
\subsection{Case of $j \geq k+i+1$}
% \begin{align}
% 	\mathcal{P}_{k,\, j\in J_5} {}
% 	=& \sum_{r_1=0}^{L} \sum_{r_2 \neq r_1} \binom{L}{r_1}\binom{L}{r_2} \psi(r_1,r_2) \\
% 	&\times \left[ \frac{1}{(\ln 2)^2} \times \mu(r_1,r_2) - \sum_{i=1}^{\infty}\left\{ g(i)\left( 1-\frac{w_{r_2}}{1-w_{r_1}} \right)^{i} \times \sum_{j=1}^{k+i} g(j) (1-w_{r_1})^{j} \right\} \right] \\
% 	&\times \sum_{r_1=0}^{L} \sum_{r_2 \in \{r_1\}} \binom{L}{r_1}\left\{\binom{L}{r_1}-1\right\} \psi(r_1,r_2) \\
% 	&\times \left[ \frac{1}{(\ln 2)^2} \times \mu(r_1,r_2) - \sum_{i=1}^{\infty}\left\{ g(i)\left( 1-\frac{w_{r_2}}{1-w_{r_1}} \right)^{i} \times \sum_{j=1}^{k+i} g(j) (1-w_{r_1})^{j} \right\} \right].
% \end{align}
% where $\psi$ and $\mu$ are defined as follows:
% \begin{align}
% 	\psi(r_1,r_2) &= w_{r_1}^2 w_{r_2}^2 (1-w_{r_1})^{-2} (1-w_{r_1}-w_{r_2})^{-1} \\
% 	\mu(r_1,r_2) &= \frac{1-w_{r_1}}{w_{r_1}} \times \frac{1-w_{r_1}-w_{r_2}}{w_{r_1}} \times \ln w_{r_1} \times \ln \frac{1-w_{r_2}}{w_{r_1}}.
% \end{align}
%--------------------------------------------------------------------------------%
Recall that the joint distribution for $j \geq k+i+1$ is written as
% When $k+1 \leq j \leq k+i-1$, the joint distribution is provided as
\begin{align}\begin{split}
  \mathrm{Pr}[A_n=i,\, A_{n+k}=j] 
  =& \sum_{r_1=0}^{L} \sum_{r_2 \neq r_1} \binom{L}{r_1}\binom{L}{r_2}\mathrm{Pr}[e_5(b_1,b_2)] \\
  &+ \sum_{r_1=0}^{L} \sum_{r_2 \in \{r_1\}} \binom{L}{r_1}\left\{\binom{L}{r_1}-1\right\}\mathrm{Pr}[e_5(b_1,b_2)],
\end{split}\end{align}
where $\mathrm{Pr}[e_5(b_1,b_2)]$ is expressed as
\begin{align}\begin{split}
  \mathrm{Pr}[e_5(b_1,b_2)]
  % =& \mathcal{Q}_{r_1}^2  \times \mathcal{Q}_{r_2}^2 
 %  \times (1-\mathcal{Q}_{r_1})^{i-j+k-1} 
 %  \times (1-\mathcal{Q}_{r_1}-\mathcal{Q}_{r_2})^{j-k-1}
 %  \times (1-\mathcal{Q}_{r_2})^{k-1} \\
 %  =&\mathcal{Q}_{r_1}^2  \times \mathcal{Q}_{r_2}^2 
 %  \times (1-\mathcal{Q}_{r_1})^{k-1} 
 %  \times (1-\mathcal{Q}_{r_1}-\mathcal{Q}_{r_2})^{-k-1}
 %  \times (1-\mathcal{Q}_{r_2})^{k-1} \\
 %  &\times (1-\mathcal{Q}_{r_1})^{i} \times \left(1-\frac{\mathcal{Q}_{r_2}}{1-\mathcal{Q}_{r_1}} \right)^{j}
  =& w_{r_1}^2  \times w_{r_2}^2 
  \times (1-w_{r_1})^{-i+j-k-1} 
  \times (1-w_{r_1}-w_{r_2})^{i-1}
  \times (1-w_{r_1})^{k-1} \\
  % =&w_{r_1}^2  \times w_{r_2}^2 
  % \times (1-w_{r_1})^{k-1} 
  % \times (1-w_{r_1}-w_{r_2})^{-k-1}
  % \times (1-w_{r_2})^{k-1} \\
  =&\psi_k(r_1,r_2)\times \left(1-\frac{w_{r_2}}{1-w_{r_1}} \right)^{i} \times (1-w_{r_1})^j,
\end{split}\end{align}
%
where
\begin{align}\begin{split}\label{eq:psi_k}
  \psi_k(r_1,r_2) 
  &= w_{r_1}^2 w_{r_2}^2 
    (1-w_{r_1})^{-k-1} 
    (1-w_{r_1}-w_{r_2})^{-1}
    (1-w_{r_1})^{k-1} \\
  &= w_{r_1}^2 w_{r_2}^2 (1-w_{r_1})^{-2} (1-w_{r_1}-w_{r_2})^{-1}.
\end{split}\end{align}
%
Then, Eq. (\ref{eq:Sk}) is expressed as
\begin{align}\begin{split}\label{eq:Sk_case5}
  % \mathcal{P}_k 
  % =& \sum_{i=1}^{\infty}\sum_{j=k+1}^{k+i-1} g(i)g(j) \sum_{r_1=0}^{L} \sum_{r_2 \neq r_1} \binom{L}{r_1}\binom{L}{r_2}\mathrm{Pr}[e_3(b_1,b_2)]\\ 
  % &+ \sum_{i=1}^{\infty}\sum_{j=k+1}^{k+i-1} g(i)g(j) \sum_{r_1=0}^{L} \sum_{r_2 \in \{r_1\}} \binom{L}{r_1} \left\{\binom{L}{r_1}-1 \right\}\mathrm{Pr}[e_3(b_1,b_2)]
  \overline{S}_k 
  =& \sum_{i=1}^{\infty}\sum_{j=k+i+1}^{\infty} g(i)g(j) \sum_{r_1=0}^{L} \sum_{r_2 \neq r_1} \binom{L}{r_1}\binom{L}{r_2}\mathrm{Pr}[e_5(b_1,b_2)]\\ 
  &+ \sum_{i=1}^{\infty}\sum_{j=k+i+1}^{\infty} g(i)g(j) \sum_{r_1=0}^{L} \sum_{r_2 \in \{r_1\}} \binom{L}{r_1} \left\{\binom{L}{r_1}-1 \right\}\mathrm{Pr}[e_5(b_1,b_2)].
\end{split}\end{align}
%
Firstly, we consider the first term in the r.h.s. of Eq. (\ref{eq:Sk_case3}). Let $\overline{B}_1$ be the first term of Eq. (\ref{eq:Sk_case3}). Then, $\overline{B}_1$ is written as
\begin{align}\begin{split}\label{eq:B_1}
  % \mathcal{S}_1
  % &=\sum_{i=1}^{\infty}\sum_{j=k+1}^{k+i-1} g(i)g(j) \sum_{r_1=0}^{L} \sum_{r_2 \neq r_1} \binom{L}{r_1}\binom{L}{r_2}\mathrm{Pr}[e_3(b_1,b_2)] \\
  % &=\sum_{r_1=0}^{L} \sum_{r_2 \neq r_1} \binom{L}{r_1}\binom{L}{r_2}\phi_k(r_1,r_2)
  % \sum_{i=1}^{\infty} \sum_{j=1}^{k+i-1} g(i)g(j)(1-\mathcal{Q}_{r_1})^{i} \times \left(1-\frac{\mathcal{Q}_{r_2}}{1-\mathcal{Q}_{r_1}} \right)^{j} \\
  % &=\sum_{r_1=0}^{L} \sum_{r_2 \neq r_1} \binom{L}{r_1}\binom{L}{r_2}\phi_k(r_1,r_2)
  % \sum_{j=k+1}^{\infty} \sum_{i=k+1}^{k+i-1} g(i)g(j)(1-\mathcal{Q}_{r_1})^{i} \times \left(1-\frac{\mathcal{Q}_{r_2}}{1-\mathcal{Q}_{r_1}} \right)^{j} \\
  % &=\sum_{r_1=0}^{L} \sum_{r_2 \neq r_1} \binom{L}{r_1}\binom{L}{r_2}\phi_k(r_1,r_2)
  % \sum_{j=k+1}^{\infty} \left\{ g(j) \left(1-\frac{\mathcal{Q}_{r_2}}{1-\mathcal{Q}_{r_1}} \right)^{j} \times \sum_{i=j-k+1}^{\infty} g(i)(1-\mathcal{Q}_{r_1})^{i} \right\} 
  \overline{B}_1
  &=\sum_{i=1}^{\infty}\sum_{j=k+i+1}^{\infty} g(i)g(j) \sum_{r_1=0}^{L} \sum_{r_2 \neq r_1} \binom{L}{r_1}\binom{L}{r_2}\mathrm{Pr}[e_5(b_1,b_2)] \\
  &=\sum_{r_1=0}^{L} \sum_{r_2 \neq r_1} \binom{L}{r_1}\binom{L}{r_2}\psi_k(r_1,r_2)
  \sum_{i=1}^{\infty} \sum_{j=k+i+1}^{\infty} g(i)g(j) \left(1-\frac{w_{r_2}}{1-w_{r_1}} \right)^{i} \times (1-w_{r_1})^j \\
  &=\sum_{r_1=0}^{L} \sum_{r_2 \neq r_1} \binom{L}{r_1}\binom{L}{r_2}\psi_k(r_1,r_2)
  \sum_{i=1}^{\infty} g(i)\left(1-\frac{w_{r_2}}{1-w_{r_1}} \right)^{i} \times \sum_{j=k+i+1}^{\infty} g(j) (1-w_{r_1})^j.
  % &=\sum_{r_1=0}^{L} \sum_{r_2 \neq r_1} \binom{L}{r_1}\binom{L}{r_2}\psi_k(r_1,r_2)
  % \sum_{j=k+1}^{\infty} \left\{ g(j) \left(1-\frac{w_{r_2}}{1-w_{r_1}} \right)^{j} \times \sum_{i=j-k+1}^{\infty} g(i)(1-w_{r_1})^{i} \right\} 
\end{split}\end{align}
In the course of the derivation of the above relations, the second equation has been obtained from Eq. (\ref{eq:phi_k}). The third equation has been obtained by exchange of infinite series.
%
Then, the infinite sum with respect to $j$ in the last equation of Eq. (\ref{eq:B_1}) can be calculated as
\begin{align}\begin{split}
  % \sum_{i=j-k+1}^{\infty} g(i)(1-\mathcal{Q}_{r_1})^{i} 
  % &= \sum_{i=1}^{\infty} g(i)(1-\mathcal{Q}_{r_1})^{i} - \sum_{i=1}^{j-k} g(i)(1-\mathcal{Q}_{r_1})^{i} \\
  % &= -\frac{1}{\ln 2} \times \frac{1-\mathcal{Q}_{r_1}}{\mathcal{Q}_{r_1}} \times \ln \mathcal{Q}_{r_1} - \sum_{i=1}^{j-k} g(i)(1-\mathcal{Q}_{r_1})^{i}.
  \sum_{j=k+i+1}^{\infty} g(j) (1-w_{r_1})^j 
  &= \sum_{j=1}^{\infty} g(j)(1-w_{r_1})^{j} - \sum_{j=1}^{k+i} g(j)(1-w_{r_1})^{j} \\
  &= -\frac{1}{\ln 2} \times \frac{1-w_{r_1}}{w_{r_1}} \times \ln w_{r_1} - \sum_{j=1}^{k+i} g(j)(1-w_{r_1})^{j}.
\end{split}\end{align}
In the above relations, we use the result of Lemma \ref{lemma:1}.
Hence, the first term of Eq. (\ref{eq:Sk_case5}) can be expressed as
\begin{align}
  % \mathcal{S}_1 
  % =& \sum_{r_1=0}^{L} \sum_{r_2 \neq r_1} \binom{L}{r_1}\binom{L}{r_2}\phi_k(r_1,r_2)\\
  % &\times\sum_{j=k+1}^{\infty} \left[ g(j) \left(1-\frac{\mathcal{Q}_{r_2}}{1-\mathcal{Q}_{r_1}} \right)^{j} \times \left\{ -\frac{1}{\ln 2} \times \frac{1-\mathcal{Q}_{r_1}}{\mathcal{Q}_{r_1}} \times \ln \mathcal{Q}_{r_1} - \sum_{i=1}^{j-k} g(i)(1-\mathcal{Q}_{r_1})^{i} \right\} \right].
  \overline{B}_1 
  =& \sum_{r_1=0}^{L} \sum_{r_2 \neq r_1} \binom{L}{r_1}\binom{L}{r_2}\psi_k(r_1,r_2)\\
  &\times\sum_{i=1}^{\infty} \left[ g(i) \left(1-\frac{w_{r_2}}{1-w_{r_1}} \right)^{i} \times \left\{ -\frac{1}{\ln 2} \times \frac{1-w_{r_1}}{w_{r_1}} \times \ln w_{r_1} - \sum_{j=1}^{k+i} g(j)(1-w_{r_1})^{j} \right\} \right].
\end{align}
We can derive the second term of Eq. (\ref{eq:Sk_case5}) in the same way as the first term. Let $\overline{B}_2$ be the second term of Eq. (\ref{eq:Sk_case5}). Then, we have
\begin{align}\begin{split}
  \overline{B}_2 =& \sum_{r_1=0}^{L} \sum_{r_2 \in \{r_1\}} \binom{L}{r_1}\left\{\binom{L}{r_1}-1\right\} \psi(r_1,r_2) \\
  &\times\sum_{i=1}^{\infty} \left[ g(i) \left(1-\frac{w_{r_2}}{1-w_{r_1}} \right)^{i} \times \left\{ -\frac{1}{\ln 2} \times \frac{1-w_{r_1}}{w_{r_1}} \times \ln w_{r_1} - \sum_{j=1}^{k+i} g(j)(1-w_{r_1})^{j} \right\} \right].
\end{split}\end{align}
%
Therefore, Eq. (\ref{eq:Sk_case5}) can be expressed as
\begin{align}\begin{split}
  \overline{S}_k =& \overline{B}_1 + \overline{B}_2 \\
  =& \sum_{r_1=0}^{L} \sum_{r_2 \neq r_1} \binom{L}{r_1}\binom{L}{r_2}\psi_k(r_1,r_2)\\
  &\times\sum_{i=1}^{\infty} \left[ g(i) \left(1-\frac{w_{r_2}}{1-w_{r_1}} \right)^{i} \times \left\{ -\frac{1}{\ln 2} \times \frac{1-w_{r_1}}{w_{r_1}} \times \ln w_{r_1} - \sum_{j=1}^{k+i} g(j)(1-w_{r_1})^{j} \right\} \right] \\
  &+\sum_{r_1=0}^{L} \sum_{r_2 \in \{r_1\}} \binom{L}{r_1}\left\{\binom{L}{r_1}-1\right\} \psi(r_1,r_2) \\
  &\times\sum_{i=1}^{\infty} \left[ g(i) \left(1-\frac{w_{r_2}}{1-w_{r_1}} \right)^{i} \times \left\{ -\frac{1}{\ln 2} \times \frac{1-w_{r_1}}{w_{r_1}} \times \ln w_{r_1} - \sum_{j=1}^{k+i} g(j)(1-w_{r_1})^{j} \right\} \right],
\end{split}\end{align}
where $\psi_k(r_1,r_2)$ is given in Eq. (\ref{eq:psi_k}).
% \begin{align}
%   \phi_k(r_1,r_2) = w_{r_1}^2 w_{r_2}^2 (1-w_{r_1})^{k-1} (1-w_{r_1}-w_{r_2})^{-k-1} (1-w_{r_2})^{k-1}.
% \end{align}


















