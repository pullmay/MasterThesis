This thesis is organized as follows. 
%
\par
%
In Section \ref{sec:universal}, we introduce the statistical tests proposed by Maurer in 1992 \cite{maurer1992universal} and by Coron in 1999 \cite{coron1999security}. These statistical tests are referred to as ``Maurer's universal test'' and ``Coron's universal test'', respectively. 
Coron's universal test is an improvement of Maurer's universal test.
%
We also introduce the statistical test proposed by Yamamoto and Liu in 2016 \cite{yamamoto2016highly}. The test is referred to as ``highly sensitive universal statistical test'' and the test is constructed on the basis of Maurer's universal test and Coron's universal test.
%
In the highly sensitive test, a tested sequence is converted to another binary sequence as each bit of a tested sequence is stochastically flipped under a certain distribution.
%
It is suggested in \cite{yamamoto2016highly} that the converting make the test more sensitive.
%
\par
In Section \ref{sec:distribution}, we derive one and two dimensional distributions. We need these distribution to derive the variance for the reference distribution of the highly sensitive test. In existing literature, the theoretical results for a truly random sequence without flipping have been obtained. However, we cannot apply the results directly to the highly sensitive test since a tested sequence is biased.
%
\par
In Section \ref{sec:3}, we derive the theoretical variance for the reference distribution of the highly sensitive test. 
We also show some results of experiments in this section. 
Firstly, we show the variance can be computed accurately by the derived equation. 
Secondly, we show the fitted curve for computing the variance for effectively. 
Thirdly, we show the difference between the highly sensitive test with proposed parameter and the test with the existing value. 
%
\par
In Section \ref{sec:conclusion}, we conclude this thesis.