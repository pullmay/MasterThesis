This thesis is organized as follows. 
%
%Section2: Universal Statistical Test
\par
%
In Section \ref{sec:universal}, we introduce the statistical tests proposed by Maurer in 1992 \cite{maurer1992universal} and by Coron in 1999 \cite{coron1999security}. These statistical tests are referred to as ``Maurer's universal test'' and ``Coron's universal test'', respectively. 
%Maurer's test と Coron's test の関係は?
Coron's universal test is an improvement of Maurer's universal test.
%
We also introduce the statistical test proposed by Yamamoto and Liu in 2016 \cite{yamamoto2016highly}. The test is referred to as ``highly sensitive universal statistical test'' and the test is constructed on the basis of Maurer's universal test and Coron's universal test.
% We also introduce ``Highly sensitive universal statistical test'' proposed in 2016 \cite{yamamoto2016highly}. The highly sensitive test is constructed on the basis of Maurer's universal test and Coron's universal test. 
%主な相違点は,検定対象の系列をそのまま使うのではなく,あえて偏りを持たせて検定する点にある.こうすることによって,偏りが検出しやすいということを動機としている.
% The main difference lies in the way of using a binary sequence being tested. In Maurer's universal test and Coron's universal test (and most of the statistical tests for evaluating a randomness), a tested sequence are used without change. On the other hand, in highly sensitive universal test, a tested sequence are converted into another binary sequence as a probability of taking each bit with some probability.
%
In the highly sensitive test, a tested sequence is converted to another binary sequence as each bit of a tested sequence is stochastically flipped under a certain distribution.
%
% In general, most of the statistical tests for evaluating a randomness consider to use the sequence with no change, whereas the highly sensitive test do not use a binary sequence as it is. 
% It is suggested in \cite{yamamoto2016highly} that the non-randomness can be detected much more sensitive than 
% Maurer's and Coron's universal tests.
It is suggested in \cite{yamamoto2016highly} that the converting make the test more sensitive.
%
%Section3: Distribution of A_n that is necessary to derive the variance for the reference distribution of highly sensitive test.
%導出するのは,A_nに関する分布である.ここで,Aというのは出てきていないから,integer valued variableとでも言うしかないか.
%このA_nは何かと言うと,要は系列から算出される値である.A_nの分布はHSTにおける帰無仮説のもとでの参照分布を導出するために必要.
%truly random sequence 
\par
In Section \ref{sec:distribution}, we derive one and two dimensional distributions. We need these distribution to derive the variance for the reference distribution of the highly sensitive test. In existing literature, the theoretical results for a truly random sequence without flipping have been obtained. However, we cannot apply the results directly to the highly sensitive test since a tested sequence is biased.
%
% In Section \ref{sec:distribution}, we derive the marginal distribution and joint distribution of integer valued variable whose value is determined by a binary sequence. The distribution is necessary to derive the variance for the reference distribution of highly sensitive universal test under the null hypothesis. In existing literature, the distribution for a truly random sequence are studied and has been obtained theoretical value, however, the result cannot apply directly to highly sensitive universal test since a tested sequence is biased.
% 
%Section4: Deriving the variance for the rectangle
%参照分布の分散を理論的に導出する.先行研究では,擬似乱数を用いた実験によって得られた数値を検定で用いている.
\par
% In Section \ref{sec:3}, we derive the variance for the reference distribution of highly sensitive universal test under the null hypothesis theoretically. 
In Section \ref{sec:3}, we derive the theoretical variance for the reference distribution of the highly sensitive test. 
% In existing literature, a value obtained by a numerical experiment using a pseudo random number generator has been used.
%
%実験結果について(順番を再検討するべき)
We also show some results of experiments in this section. 
Firstly, we show the variance can be computed accurately by the derived equation. 
Secondly, we show the fitted curve for computing the variance for effectively. 
Thirdly, we show the difference between the highly sensitive test with proposed parameter and the test with the existing value. 
%
\par
In Section \ref{sec:conclusion}, we conclude this thesis.