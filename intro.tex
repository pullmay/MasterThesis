Throughout this thesis, we only consider a binary sequence.
%
A random sequence is intuitively considered as a number sequence without any recognizable patterns or regularities, and it is not easy to define such properties mathematically.
The reader may consider a sequence to be random if its each bit is independent and symmetrically distributed, but this description is inconsistent with an intuitive definition. When we intuitively say ``random sequence'', this term is used to represent a specific sequence, not to represent a sequence of random variables.
\par
Several approaches to define a random sequence have been proposed.
In Algorithmic information theory, several definitions have been proposed based on Kolmogorov 
complexity \cite{kolmogorov1968three,chaitin1966length,chaitin1969length,chaitin1975theory}. 
Kolmogorov complexity of a finite sequence is defined as the minimal length of a program which generates the sequence with a given universal machine. Let $s$ be a finite sequence and $u$ be a universal machine. Then, Kolmogorov complexity of $s$ for $u$, $K_u(s)$ is written as 
\begin{align}
	K_u(s):= \min_{p:u(p)=s} r(p),	
\end{align}
where $p$ is a program that generates $s$ by $u$ and $r (p)$ is the length of the program $p$. 
%
Then, a finite binary sequence is regarded as random if its Kolmogorov complexity is almost equal to its length. In other words, a finite binary sequence which cannot be compressed is random. 
%
We can define randomness for given finite sequences by its notion, however, it is shown to be impossible to compute Kolmogorov complexity. Furthermore, this complexity depends on a choice of universal machine.
Other definitions for randomness which is associated with the definition by Kolmogorov complexity have been proposed by Demuth \cite{demuth1988remarks}, Martin-L\"{o}f \cite{martin1966definition,martin1971complexity} and Schnorr \cite{schnorr1971unified,schnorr1973process,schnorr1977general}.
\par
In cryptography, an approach based on a notion of indistinguishability has been proposed. 
Let $\{X_n\}_{n\geq 1}$ and $\{Y_n\}_{n\geq 1}$ be sequences of random variables. 
%
We say $\{X_n\}_{n\geq 1}$ and $\{Y_n\}_{n\geq 1}$ are computationally indistinguishable if for any probabilistic polynomial time algorithm $\mathcal{A}$ and positive polynomial $p$, there exists $k_0$ such that 
\begin{align}\label{eq:indistinguishable}
	\bigl| \mathrm{Pr}[\mathcal{A}(X_k) = 1] - \mathrm{Pr}[\mathcal{A}(Y_k) = 1] \bigr| < \frac{1}{p(k)},
\end{align}
for any $k>k_0$.
With this computational indistinguishability, we can define a cryptographically secure pseudo random number generator (CSPRNG). 
Let $g_n$ be a map from $\{0,1\}^n$ to $\{0,1\}^{h(n)}$, where $h(n)$ is a polynomial of $n$.
We say that a sequence of maps $g =\{g_n\}_{n\geq 1}$ is CSPRNG if this sequence satisfies the following three properties:
\begin{itemize}
	\item The relation $n < h(n)$ holds for any $n\geq 1$.
	\item For any $n\geq 1$ and input $x\in \{0,1\}^n$, there exists a deterministic algorithm for computing $g_n(x)$ in polynomial time depending on $n$.
	\item $\{g(U_n)\}_{n\geq 1}$ and $\{U_{h(n)}\}_{n\geq 1}$ are computationally indistinguishable, where $U_n$ is a random variable uniformly distributed on $\{0,1\}^n$.
\end{itemize}
Pseudo random numbers in cryptography are referred to as sequences generated by a CSPRNG. 
Under some assumptions, several algorithms for CSPRNG such as Blum--Blum--Shub algorithm \cite{blum1986simple} and Blum--Micali algorithm \cite{blum1984generate} have been proposed. 
However, the definition based on computationally indistinguishability also has obstacles such as:
\begin{itemize}
	\item For an arbitrary finite sequence, there exist CSPRNG and $x\in\{0,1\}^\ast$ such that the CSPRNG outputs the sequence when $x$ is input. Thus, it is meaningless to define ``randomness'' for a specific sequence by the above definition.
	\item It is impossible to verify whether a sequence satisfies Eq. (\ref{eq:indistinguishable}) for any probabilistic polynomial time algorithm.
	\item There exists no sequence satisfying the definition if $\mathrm{P}=\mathrm{NP}$.
\end{itemize}
To summarize, it is not easy to define randomness to a specific finite sequence.
%
\par
In spite of the difficulty of defining randomness, ``random sequences'' are widely used in many fields such as numerical simulations (e.g., Monte Carlo method), randomized algorithm (e.g., Simulated Annealing), and cryptography (e.g., key generation). 
In engineering applications, a sequence generated by a hardware (or physical) random number generator (HRNG) or a pseudo random number generator (PRNG) are extensively used.
%
Note that PRNG does not mean CSPRNG. 
An HRNG is a device for generating numbers from a physical process such as thermal noise in a transistor, whereas a PRNG is a deterministic algorithm for generating numbers.
%
In general, it is not easy to predict the output generated by an HRNG. On the other hand, a PRNG is important in practice because it generates a sequence much faster than an HRNG and a sequence can be reproductive by using the same initial value called seed. These advantages are useful in many situations such as numerical experiments.
%
However, both generators are nothing but generate a number sequence regarded as a random sequence approximately which is required for applications.
Hence, such sequences are necessary to be examined whether it satisfy the properties as ``random sequences'' or not.