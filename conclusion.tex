%highly sensitive testの帰無仮説の下での参照分布を理論的に導出した.
In this thesis, we have theoretically derived the variance for the reference distribution of highly sensitive universal statistical test. 
%
In deriving process, the marginal distribution of $A_n$ and the joint distribution of $(A_n,A_{n+k})$ have been derived theoretically with the fact that the sequence of $\{A_k\}_{k=1}^{K}$ can be considered as stationary ergodic under the assumption of $Q\to\infty$. 
%
We have shown that the value of the variance can be numerically computed for the block size $L=4$.
%
Because of computational cost, we used an fitted curve to get the numerical value of the variance for $L=8$.
%
% We also have derived the approximated curve in order to obtain the variance when $L=8$. 
Since the value obtained by the fitted curve may have some error, we have compared with the unbiased variance computed from binary sequences generated by a pseudo random number generator. By this experiment, we have confirmed that the value obtained from the fitted curve is more consistent with the experimental result than the existing value which has been obtained by a numerical experiment.
%
We can state that the value of the variance in this thesis is superior subject to the existing one. 
%
Thus, we recommend that the value obtained in this thesis should be used when the highly sensitive universal statistical test is performed so that randomness of binary sequences can be tested more precisely.