A random sequence is intuitively considered as a number sequence having no any recognizable patterns or regularities. 
しかし,数学的にそう言う性質を定義するのはなかなか難しい.
なお,弊修論ではbunary sequenceのみを考察の対象とする.
その場合,ある人は以下のように言うかもしれない: a sequence whose each bit is independent and symmetrically distributed is considered as a random sequence. 
しかし,我々が直感的に「乱数」と言う場合,それは具体的な数列に対して使われる言葉であり,確率変数列に対して使う用語ではないので,問題はもっと複雑じゃ.

%
So far, several approaches to define a random sequence have been proposed and studied the relationship between these definitions. 


計算論においてはKolmogorov complexityをベースにした定義が複数提案されている.
The definition based on Kolmogorov complexity is one of the most important definition which was proposed by Kolmogorov \cite{kolmogorov1968three} and Chaitin\cite{chaitin1966length,chaitin1969length,chaitin1975theory}. 
%
In this approach, the randomness of a sequence is evaluated by the Kolmogorov complexity. Note that Kolmogorov complexity of a finite string is defined to be the minimal program size that generates the string. Then, a finite sequence is considered to be random if its Kolmogorov complexity is almost equal to its length.
%
weakly chaitin random と包含関係にある定義として
\begin{itemize}
  \item Demuth randomness \cite{demuth1988remarks}
  \item Martin-L\"{o}f randomness \cite{martin1966definition,martin1971complexity}
  \item Schnorr randomness \cite{schnorr1971unified,schnorr1973process,schnorr1977general}
  \item Kurtz randomness \cite{kautz1991degrees}
  % \item Kurtz randomness
\end{itemize}
などがある.

ただし,複雑度は計算不能.
しかも,プログラムを実行するコンピュータに依存する.など

暗号の文脈では,識別不能という概念で乱数にアプローチする.
(識別不能・擬似乱数生成器・擬似乱数の説明)
しかし,こいつにも問題点がある.
・具体的な有限長の数列に対して乱数を定義しているわけではない
・任意の多項式時間アルゴリズムを相手にするのは現実的には不可能
・P=NPならそもそも定義を満たすものがなくなってしまう.

結論として,
・具体的な有限長の数列に対して乱数を定義することは難しい
%
\par
なんだけれども,
Random sequences are widely used in many fields such as numerical simulations (e.g., Monte Carlo method), randomized algorithm (e.g., Simulated Annealing), and cryptography (e.g., key generation). 
工学的には物理乱数生成器や擬似乱数生成器(暗号でいうところの擬似乱数生成器ではない)で生成された数列を乱数として使うお.
物理乱数生成器・擬似乱数生成器とは
There are mainly two types of methods for generating a random sequence. One is a hardware (or physical) random number generator (HRNG) that is a device for generating random numbers from a physical process such as thermal noise in a transistor. The other is a pseudo random number generator (PRNG) that is a deterministic algorithm for generating a sequence of numbers. 
In general, 物理乱数生成器の出力は予測しにくい.
It may seem a HRNG is better random number generator, however, a PRNG is important in practice because it generates a sequence faster than a HRNG, and a sequence can be reproductive by using the same seed. These advantages are useful in many situations such as numerical simulations. 
ただ,結局,こいつらは応用先で求められる乱数としての性質を近似的にもつ数列を生成するものにすぎない.
しかも,ちゃんと近似しているのか,チェックが必要.