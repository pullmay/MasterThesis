As seen in the previous subsection, it is a hard task to evaluate whether a finite binary sequence satisfies ``randomness'' or not. In practice, a statistical hypothesis test has been extensively considered to evaluate the randomness of a binary sequence. A number of statistical tests has been proposed. In most of the cases, the randomness is evaluated by multiple statistical tests since one statistical test is designed to detect the specific defect of a binary sequence and cannot detect other types of defects.
%
\par
There are some test suites such as TestU01 test suite \cite{l2007testu01}, the BSI (Bundesamt fr Sicherheit in der Informationstechnik) test suite \cite{schindler1999functionality,killmann2001proposal}, Marsaglia's DIEHARD test suite, Crypt-X statistical test suite \cite{caelli1992crypt} and FIPS 140-2 test suite \cite{fips2001140}.
\par
%
NIST Special Publication 800-22 (NIST SP 800-22) \cite{rukhin2001statistical,bassham2010sp} proposed by National Institute of Standards and Technology (NIST) is one of the standard statistical test suites that was originally used for selecting Advanced Encryption Standard (AES) \cite{rijmen2001advanced}. 
%
NIST SP 800-22 consists of fifteen kinds of statistical tests, and the null hypothesis $\mathcal{H}_0$ is that a given binary sequence is truly random. We can regard a $n$-bit given binary sequence as a sample from a uniform distribution on $\{0,1\}^n$.
Associated with the null hypothesis, the alternative hypothesis $\mathcal{H}_1$ is that the sequence is not random.
NIST SP 800-22 specifies the evaluating process as follows.
For a tested binary sequence of length $n$, a p-value is computed. If p-value is equal or larger than $\alpha$, the null hypothesis $\mathcal{H}_0$ is accepted, where $\alpha$ is the significance level of the test. Repeat the same procedure for $m$ sample sequences and obtain $m$ p-values. 
%
NIST SP 800-22 recommends to perform additional statistical tests for the $m$ p-values.
The following two tests are specified under the null hypothesis that $m$ p-values independently follow a uniform distribution in $[0,1]$.
\begin{enumerate}
  \item (Proportion test) Let $m_p$ be the number of sequences whose p-value satisfies p-value $\geq \alpha$ for the given $m$ sequences. The null hypothesis is rejected if $m_p$ lies outside the significant interval $[m(1-\alpha)-\xi\sigma, m(1-\alpha)+\xi\sigma]$, where $m(1-\alpha)$ and $\sigma = \sqrt{m\alpha(1-\alpha)}$ is the expected value and  standard deviation of $m_p$, respectively.
  %
  \item (Uniformity test) The distribution of the $m$ p-values against the uniform distribution on $[0,1]$ is tested with a Chi-Square goodness of fit test in $k$ bins. This is again a statistical test, which yields a level-two p-value $p_T$. Given a significance level $\alpha_T$, the null hypothesis is rejected if $p_T \leq \alpha_T$.
\end{enumerate}
%
NIST SP 800-22 recommends to choose parameters as $m=1000,\,\alpha=0.01,\,\xi=3, \,k=10$ and $\alpha_T=0.0001$.
